% ****** Start of file apssamp.tex ******
%
%   This file is part of the APS files in the REVTeX 3.0 distribution.
%   Version 3.0 of REVTeX, November 10, 1992.
%
%   Copyright (c) 1992 The American Physical Society.
%
%   See the REVTeX 3.0 README file for restrictions and more information.
%
%
%
\documentstyle[prd,eqsecnum,aps,floats]{revtex}
\input psfig
%\input feynman
\def\D0{D\O}

\begin{document}
%
%\small
\begin{flushright}
A.M.Jonckheere

8-Aug-1987

Revised - 16-Feb-1989
\end{flushright}
\vspace{1.0in}
\centerline{***** CALTOWER.MEM *****}
\centerline{ V2.0 }

\section{Introduction}

This note briefly describes the operation of the routines contained in
\D0\$\D0GEANT's Group Caltower. These routines calculate and store Calorimeter
energies in Cells and Towers. We also describe some routines in other libraries
used by the Caltower routines that may be used to extract the information
from the /ZEBCOM/ banks for analysis.

When turned on, the Caltower routines accumulate the energy deposited in each
Calorimeter readout cell for a given primary track in a large array in
common /CTWRK/. After a track is finished, the energies in /CTWRK/ are smeared
and added to any energy from previously stored tracks in the Zebra banks CAEP in
/ZEBCOM/. Live material data is stored in one bank, dead material in another.
These banks are tagged by the bank version number. The dead material banks have
version numbers above 1000. When one uses GZCAEP() to find the CAEP bank, one
will normally find the live material bank. If desired, Idealised HITS (Banks
JETS) may also be calculated. The individual, unsmeared track energies may be
saved in the linear bank structure GCAH. The banks in the GCAH structure are
identified by it's bank number set equal to the Geant track number. If they are
saved 19 words of additional data, including ISAJET and GEANT data are saved as
well. At the end of an event, the CAEP ``hits'' bank can be converted into the Raw
Data format and stored in the CAD1/2 banks. In addition the ICD Hits information
can be save in bank ICDH (under GHIT). This last, does not depend on
CALTOWERs, but if CALTOWERs is used a subset of the information (Light output)
is stored in CAEP and GCAH as well.

\section{Control of Operation}

The operation of the Caltower routines is controlled by the values given to
various members of the user array SCAL during the initial FFREAD Dialog. The
format of this command is:

\begin{verbatim}

    SCAL a b c d e f    where a... are floating point numbers.

SCAL(1) = 0            No Caltowers,
        = 1 (default)  Use software Caltowers
        = 2            Create Shower library
    (2) = 0 (default)  Drop individual Unsmeared Track energies in GCAH banks
        > 0            Save Unsmeared Track energies in GCAH
        > 1            Save Total & Hadron only energies separately
    (3) = 0 (default)  Drop CAEP banks calculated in Caltowers
        > 0            Calculate and save summed, smeared energies in CAEP
    (4) = 0            No JETS bank calculated
        > 0 (default)  Calculate Idealised hits, JETS banks
    (5) > 0            Calculate Raw data banks (CAD1/2) with zero suppression
                       on cells with   ABS(ENERGY) < SCAL(5) in GeV.
            (default = 1.0E-3)
        = 0            Calculate FULL Raw data banks (CAD1/2)
        < 0            Do not calculate Raw data banks (CAD1/2)
    (6) = 0 (default)  Drop ICD Hits bank (ICDH) before saving
        > 0            Keep ICD Hits bank (ICDH)
\end{verbatim}

SCAL(1) MUST be equal to 1 for any of the others to be active (except SCAL(6)),
but any combination of the others is possible. If $SCAL(5)>=0$ and $SCAL(3)=0$,
then the CAEP banks are created, but dropped after the CAD banks are formed.
SCAL(5) is used both as a switch and as real number threshold for zero
suppression in the CAD banks. For example, $SCAL(5)=1.0E-3$ would create zero
suppressed CAD banks containing only cells with ABS(ENERGY) greater than 1 MeV.

The data is output with the rest of the /ZEBCOM/ data at the end of
each event only if ``ZEBIO n m'' is specified in the initial dialog and m is
non-zero.

\section{Data Retrieval}

The major changes that are being made, effective with release V2.0 of
\D0Geant are that Readout Cells are now indexed by their Offline or Physics
indices and that the energies are now given in GeV in floating point form. See
\D0~NOTE~774 by Jim Linnemann for details on the Physics indices.

The one exception to Jim's \D0~Note is that for the Monte Carlo more layers have
been added to store energy lost in various dead material. The extra layers are
listed graphically in Appendix~A. This appendix is the equivalent of Appendix~A
of the \D0~Note.

The extraction of data from the appropriate banks for analysis is not
complicated. The following code fragments illustrate the steps needed.

To extract smeared data from the CAEP banks for analysis:

\begin{verbatim}

        INCLUDE 'D0$INC:ZEBCOM.INC'     ! Main D0 Data Store
        INTEGER LCAEP,GZCAEP            ! Zebra links and routines
        INTEGER NDATA,NCELL             ! Data lengths
        INTEGER IADDR                   ! Packed Cell Address
        INTEGER IETA,IPHI,LAYER,ITAG    ! Unpacked Cell Address
        REAL    EDATA                   ! Cell Energy
                . . .

        LCAEP = GZCAEP()                ! Get link to live material CAEP bank
     10 IF ( LCAEP.LE.0 ) GOTO 200      ! Error - No data
        NDATA = IQ(LCAEP-1)             ! Number of data words
        NCELL = IQ(LCAEP+3)             ! Number of cells with data
        IF ( 2*NCELL+3.NE.NDATA ) GOTO 998      ! Error - lengths don't agree
        DO 100 I = 1, NCELL             ! Loop over cells
C
          IADDR = IQ(LCAEP+2*(I-1)+4)   ! Packed Addr in Physics Indices
          IETA  = IBITS(IADDR,24,8)     ! High order byte is IETA
          IF ( IETA.GE.128 ) THEN
            IETA = IETA - 256           ! extend sign
          ENDIF
          IPHI  = IBITS(IADDR,16,8)     ! Next byte is IPHI
          LAYER = IBITS(IADDR,8,8)      ! Next byte is LAYER
          ITAG  = IBITS(IADDR,0,8)      ! Low order is Tag bits
C
          EDATA =  Q(LCAEP+2*(I-1)+5    ! Smeared Energy in GeV
              . . .                     ! Process

    100 CONTINUE
C
C **** Loop over 2d CAEP bank (dead material), this is present ONLY if data is
C ****    from Geant (obviously)
C
        LCAEP = LQ(LCAEP)               ! 2d bank in linear structure
        GOTO 10
    200 CONTINUE
              . . .

\end{verbatim}

The Unsmeared data is more complicated to extract, so routines are supplied to
do most of the work. The following code fragment illustrates the necessary
steps.

\begin{verbatim}

        ITRAK = 1                       ! Init the Track index
C  Get track data
    100 CALL GTGCAH(ITRAK,ITRA,VERT,P,IDATA,TOTAL,NPOINT)
        IF ( ITRA.LT.0 ) GOTO 999       ! No more tracks
        ITRAK = ITRA                    ! Actual track found
                . . .                   ! Process Track totals data

        DO 200 I = 1, NPOINT            ! Loop over data points
C  Get energy and indices for ONE cell
          CALL GEGCAH(ITRAK,I,CELLE,IETA,IPHI,LAYER,ITAG)
          IF ( IETA.NE.0 ) THEN
                . . .                   ! Process
          ELSE
                . . .                   ! Error
          ENDIF
    200 CONTINUE
        ITRAK = ITRAK + 1               ! Next track
        GOTO 100                        ! Get next track

\end{verbatim}

The data returned from GTGCAH and GEGCAH are described below.

\begin{verbatim}

      SUBROUTINE GTGCAH(ITRAK,ITRA,VERT,P,IDATA,TOTAL,NPOINT)
C----------------------------------------------------------------------
C-
C-   Purpose and Methods : Finds GCAH bank for first track number .GT. ITRAK
C-                              returns track data
C-
C-   Inputs  : ITRAK    = Lowest Geant track number to look for
C-   Outputs : ITRA     = Track actually found (-1) if no more available
C-             VERT(3)  = X,Y,Z of track vertex
C-             P(4)     = ISAJET track 4-momentum
C-             IDATA(1) = Geant track type
C-                  (2) = Geant origin vertex #
C-                  (3) = Geant parent vertex #
C-                  (4) = Geant parent track #
C-                  (5) = Reason for entry               Geant KCASE
C-                      = 1   -> ISAJET track                - -
C-                      = 11  -> decay product              'DCAY'
C-                      = 12  -> hadronic interaction       'HADR'
C-                      = 13  -> Muon-nuclear interaction   'MUNU'
C-                      = 14  -> Photo fission              'PFIS'
C-                      = 15  -> Pair production            'PAIR'
C-                      = 16  -> Compton scattering         'COMP'
C-                      = 17  -> Photo production           'PHOT'
C-                      = 18  -> Annihilation               'ANNI'
C-                      = 21  -> Punchthrough                - -
C-                      = 22  -> Bremstrahlung              'BREM'
C-                      = 23  -> Delta ray electron         'DRAY'
C-                      = 999 -> Stopping end vertex         - -
C-                  (6) = Isajet parent vertex #
C-                  (7) = Isajet track #
C-             TOTAL(1) = Calorimeter energy UCAL + ECAL
C-                  (2) = Energy in Massless Gaps
C-                  (3) = Scintillator Hit Total (GeV)
C-                  (4) = Energy in cracks + cryostat + end plates
C-             NPOINT   = Number of Two word data entries following
C- **** The Address and Energy data are NOT returned by this routine. ****
C-    For V0.10 and earlier versions of D0Geant:
C-              CALL GCELLE(ECELL,IZ,IE,IP,IL,EOD)
C-    to retrieve the energies - cycle through until EOD = .TRUE.
C-              ECELL is an INTEGER energy in KeV
C-
C-     For V0.11 to V1.7 use:
C-              CALL GTENXT(ITRAK,IPOINT,ECELL,IZ,IETA,IPHI,LAYER,EOD)
C-     to retrieve the IPOINT'th entry
C-              1 <= IPOINT <= NPOINT
C-              ECELL is an INTEGER energy in KeV
C-
C-      For V2.0 and later versions use:
C-              CALL GEGCAH(ITRAK,IPOINT,CELLE,IETA,IPHI,LAYER)
C-      to retrieve the IPOINT'th entry
C-              1 <= IPOINT <= NPOINT
C-              CELLE is an REAL energy in GeV
C-
C-   Created  12-MAY-1987   Alan M. Jonckheere
C-   Updated  16-FEB-1989   Alan M. Jonckheere
C-
C----------------------------------------------------------------------



      SUBROUTINE GEGCAH(ITRAK,IPOINT,CELLE,IETA,IPHI,LAYER,ITAG)
C----------------------------------------------------------------------
C-
C-   Purpose and Methods : Return cell energy and unpacked Physics cell indices
C-                              for IPOINT'th cell of ITRAK track
C-           **** Valid for D0Geant V2.0 and above ****
C-
C-   Inputs  : ITRAK  = Bank (track) number to access
C-             IPOINT = Sequence number of cell to return
C-   Outputs : CELLE  = Cell energy (REAL) in GeV
C-             IETA   = Physics system Eta index (-37:37)
C-             IPHI   = Physics Phi index (1:64)
C-             LAYER  = Physics Layer index (augmented) (1:27)
C-             ITAG   = Tag bits
C-
C-   Created  16-FEB-1989   Alan M. Jonckheere
C-
C----------------------------------------------------------------------
\end{verbatim}

\newpage
\section{Additional Routines}

Additional routines that may be of interest are:
\begin{verbatim}

DSCHIT.FOR      Called in STPCAL and in GTNINO instead of GSCHIT
DCHIT.FOR       Calculates Crack hits - layers 18-27
DEHIT.FOR       Calculates EC hits
DSHIT.FOR       Calculates Scintillator (between CC and EC) hits - layer 9
DUHIT.FOR       Calculates UC hits
DHSTOR.FOR      Stores hits for all detectors into /CTWRK/

CALTRK.FOR      Called in GUSTEP after each track (including temp STAK tracks)
CALBLD.FOR      Called in CALTRK after each primary track - smears data and
                    moves it into /ZEBCOM/

CACMPR.FOR      Calculates cell indices in region where .1 x .1 pads -> ?
SMRFAC.FOR      Function returns smearing factor

BKGCAH.FOR      Book GCAH banks
FLGCAH.FOR      Fill GCAH bank with Isajet + Geant data for each track
PRGCAH.FOR      Print GCAH bank
PRCAEP.FOR      Print CAEP banks

GTGCAH.FOR      Find track bank, return header data
GZGCAH.FOR      Return Zebra link to GCAH bank for track ITRAK
GEGCAH.FOR      Return Energy and Cell indices - unsmeared for one track
GZCAEP.FOR      Return Zebra link to the first (live) CAEP bank

\end{verbatim}
\newpage

\centerline{Appendix A}

\begin{verbatim}

    The layers in each eta bin in the OFFLINE system are listed below:


 eta      IETAC           Layers included
 range
                     E             C I E F       C
                     M             C C C H       H
                                   M D M
                                   G   G
                                       1 1 1 1 1 1 1 1 1 1 2 2 2 2 2 2 2 2
                     1-2-3-4-5-6-7-8-9-0-1-2-3-4-5-6-7-8-9-0-1-2-3-4-5-6-7
                                             <-CC |EC | Dead->
   0   .1     1      x x x x x x x       x x x   x|   |d   d   d d d d
  .1   .2     2      x x x x x x x       x x x   x|   |d   d   d d d d
  .2   .3     3      x x x x x x x       x x x   x|   |d   d   d d d d
  .3   .4     4      x x x x x x x       x x x   x|   |d   d   d d d d
  .4   .5     5      x x x x x x x       x x x   x| O |d   d   d d d d d d
  .5   .6     6      x x x x x x x       x x x   x| H |d d d d d d d d d d
  .6   .7     7      x x x x x x x  _____x x x  __| | |d d d d d d d d d d
  .7   .8     8      x x x x x x x x| |x|x x x  |x  v |d d d d d d d d d d
  .8   .9     9      x x x x x x x x|x|x|x x    |x x  |d d d d d d d d d d
  .9  1.0    10   ^  x x x x x x x x|x|x|x______|x x  |d d d d d d d d d d
 1.0  1.1    11   |  x x x x x x x x|x|x/x\_     x x x|d d d d d d d d d d
 1.1  1.2   +12*  C  x x x x       x|x/x x x\    x x x|d d d d d d d d d d
-1.1 -1.2   -12*  C  x x     x x   x|x|x x x \__ x x x|d d d d d d d d d d
 1.2  1.3    13   _  _______________|x|x x x x x\_ x x|d d d d d d d d d d
 1.3  1.4   +14*             x x x  |x|  x x x x x\  x|  d   d d   d   d d
-1.3 -1.4   -14*  E      x x     x  |x|  x x x x x \ x|  d   d d   d   d d
 1.4  1.5    15   C  x x x x x x x  |_|\ x x x x x =\g|  d   d         d d
 1.5  1.6    16   |  x x x x x x x      \x x x x x   \|  d   d         d d
 1.6  1.7    17   v  x x x x x x x       x\x x x x    |  d   d         d d
 1.7  1.8    18      x x x x x x x       x x\_ x x<-MH|  d   d         d d
 1.8  1.9    19      x x x x x x x       x x x\_ x    |  d   d         d d
 1.9  2.0    20      x x x x x x x       x x x x\x    |  d   d         d d
 2.0  2.1    21      x x x x x x x       x x x x x\___|  d   d         d d
 2.1  2.2    22      x x x x x x x       x x x x x    |  d   d         d d
 2.2  2.3    23      x x x x x x x       x x x x x    |  d   d         d d
 2.3  2.4    24      x x x x x x x       x x x x x    |  d   d         d d
 2.4  2.5    25      x x x x x x x       x x x x x    |  d   d         d d
 2.5  2.6    26      x x x x x x x       x x x x x<-IH|  d   d         d d
 2.6  2.7    27      x x x       x       x x x x x    |  d   d         d d
 2.7  2.8    28      x x x       x       x x x x x    |  d   d         d d
 2.8  2.9    29      x x x       x       x x x x x    |  d   d         d d
 2.9  3.0    30      x x x       x       x x x x x    |  d   d         d d
 3.0  3.1    31      x x x       x       x x x x x    |  d   d         d d
 3.1  3.2    32      x x x       x       x x x x x    |  d   d         d d
 3.2  3.42   33      x x x       x       x x x x x    |  d   d         d d
 3.42 3.7    34      x x x       x       x x x x x    |  d   d         d d
 3.7  4.1    35      x x x       x       x x x x x    |  d   d         d d
 4.1  4.45   36                          x x x x x    |  d   d         d d
 4.45        37                              x x x       d   d         d d

    * NOTE: These 2 IETAC bins are repeated for +-Z (+-eta)
\end{verbatim}
\newpage
\begin{verbatim}

    From IETAC = 27 to 35, EM3 is not subdivided, thus only LAYERC = 3 exists.
    From IETAC = 33 onwards, only odd values of IPHIC exist.

     Dead Material:
      Layer 18 = Module Cracks (CC)
            19 = Module Cracks (EC)
            20 = CC Endplates, includes CCEC front plate
            21 = EC Endplates
            22 = CC Warm Cryostat (inner tube)
            23 = CC Cold Cryostat (inner tube)
            24 = CC Warm Cryostat (end+outer tube)
            25 = CC Cold Cryostat (end+outer tube)
            26 = EC Warm Cryostat
            27 = EC Cold Cryostat
\end{verbatim}

\end{document}
