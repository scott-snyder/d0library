% ****** Start of file apssamp.tex ******
%
%   This file is part of the APS files in the REVTeX 3.0 distribution.
%   Version 3.0 of REVTeX, November 10, 1992.
%
%   Copyright (c) 1992 The American Physical Society.
%
%   See the REVTeX 3.0 README file for restrictions and more information.
%
%
%
\documentstyle[prd,eqsecnum,aps,floats]{revtex}
\input psfig
%\input feynman
\def\D0{D\O}
%
\begin{document}
%
%\small
\begin{flushright}

\D0 NOTE 1373a

K.Wyatt Merritt

19-FEB-1992

REV: 3-MAY-1995

A.M.Jonckheere

\end{flushright}
\vspace{1.0in}
\centerline{\D0GEANT CONDENSED USER's GUIDE}

\begin{abstract}
This is a condensed guide to using \D0GEANT, with pointers to help the
new user avoid some common mistakes.  This guide is applicable to the Feb. 92
and subsequent releases, which will be using CERN version 3.14 and above.
For more detailed information on the operation of the program, see
the other \D0\$\D0GEANT:*.MEM files, particularly \D0GEANT\_USER.MEM.
\end{abstract}

\section{introduction}
\label{intro}

There are three ways a user might choose to get an executable file for
\D0GEANT:

\begin{enumerate}

\item Use the .EXE (and perhaps also the GSAVE.DAT) provided.

\begin{quotation}
\centerline{**** NOTE ****}
 Currently the .EXE and GSAVE are only provided in the
\D0\$BETA:[\D0GEANT] area on \D0SFT.  It has been suggested that the library
release procedure should make .EXEs and GSAVEs on each cluster that subscribes
to \D0GEANT.  This of course requires a considerable amount of space.  If it
ever happens, I will try to update this guide in a timely fashion.  As of this
date (2/19/92), however, this option is only available to users on \D0SFT or on
systems with the \D0\$BETA disk DFS-mounted (e.g., FNAL\D0).

To get the \D0SFT \D0\$BETA disk mounted using DFS (Thanks to Alan
Jonckheere for this information; obviously you get your system manager to do
this!) in SYS\$STARTUP:DFS\$SYSTARTUP.COM, add:

\begin{verbatim}
          $  DFSCP :== $DFS$CONTROL
          $  DFSCP MOUNT/SYS .US.FNAL.D0.D0CMS_PRJ$HROOT D0SF$D0BETA /VOL=D0SF$D0BETA
          $  DEFINE/SYS/EXEC D0$BETA D0SF$D0BETA
\end{verbatim}
\end{quotation}

\item Create your own .EXE (optionally including some user code), using an
ordinary link file.

\item Create your own .EXE, including some user code, using MMS.  There are a
number of procedures released in \D0\$BETA\_UTIL which make using CMS and MMS
for code development easier.  For further details on the commands used in this
cookbook, see \D0\$DOCS:LOCAL\_CMS.MEM.

\end{enumerate}
 
In Sections~\ref{provide}-\ref{mms}, this guide will present a complete list of
commands for each option, using command files available in \D0\$\D0GEANT to get
an exe and run it interactively or as a batch job. These sections assume that
the user wants the standard full detector Monte Carlo, with all subdetectors
included, and that the default sizes for the Zebra stores are adequate for his
application. Section~\ref{pbd} deals with running the program builder to create
a version including different subdetectors and/or different store sizes.

Here are a few pointers for new users.

\begin{enumerate}

\item In general, the defaults for the GEANT flags and the \D0-specific flags
are appropriate for generating physics samples.  The default is to use
Linn-parameterized showers in the calorimeter and to turn on all physics
processes in GEANT.  If you use the file provided for running batch jobs
(\D0\$\D0GEANT:FULL\D0\_\D0GEANT.RUN), you will get an appropriate set of
inputs for a batch job. For example, `LBAT T' is required to suppress excess
printout. The default output will contain CAEP banks and CAD banks, but NOT GCAH
banks. If you change the defaults, be aware of the effect on CPU time and output
file size.

\item The default input file in \D0\$\D0GEANT:\D0GEANT\_SETUP.COM is a sample of
10 T-Tbar events.  This is the sample used for \D0GEANT verification; it is NOT
what you should use to test that your .EXE is working to first order, because it
takes many many hours of CPU time to run.  Try something simple like single
tracks for your first attempt.  The sections below show you how to redefine the
input files.

\end{enumerate}

\section{USING THE EXE PROVIDED}
\label{provide}

\begin{verbatim}
    $ CREATE/DIR [your_geant_dir]
    $ SET DEF [your_geant_dir]
\end{verbatim}

For an interactive session: (NOTE: check the beta area; the non-debug version
and the debug version may not both be available.)

\begin{verbatim}
    $@D0$D0GEANT:D0GEANT_SETUP
      { optionally redefine the logicals for the GSAVE.DAT (FOR001 and FOR002)
        and input/output (FOR031 and FOR032 usually) files}

    $RUN D0$BETA:[D0GEANT]D0GEANT.EXE or $RUN D0$BETA:[D0GEANT]DEB_D0GEANT.EXE
\end{verbatim}

For a batch job:

\begin{verbatim}
    $COPY D0$D0GEANT:FULLD0_D0GEANT.RUN *
    $EVE FULLD0_D0GEANT.RUN
     {Add a command at the beginning of this file "SET DEF [your_geant_dir]".
      Change the default definition of USR$EXE to D0$BETA:[D0GEANT]. Edit
      the symbol definition of D0GEANT to run USR$EXE:D0GEANT.EXE (or
      DEB_D0GEANT.EXE, if that's what's available).}
    $SUBMIT/Q=[batch_queue]/PAR=([# trig],[input file],[output file]) -
            FULLD0_D0GEANT.RUN
\end{verbatim}

    Example:

\begin{verbatim}
     SUBMIT/Q=D0GS06_BATCH/PAR=(100,SINGLE_ELECTRONS.DAT,SINGLE_ELECTRONS.GEN) -
       FULLD0_D0GEANT.RUN
\end{verbatim}
will process 100 events from the file SINGLE\_ELECTRONS.DAT in your default
directory, and put the output events into SINGLE\_ELECTRONS.GEN.

A convenient alternative for specifying input and output file names is to create
a file \D0GEANT\_USER.COM in [your\_geant\_dir]:

\begin{verbatim}
    $EVE D0GEANT_USER.COM
     Add lines similar to:

       $DEFINE FOR031 SCRATCH:[SCRATCH.USER]SINGLE_ELECTRONS.DAT
       $DEFINE FOR032 SCRATCH:[SCRATCH.USER]SINGLE_ELECTRONS.GEN

     then

    $SUBMIT/Q=D0GS06_BATCH/PAR=(100,NULL,NULL) FULLD0_D0GEANT.RUN
       !input and output designations will come from D0GEANT_USER.COM
\end{verbatim}

\section{LINKING AN EXE WITH FULL\D0\_D0GEANT.LNK}
\label{link}

\begin{verbatim}
    $ CREATE/DIR [your_geant_dir]
    $ SET DEF [your_geant_dir]
    $ COPY D0$D0GEANT:FULLD0_D0GEANT.LNK *
    $ EVE FULLD0_D0GEANT.LNK
      [Edit the LINK command to put the EXE in a scratch directory.
       This is not strictly necessary, but be warned that the EXE
       is about 6000 blocks (>10000 blocks for the debug version).]

  $@FULLD0_D0GEANT.LNK or $@FULLD0_D0GEANT.LNK DEBUG
\end{verbatim}

For an interactive session:

\begin{verbatim}
  $@D0$D0GEANT:D0GEANT_SETUP
      { optionally redefine the logicals for the GSAVE.DAT (FOR001 and FOR002)
        and input/output (FOR031 and FOR032 usually) files}
  $RUN [scratch_dir]FULLD0_D0GEANT.EXE or $RUN [scratch_dir]DEB_FULLD0_D0GEANT.EXE
\end{verbatim}

For a batch job:

\begin{verbatim}
    $COPY D0$D0GEANT:FULLD0_D0GEANT.RUN *
    $EVE FULLD0_D0GEANT.RUN
     {Add a command at the beginning of this file "SET DEF [your_geant_dir]".
      Also, if you put the .EXE in [scratch_dir], change the default definition
      of USR$EXE to [scratch_dir] instead of []. It expects to run the non-debug
      .EXE, so if you want to run the debug version, you will have to edit
      the symbol definition of D0GEANT appropriately.}
    $SUBMIT/Q=[batch_queue]/PAR=([# trig],[input file],[output file]) -
            FULLD0_D0GEANT.RUN
\end{verbatim}

    Example:

\begin{verbatim}
     SUBMIT/Q=D0GS06_BATCH/PAR=(100,SINGLE_ELECTRONS.DAT,SINGLE_ELECTRONS.GEN) -
            FULLD0_D0GEANT.RUN
\end{verbatim}
will process 100 events from the file SINGLE\_ELECTRONS.DAT in your
default directory, and put the output events into SINGLE\_ELECTRONS.GEN.

A convenient alternative for specifying input and output file names is
to create a file \D0GEANT\_USER.COM in [your\_geant\_dir]:

\begin{verbatim}
    $EVE D0GEANT_USER.COM
     Add lines similar to:

       $DEFINE FOR031 SCRATCH:[SCRATCH.USER]SINGLE_ELECTRONS.DAT
       $DEFINE FOR032 SCRATCH:[SCRATCH.USER]SINGLE_ELECTRONS.GEN

     then

    $SUBMIT/Q=D0GS06_BATCH/PAR=(100,NULL,NULL) FULLD0_D0GEANT.RUN
       !input and output designations will come from D0GEANT_USER.COM
\end{verbatim}

\section{USING MMS TO PRODUCE AN EXE}
\label{mms}

\begin{verbatim}
    $ CREATE/DIR [your_geant_dir]
    $ SET DEF [your_geant_dir]
    $ SETUP_MMS     ! You will see the following questions:

      Primary BETA-Area [TEST] : D0GEANT
      Combined-Package : FULLD0
      Link Object [USR$AREA:D0GEANT.OPT] :
      Link Object :
      USR$EXE [USR$ROOT1:[WYATT.TEST]] : USR$SCRATCH:[WYATT.TEST]
      USR$OUT [USR$SCRATCH:[WYATT.TEST]] :
      Program Name [/Task][/DI3000] [D0GEANT] :
      Framework Name [D0GEANT] :

                    ! SETUP_MMS will give this printout
      Beta-Area              D0GEANT
      Combined-Package       FULLD0
      Link-Objects           USR$AREA:D0GEANT.OPT
      USR$EXE                USR$SCRATCH:[WYATT.TEST]
      USR$OUT                USR$SCRATCH:[WYATT.TEST]
      Program-Name           D0GEANT
      Framework-Name         D0GEANT

                   ! + more printout about the files created:
                   ! MMS.COM,DEFINE.COM,DESCRIP.MMS
      {create the .FOR files in [your_geant_dir] that you want to link in}
    $ SET_PREFIX == 1              ! to get the DEB_ prefix onto the EXE name
    $ USERLIB
    $ @MMS NODEBUG/NOCMS{/NOBETA}  ! to link a non-debug .EXE including
                                   ! your code. The default is to include
                                   ! code in D0$BETA:[D0GEANT]; you can
                                   ! exclude that with the NOBETA switch.

\end{verbatim}

\begin{quote}
\centerline{****NOTE****}
The EXE will be USR\$EXE:\D0GEANT.EXE, *NOT* FULL\D0\_\D0GEANT.EXE, which is the
name produced with FULL\D0\_\D0GEANT.LNK (see Section~\ref{link}). The symbol
\D0GEANT to run this EXE is correctly defined when you execute DEFINE.COM.
\end{quote}

For an interactive session:

\begin{verbatim}
    $@D0$D0GEANT:D0GEANT_SETUP
      { optionally redefine the logicals for the GSAVE.DAT (FOR001 and FOR002)
        and input/output (FOR031 and FOR032 usually) files}
    $RUN USR$EXE:D0GEANT.EXE or $RUN USR$EXE:DEB_D0GEANT.EXE
\end{verbatim}

For a batch job:

\begin{verbatim}
    $COPY D0$D0GEANT:FULLD0_D0GEANT.RUN *
    $EVE FULLD0_D0GEANT.RUN
     [Add a command at the beginning of this file "SET DEF [your_geant_dir]"]
    $SUBMIT/Q=[batch_queue]/PAR=([# trig],[input file],[output file]) -
            FULLD0_D0GEANT.RUN
\end{verbatim}

    Example:

\begin{verbatim}
     SUBMIT/Q=D0GS06_BATCH/PAR=(100,SINGLE_ELECTRONS.DAT,SINGLE_ELECTRONS.GEN) -
            FULLD0_D0GEANT.RUN
\end{verbatim}
will process 100 events from the file SINGLE\_ELECTRONS.DAT in your
default directory, and put the output events into SINGLE\_ELECTRONS.GEN.

A convenient alternative for specifying input and output file names is
to create a file D0GEANT\_USER.COM in [your\_geant\_dir]:

\begin{verbatim}
    $EVE D0GEANT_USER.COM
     Add lines similar to:

       $DEFINE FOR031 SCRATCH:[SCRATCH.USER]SINGLE_ELECTRONS.DAT
       $DEFINE FOR032 SCRATCH:[SCRATCH.USER]SINGLE_ELECTRONS.GEN

     then

    $SUBMIT/Q=D0GS06_BATCH/PAR=(100,NULL,NULL) FULLD0_D0GEANT.RUN
       !input and output designations will come from D0GEANT_USER.COM
\end{verbatim}

\section{USING THE PROGRAM BUILDER WITH D0GEANT}
\label{pbd}

The first example shows how to make a new program builder combined package
with a different set of Zebra store sizes and/or a different subset of
the standard \D0GEANT packages.

\begin{verbatim}
    $ CREATE/DIR [your_geant_dir.PBD]
    $ SET DEF [your_geant_dir.PBD]
    $ @D0$D0GEANT:D0GEANT_HOOKS.COM

 Name of COMBINED package : MYPKG
 NOTE: The package: D0 will be included automatically.

 List all packages to be combined : GVTX,GCDC,GFDC,GTRD

 ZEBCOM [1750000]
 ZEBSTP [600000]
 ZEBWRK [20000]
 PAWC   [250000]
 GCBANK [1500000]

 Change common block sizes ? [N] : Y
 ZEBCOM [1750000] : 1000000
 ZEBSTP [600000] : 300000
 ZEBWRD [20000] :
 PAWC   [250000] :
 GCBANK [1500000] : 1000000
 Zebra_sizes : /ZEBCOM=1000000/ZEBSTP=300000/ZEBWRK=20000/PAWC=250000/GCBANK=1000000


 Take a walk, I'm working..!

$ PBD/FRAME=D0GEANT/NAME=MYPKG/PACKAGES=(D0,GVTX,GCDC,GFDC,GTRD)/ZEBCOM=1000000/
ZEBSTP=300000/ZEBWRK=20000/PAWC=250000/GCBANK=1000000

 Files created by PROGRAM BUILDER

Directory USR$ROOT1:[WYATT.TEST.PBD]

DEB_MYPKG_D0GEANT.OBJ;1
                     25-FEB-1992 14:35:42.61
DEB_MYPKG_D0GEANT.OPT;1
                     25-FEB-1992 14:35:56.52
MYPKG_D0GEANT.COM;1
                     25-FEB-1992 14:36:00.99
MYPKG_D0GEANT.FOR;1
                     25-FEB-1992 14:35:15.59
MYPKG_D0GEANT.LNK;1
                     25-FEB-1992 14:35:55.35
MYPKG_D0GEANT.OBJ;1
                     25-FEB-1992 14:35:29.31
MYPKG_D0GEANT.OPT;1
                     25-FEB-1992 14:35:56.13

Total of 7 files.

 Check log file PBD.LOG for errors.
\end{verbatim}

Now you should relink using the program builder code just created.  If you
are using the MMS utilities (Section~\ref{mms}), the search lists should
already be correctly defined, and all you need to do is relink with @MMS.  If
you are using FULL\D0\_\D0GEANT.LNK (Section~\ref{link}), you need to redefine 
\D0\$\D0GEANT as a search list including [your\_geant\_dir.PBD], or to copy the
.OBJ files up to [your\_geant\_dir], and edit FULL\D0\_D0GEANT.LNK to make the
default TASK~=~MYPKG instead of FULL\D0.  Then relink with 
@FULL\D0\_\D0GEANT.LNK. 

The next example shows how to create a new program builder package for
\D0GEANT, and link it in addition to the standard packages.

\begin{verbatim}
    $ CREATE/DIR [your_geant_dir.PBD]
    $ SET DEF [your_geant_dir.PBD]
    $ COPY D0$PBD:GCAL.PBD NEW_PKG.PBD
    $ EVE NEW_PKG.PBD
        { edit the list of hooks between .HOOKS and .END HOOKS to be the
          appropriate hooks for NEW_PKG (not all hooks need be present)}
    $ @D0$D0GEANT:D0GEANT_HOOKS.COM 

 Name of COMBINED package : MYPKG
 NOTE: The package: D0 will be included automatically.

 List all packages to be combined : GVTX,GCDC,GFDC,GTRD,GLV0,GCAL,GMUO,GSAM,NEW_PKG

 ZEBCOM [1750000]
 ZEBSTP [600000]
 ZEBWRK [20000]
 PAWC   [250000]
 GCBANK [1500000]

 Change common block sizes ? [N] :
 Zebra_sizes : /ZEBCOM=1750000/ZEBSTP=600000/ZEBWRK=20000/PAWC=250000/GCBANK=1500000


 Take a walk, I'm working..!

$ PBD/FRAME=D0GEANT/NAME=MYPKG/PACKAGES=(D0,GVTX,GCDC,GFDC,GTRD,GLV0,GCAL,GMUO,G
SAM,NEW_PKG)/ZEBCOM=1750000/ZEBSTP=600000/ZEBWRK=20000/PAWC=250000/GCBANK=1500000
%DCL-I-SUPERSEDE, previous value of D0$D0GEANT has been superseded
%DCL-I-SUPERSEDE, previous value of D0$PBD has been superseded

 Files created by PROGRAM BUILDER
                .
                .
                .
\end{verbatim}

Follow the instructions from the first example to relink with the new program
builder code.

\end{document}
