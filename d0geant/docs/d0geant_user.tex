% ****** Start of file apssamp.tex ******
%
%   This file is part of the APS files in the REVTeX 3.0 distribution.
%   Version 3.0 of REVTeX, November 10, 1992.
%
%   Copyright (c) 1992 The American Physical Society.
%
%   See the REVTeX 3.0 README file for restrictions and more information.
%
%
%
\documentstyle[prd,eqsecnum,aps,floats]{revtex}
\input psfig
%\input feynman
\def\D0{D\O}

\begin{document}
%
%\small
\begin{flushright}
\D0 NOTE 969a

A.M.Jonckheere

15-May-1986

Rev: 30-May-1990

Rev: 21-Sep-1990

Rev: 20-Jan-1992

K.W.Merritt

Rev: 4-May-1995
\end{flushright}
\vspace{1.0in}
\centerline{\D0GEANT USER's GUIDE}

\begin{abstract}
This is the user's guide to \D0GEANT, the \D0 implementation of CERN's general
High Energy Physics event Monte Carlo, GEANT3. Any comments, good or bad,
corrections, additions etc. should be addressed to Mike Diesburg via DECMAIL
at \D0SFT::DIESBURG (DIESBURG@\D0SFT.FNAL.GOV via INTERNET).
\end{abstract}

\section{INTRODUCTION}
\label{intro}

\D0GEANT is currently implemented using the CERN version GEANT~3.15. We have
taken the CERN PAM files and regenerated the object libraries and .FOR files in
the \D0\$CERNLIB area. GEANT~3.15 differs from 3.14 only in improved low energy
behavior and required no changes to \D0GEANT. Prior to the 1/92 test release
of \D0GEANT corresponding to the use of GEANT~3.14, the implementation used
GEANT~3.11.

GEANT~3.14 and 3.15 differ from V3.11 in several major ways which affect
\D0GEANT. The new GEANT uses the KUIP command interface instead of ZCEDEX.  The
menus contain an expanded functionality, and the interactive mode now
allows the specification of more than one trigger, eliminating the need
for a separate `batch' mode.  The graphics views will now be provided with
KUIP macros, instead of Fortran routines.  The tracking routines have been
completely recoded.  (Note: The option of parallel tracking, widely deemed
not to work, is turned off by default in 3.14 and 3.15, and will not be
discussed here.) This requires changes in the way the secondary tracks are
handled in \D0GEANT.  Also, by default, the new GEANT will recompute the
tracking parameters for the different tracking media and use those instead of
the parameters specified in the STP files (or hard-coded, depending on the 
subdetector). Work is still underway to determine whether it is best to run in
this mode, or to turn off the automatic feature using the input card
``AUTO~0''. In addition, many improvements (or at least changes) have been made
to the coding of the physics processes.  This last difference is the primary
reason for making the upgrade, but requires no code changes in \D0GEANT.

Minor technical changes include changes in a number of the GEANT commons
and changes to some subroutine calling sequences.  These of course force the
recompilation of a number of \D0GEANT routines which have no substantive
changes.

Another change dating from the 1/92 test release is the use of the new
version of the \D0 Program Builder, which uses a common hook template for all
the \D0 program frameworks, including \D0GEANT.  This has required that
the names of the detector packages and of the standard combined package change,
to avoid conflicts with packages designed for the offline software. The .PBD
files for \D0GEANT packages are released in the area \D0\$PBD, in accordance
with the new program builder Zeitgeist.

\begin{verbatim}
        Old PBD package             New PBD package
        --- --- -------             --- --- -------
        FULL                        FULLD0   (combined package with all subdetectors)
        CAL                         GCAL     (calorimeter)
        CDC                         GCDC     (central drift chamber)
        FDC                         GFDC     (forward drift chamber)
        LVO                         GLVO     (level O trigger counters)
        MUO                         GMUO     (wide angle muon chambers)
        SAM                         GSAM     (SAMUS chambers)
        TRD                         GTRD     (transition radiation detector)
        VTX                         GVTX     (vertex detector)
        TBM                         GTBM     (test beam calorimeter)
\end{verbatim}

See Section~\ref{pbd} for more details on using the Program Builder with \D0GEANT.

\section {SAMPLE BATCH JOB}
\label{Samplejob}

For a sample batch job to run \D0GEANT.EXE in your default
directory see:

\begin{verbatim}
        D0$D0GEANT:FULLD0_D0GEANT.RUN
\end{verbatim}

This command file makes all necessary assignments. It then runs the EXE twice
from your default directory. During the first run the geometry and necessary
runtime tables are created and saved (USR\$EXE:GSAVE.DAT).
The second run reads the saved geometry and runs the program on the
input file specified (default is the standard 10 TTB events, which
will take days of CPU time!).

The command file invokes two other command files:

\begin{verbatim}
        DEFINE.COM             : To locate the .EXE and GSAVE in USR$EXE
        D0GEANT_SETUP.COM      : To make D0GEANT specific assignments
\end{verbatim}

\section{LINKING FULL\_\D0GEANT}
\label{linking}

To link the ``standard'' version of FULL\D0\_\D0GEANT all you need to do is:

\begin{verbatim}
        $ @D0$D0GEANT:FULLD0_D0GEANT.LNK
\end{verbatim}

The command file puts the current release version of FULL\_\D0GEANT.EXE into
your default directory.  If you want a ``debug'' version, add the parameter:
``DEBUG'' to the above command.  In this case the .EXE file will be
DEB\_FULL\D0\_\D0GEANT.EXE. If you have any new routines to add, or have made
changes to any existing  routines, compile them in your default directory and
write a file: \D0NEW.OPT on your default directory. \D0NEW.OPT is a standard
linker option file. The routines listed will be linked {\em before} anything else.
For example:

\begin{verbatim}
        ! D0NEW.OPT this is an example
        !
        File1,file2,file3, -
        File4,file5/lib
        !
\end{verbatim}

\section{CREATING A CUSTOMISED xxx\_\D0GEANT (PBD)}
\label{pbd}

There are a number of reasons why you may wish to customise \D0GEANT to your
specific needs. The most obvious is to ``tune'' the sizes of the three main Zebra
stores used in the program. The default sizes in FULL\D0\_\D0GEANT are large
enough that almost any Tevatron event will run. It also contains all of the code
needed to run the full detector. However, because of this, it is huge. It loads
in just about 35 MegaBytes. It may be impossible for some users to load the
program. This size typically requires changes both to individual process quotas
and to system parameters.

If the full program is not needed, all one needs to do is run Program Builder,
specifying only those parts of the program one needs and/or reduced sizes for
some of the stores. It'll take some experimentation to determine the minimum
sizes for the stores. A command file \D0GEANT\_HOOKS.COM is provided to run
the Program Builder and make a new combined package with different detector
packages and/or store sizes.  The example below shows how to run this file.
For more details on Program\_Builder see: \D0\$PROGRAM\_BUILDER:USER\_GUIDE.MEM.

\begin{verbatim}
      $ @D0$D0GEANT:D0GEANT_HOOKS

 Name of COMBINED package : TESTMC
 NOTE: The package: D0 will be included automatically.

 List all packages to be combined : GVTX,GCDC,GFDC,GTRD,GLVO,GCAL,GMUO,GSAM

 ZEBCOM [1750000]
 ZEBSTP [600000]
 ZEBWRK [20000]
 PAWC   [250000]
 GCBANK [1500000]

 Change common block sizes ? [N] : Y
 ZEBCOM [1750000] : 2000000
 ZEBSTP [600000] :
 ZEBWRK [20000] :
 PAWC   [250000] : 150000
 GCBANK [1500000] :
 Zebra_sizes : /ZEBCOM=2000000/ZEBSTP=600000/ZEBWRK=20000/PAWC=150000/GCBANK=1500000


 Take a walk, I'm working..!

$ PBD/FRAME=D0GEANT/NAME=TESTMC/PACKAGES=(D0,GVTX,GCDC,GFDC,GTRD,GLVO,GCAL,GMUO,
GSAM)/ZEBCOM=2000000/ZEBSTP=600000/ZEBWRK=20000/PAWC=150000/GCBANK=1500000

 Files created by PROGRAM BUILDER

Directory USR$ROOT1:[WYATT.V314.PBD]

DEB_TESTMC_D0GEANT.OBJ;1
                     20-JAN-1992 13:49:01.51
DEB_TESTMC_D0GEANT.OPT;1
                     20-JAN-1992 13:49:18.17
TESTMC_D0GEANT.COM;1
                     20-JAN-1992 13:49:19.27
TESTMC_D0GEANT.FOR;1
                     20-JAN-1992 13:48:24.32
TESTMC_D0GEANT.LNK;1
                     20-JAN-1992 13:49:17.26
TESTMC_D0GEANT.OBJ;1
                     20-JAN-1992 13:48:44.89
TESTMC_D0GEANT.OPT;1
                     20-JAN-1992 13:49:17.70

Total of 7 files.

 Check log file PBD.LOG for errors.
\end{verbatim}

\section{Using Local CMS to link a personal version}
\label{betacms}

The \D0 Local CMS system can be used to maintain and link a personal
version of \D0GEANT in which changes to your code are automatically
compiled and linked.  For a general introduction, see \D0\$DOCS:LOCAL\_CMS.MEM.
This section details the specific use of this system for the beta release of
\D0GEANT.

\begin{verbatim}
    Step 1.   In your working directory,

 $ SETUP_MMS

 Primary BETA-Area [SCRATCH] : D0GEANT
 Combined-Package : FULLD0
 Link Object [USR$AREA:D0GEANT.OPT] :
 Link Object :
 USR$EXE [USR$ROOT1:[WYATT.SCRATCH]] : USR$SCRATCH:[WYATT]
 USR$OUT [USR$SCRATCH:[WYATT]] :
 Program Name [/Task][/DI3000] [D0GEANT] :
 Framework Name [D0GEANT] :

 Beta-Area              D0GEANT
 Combined-Package       FULLD0
 Link-Objects           USR$AREA:D0GEANT.OPT
 USR$EXE                USR$SCRATCH:[WYATT]
 USR$OUT                USR$SCRATCH:[WYATT]
 Program-Name           D0GEANT
 Framework-Name         D0GEANT


CMMS ---- Build standard MMS.COM file for D0 code development
          V3.3 10-Sep-1991


Beta release area  : D0$BETA:[D0GEANT]

Creating file: MMS.COM..

%CMMS-I-SUCCESS, built MMS.COM


CDEFINE ---- Build standard DEFINE.COM file for D0 code development
             V2.9 14-Dec-1991


Creating file: DEFINE.COM..


CDEFINE ---- End



CDESCRIP ---- Build standard description file for D0 code development
              V3.6 16-Sep-1991


Descrip      : DESCRIP.MMS
Program      : D0GEANT
User library : USERLIB
Option       : NODI3000


Creating file: DESCRIP.MMS

%CDESCRIP-I-SUCCESS, built DESCRIP.MMS

$
\end{verbatim}

The three files created are DEFINE.COM, MMS.COM, and DESCRIP.MMS. You should
execute DEFINE.COM to get all the logical search lists defined properly. The
file MMS.COM will recompile as necessary the code you put in your area as in
Step 2, and link \D0GEANT. DESCRIP.MMS is used by MMS when you link.

\begin{verbatim}
    Step 2.   Place in your working directory the code you want to add or
              change. Then

$userlib    ! Makes a list of your code and its dependencies (on INC's, etc.)
$@mms [no]debug/nocms/skip FULLD0 
            ! Compiles your code and links D0GEANT.
            ! The .EXE will be [DEB_]D0GEANT.EXE and will reside in USR$EXE
            ! which is defined in DEFINE.COM.
\end{verbatim}

When you change your code, just re-execute the mms.com.  You do not need
to use the userlib command again unless you add another routine (or delete
one).

\section{DATA FILES}
\label{files}

Before \D0GEANT can be executed a number of data files need to be defined.

\begin{verbatim}
        @D0$D0GEANT:D0GEANT_SETUP.COM
\end{verbatim}

defines the "standard" files. Any of these may be redefined as you wish.

The Standard definitions are: (The first name is the logical name, the second
is the default file)

\begin{verbatim}
-- FOR001.DAT          USR$EXE:GSAVE.DAT
   = Input file for the Geant GET command, must previously have been SAVE'd.

-- FOR002.DAT          USR$EXE:GSAVE.DAT
   = Output file for the Geant SAVE command.  This SAVE/GET sequence can save
     a very large amount of time for repetitive tasks. Note that the GSAVE.DAT
     is a very large file (over 5000 blocks if the FULLD0 package is used),
     so you should redefine this logical to point to a scratch area
     if your own quota might not accommodate it.

-- xxx_STPFILE         D0$STP:xxx_STPFILE.DAT                  (READONLY)
   = Zebra format Static Parameters Banks for detector xxx

-- yyyyyyy             D0$D0GEANT$DATA:yyyyyyy.DAT             (READONLY)
   = where yyyyyyy are a number of TRD data files, including:
        XRAYDIST   = TRD x-ray distributions
        DELDIST    = TRD delta ray distributions
        SHAPE_TRD  = TRD geometry constants
        XRSPECT    = TRD x-ray spectrum, written and/or read by TRD routines
                        See D0$D0GEANT:D0GEANT_TRD.MEM
        TRD_STRIPS = TRD Z strip geometry constants

-- FOR031              D0$D0GEANT$DATA:TTB_140.DAT
   = Event input file (output of ISAZEB, Zebra version of ISAJET)

-- FOR032              TTB_140.GEN
   = Event output file (ISAJET + Geant output)
       invoke with:  ZBIO 31 32 in the initial dialogue.
       See D0$D0GEANT:D0GEANT_ZEBIO.MEM

-- RUNSAVE.DAT     = Output file that Geant uses to save its place in a run.
       Using this file it is possible to restart/continue a run that
       bombed partway through. It contains the current event number plus the
       random number seed at the start of the current event. If you *delete*
       this file while a run is in progress, the program will exit gracefully,
       just as if it had finished.
\end{verbatim}

\section{RUNNING FULL\D0\_\D0GEANT or DEB\_FULL\D0\_\D0GEANT}
\label{running}

After you have linked FULL\D0\_\D0GEANT.EXE and have defined the appropriate
data files (defaults defined with @\D0\$\D0GEANT:\D0GEANT\_SETUP.COM): 

\begin{verbatim}
        $ RUN FULLD0_D0GEANT or $ RUN/NODEBUG DEB_FULLD0_D0GEANT

        ***NOTE*** If you use SETUP_MMS to link, the exe will be
        called D0GEANT.EXE or DEB_D0GEANT.EXE.
\end{verbatim}

You will be answered by a fair amount of printout related to ZEBRA and
the versions of Zebra, Geant and \D0GEANT that you are using. Then you will see:

\begin{verbatim}
----------------------------------------------------------------------
 **** GEANT FREE FORMAT SWITCH PROCESSOR ****
LBAT  F                (BATCH IF TRUE)
SHWG  1                (GHEISHA=0, SHOWER PARAM=1, FROZEN SHOWER=2,SHOWERLIB=3)
SSEC .200              (SAVE SECONDARYS>=0, DON'T SAVE<0)
     SD0   0.00  0.00  0.00  0.00  0.00  0.00  0.00  0.00  0.00
DTRK  0 DHIT  1 DDIG  1 (0=OFF,1=PRINT,2=DRAW)
RKIN  0                (.GT.0: EVENT FROM ASCII FILE)
ZBIO  0  0             (ZEBCOM INPUT/OUTPUT UNIT  'S (0=OFF))
PD0   1                (PRINT LEVEL)
DLAM  0                (1->BLUE BOOK ABSORPTION LENGTHS)
 **** Package Specific Switches ****
DCEN  3 DCAL  3 DMUO  3 DLVO  3 (0=OFF,1=DEAD,2=HITS,3=DIG,4=ANALY)
PCEN  1 PCAL  1 PMUO  1 PLVO  1
FFIVTX ** DVTX   3 PVTX   1
          SVTX    1.0   0.0   0.0   1.0   0.0  -1.0  -1.0  -1.0  -1.0  -1.0
FFICDC ** DCDC   3 PCDC   1
          SCDC    1.0   0.0   0.0   1.0   0.0  -1.0  -1.0  -1.0  -1.0  -1.0
FFITRD ** DTRD   3 PTRD   1
          STRD    0.0   0.0   0.0   0.0   0.0  10.0   0.0   0.0   0.0   0.0
FFIFDC ** DFDC   3 PFDC   1
          SFDC    1.0   0.0   0.0   1.0   0.0  -1.0  -1.0  -1.0  -1.0  -1.0
FFILVO ** SLVO    0.0   0.0   0.0   0.0   0.0   0.0   0.0   0.0   0.0   0.0
FFICAL ** DECA   3 DUCA   3 DCRY   3 DEAD   3
          LOECUT 0.010000 HIECUT 0.200000 X0STEP    1.000
          SCAL(1:10):  1.0 0.0 0.0 0.0  0.0010 0.0 0.0 0.0 0.0 0.0
FFIMUO ** SMUO    0.0   0.0   0.0   0.0   0.0   0.0   0.0   0.0   0.0   0.0
 ===> Enter your Detector switches
 ===> Last switch must be STOP
----------------------------------------------------------------------
\end{verbatim}

This is a list of switches that control the \D0 version of Geant.  The
default values are listed after the relevant parameter.  GEANT will next invoke
the FFREAD package which allows the user to change the above switches
particular to \D0, plus a number of others used by generic GEANT, before the
initialization of the detector geometry and the physics process cross sections.
A list of FFREAD cards terminated by STOP is mandatory both for an interactive
and a batch job; it can consist only of the card STOP if defaults are desired
for everything.  All or some of the FFREAD commands can be put into a file
of the form of a Fortran unit number, e.g.\ FOR009.DAT, and then read with the
FFREAD command READ 9.  The commands given to FFREAD determine how the detector
geometry is generated; there are three options:

\begin{verbatim}
    No SAVE or GET card     The geometry banks and cross sections are generated
                            from the information in the STP files, and no
                            GSAVE.DAT is created

    SAVE 'INIT'             The geometry banks and cross sections are generated
                            from the information in the STP files, then the
                            banks are written to the file specified by FOR002
                            (default: GSAVE.DAT)

    GET  'INIT'             The geometry banks and cross sections are retrieved
                            from the file specified by FOR001 (default:
                            GSAVE.DAT)

    *** WARNING ***
    A potential conflict exists between the use of GET 'INIT' and the
    specification of some of the D0 or GEANT parameters.  For example, if
    you create a GSAVE.DAT with a given set of physics processes turned on,
    then try to use GET 'INIT' and turn off one of those processes, it will
    not work.  The cross section and geometry banks will still have that
    process turned on.  Similarly, if you create a GSAVE containing only the
    D0 calorimeter, you cannot later try to turn on the central detector
    switches and use that GSAVE, because the volumes for the central detector
    are not present.
\end{verbatim}

The next Section details the actions and allowed values of each of the
switches.

After the FFREAD cards are processed, control is given to the KUIP command
processor, which may be familiar to the user as the same processor used for
PAW.  The best current resource for the large array of GEANT commands available
interactively is the interactive HELP facility, because the GEANT documentation
is so out-of-date that it
still claims to use ZCEDEX.  At present, there are no \D0-specific KUIP
commands.
For a batch job, the KUIP commands can be
reduced to the following set:

\begin{verbatim}
    GEANT>  SET/BREAK OFF   ! Disables KUIP'S interception of breaks so that
                            ! error tracebacks are available in case of crashes
    GEANT>  TRIG  n         ! n = number of events to be generated
    GEANT>  EXIT
\end{verbatim}

\section{SWITCHES and COMMANDS}
\label{switch}

The FFREAD switches used by the \D0 version of Geant were listed in the sample
printout in the preceding Section. This section lists the current switches,
their functions and allowed values. For ``production'' running, the default
values are used, except for:

\begin{verbatim}
        LIST
        LBAT TRUE
        GET 'INIT'
        ZBIO 31 32
        STOP
\end{verbatim}

\subsection{FIRST LEVEL (SYSTEM) SWITCHES}

\begin{verbatim}
LBAT -> chooses BATCH or INTERACTIVE running  (LBAT TRUE or just LBAT => BATCH)
                (default = .FALSE.)  In the 3.15 version, the only effect of
                this switch is to reduce to printout per event so that batch
                log files are manageable.  With LBAT .FALSE., there will be
                a print out call for each primary track, which in complex
                events is far too much.

SHWG -> chooses shower program to use  (SHWG n) (default = 1)
                0 = Gheisha - track all particles down to the GEANT cutoffs
                1 = Shower Parameterization (Linn parametrization) -
                        Parameterize EM tracks below
                        HIECUT (see LOECUT/HIECUT/X0STEP below)
                        for particles with Pt < SCAL(7) (default SCAL(7)=999.)
                2 = OBSOLETE - Do not use
                3 = Shower library - use library of pretracked showers

           SHWG 0 sets LOECUT=HIECUT=X0STEP=0
           SHWG 1 sets LOECUT etc. to nonzero values.

                LOECUT is the lowest energy tracked.
                HIECUT is the energy at which parameterization starts
                X0STEP is the number of radiation lengths between steps

SSEC -> Save Secondaries produced in MCEN on temporary STAK (SSEC < 0) or in
        permanent JTRAK/JVERT structures (SSEC >= 0). In addition if the track
        momentum is GREATER than SSEC, the tracks will be saved as ISAJET
        tracks in ISP2/ISP3 and ISV2 banks on the output in the /ZEBCOM/
        output. (Default = 0.200)


DTRK,DHIT,DDIG -> control printing and drawing of Tracks, Hits and
       Digitization respectively. (default = 0 1 1)
        0 -> NO PRINTING/DRAWING
        1 -> print data
        2 -> draw data

RKIN -> specifies source of input data. (OVERRIDDEN BY ZEBIO see below)
       if greater than zero (ie RKIN n) will read input data from an ASCII
       file, see below. If equal to 0 will obtain data from the KINE card, one
       track at a time using the KUIP commands. The KINE command:

         KINE ipart pl pt phi x0 y0 z0

     where ipart = the GEANT particle ID
           pl    = longitudinal momentum (GeV)
           pt    = perpendicular momentum (GeV)
           phi   = particle phi angle (in degrees from the x axis)
           x0,y0,z0 = vertex position in the master frame of reference (cm)

     [ IPART  for some common particles:  Electron     3
                                          Positron     2
                                          Muon (-)     6
                                          Pi Minus     9 ]

       RKIN n    n not equal to 0 reads from file Fortran unit n (FOR00n.DAT)
           This file is a formatted ASCII file read by:

        READ n,*,nptcl,x0,y0,z0
        READ n,*,(ident(I),px(I),py(I),pz(I),I=1,nptcl)

     where NPTCL = number of "primary" particles in the event
           X0,Y0,Z0 = vertex posistion in the Geant_detector master system (cm)
           IDENT = the ISAJET particle ID ************** NOTE THIS *******
           PX,PY,PZ = the particle 3 momentum (GeV)
                (default = 0)

ZBIO inunit outunit nskip -> turns on Zebra I/O in the D0 format.  If omitted,
    the RKIN switch is used to determine the input device. If outunit is
    omitted no data will be output.  nskip specifies the number of events on
    the input file to be skipped, defaults to 0. Data comes from/goes to file
    FOR0xx.DAT where xx=inunit or outunit.  See discussion above. Additional
    parameters can be added for debugging. See D0GEANT:D0GEANT_ZBIO.MEM.
                (default = 0 0 0)

PD0 n   -> provides levels of printout for the top level routines.
                0 -> None, 1 -> Minimal,..... as desired
                (default = 1)


SD0     -> general switch array (REAL SD0(10)) that can be used to perform
           special functions within the framework of the program. Only two of
           these are currently in use:
                SD0(1) <> 0 - store drawing hits rather than drawing them. These
                              may be saved (SAVE 'JXYZ') and then retrieved
                              (GET 'JXYZ') to replay an event
                SD0(2) <> 0 - DROP ALL ZEBCOM output banks except the RAW data
                              banks. This gives mimimum output.

DLAM -> specifies special interaction length computation.
        0 -> use energy dependent interaction lengths - normal
        1 -> calculate "Blue Book" interaction lengths - no showering
                (default = 0)

DCEN/DCAL/DMUO/DLVO -> controls detector processing level:
     0 => Off,  do not call geometry routines, the detector is not there
     1 => On,   Detector is present as material only, not an active element
     2 => Hits, Accumulate hits only for detector
     3 => Digi, Digitize detector
     4 => Anal, Analyse detector
        DCEN = Central tracker
        DCAL = Calorimeter
        DMUO = Muon system
        DLVO = Level 0 system
                (default = 3)

PCEN/PCAL/PMUO/PLVO -> provides levels of printout for the
        appropiate detector. 0 -> None, 1 -> Minimal,..... as desired
                (default = 1)
\end{verbatim}

\subsection{SECOND LEVEL (PREDEFINED USER) SWITCHES}

In addition to the ``global'' switches above, user (detector specific)
switches have been defined for each detector.  These include:

\begin{verbatim}
DVTX/DCDC/DFDC/DTRD - same as DCEN etc. The upper limit on their value is 
                      given by DCEN.
                        DVTX -> Vertex Chamber
                        DCDC -> Central Drift Chamber
                        DFDC -> Forward Drift Chamber
                        DTRD -> Transition Radiation Detector

PVTX/PCDC/PFDC/PTRD - same as PCEN etc.

SVTX/SCDC/SFDC/STRD - user arrays of length 10, REAL values.

DECA/DUCA/DCRY/DEAD - same as DCAL etc. The upper limit on their value is
                      given by DCAL.
                        DUCA -> Central Calorimeter
                        DECA -> End Cap Calorimeter
                        DCRY -> Cryostat (DCRY 0 or 1 only)
                        DEAD -> Dead Material, includes cracks.

SCAL                - user array of length 10, REAL values.

SMUO                - user array of length 10, REAL values.

SLVO                - user array of length 10, REAL values.
\end{verbatim}

At the moment I am aware of only a few of the user words (Sxxx) that are being
used. These are used to determine the type of digitization to be done and the
specific banks to be output.

\begin{verbatim}
SCAL(1) = 0            No Caltowers,
    (1) = 1 (default)  Use software Caltowers
    (2) = 0 (default)  Drop individual Unsmeared Track energies in GCAH banks
    (2) = 1            Save Unsmeared Track energies in GCAH
    (3) = 0 (default)  Drop CAEP bank calculated in Caltowers
    (3) = 1            Calculate and save summed, smeared energies in CAEP
    (4) = 0 (default)  No JETS bank calculated
    (4) = 1            Calculate Idealised JETS banks
    (5) < 0            Do not calculate Raw data banks (CAD1/2)
    (5) = 0            Calculate FULL Raw data banks (CAD1/2)
    (5) > 0            Calculate Raw data banks (CAD1/2) with zero suppression
                       on cells with   ABS(ENERGY) < SCAL(5) in GeV.
            (default = 1.0E-3 ie 1 ADC count in all cells)
    (6) = 0 (default)  Do not save ICD Hits
        = 1            Save ICD Hits banks
    (7) = 999. (default)Use the default SHWG value all EM primary particles
                       with PT below this value.
        = 10.0 (e.g)   Force SHWG = 0 for all EM particles with Pt > 10 Gev

STRD(1) -> Chooses the type of TRD digitization to be done.
        = 0.  Reads distributions from XRSPECT
        = 1.  Calculates distributions (and writes them onto XRSPECT)
        = 2.  Does total energies only. Reads data from:
               XRAYDIST and DELDIST
                (default = 0)
STRD(2), STRD(3), STRD(4) control the banks output in the store /ZEBCOM/

STRD(2) = 0. (default) cathode + anode are output : The kind of output is
            defined by STRD(3)
        = 1. No treatment of the cathodes
STRD(3) = 0. ( default) Digitization ( CDD4 ) + Geant specific banks GTLY
        = 1. CDD4 only
STRD(4) = 0. (default) No hits banks written
        = 1. Hits banks TRDH written in the output

SCDC(1:3) -> controls which banks are output in /ZEBCOM/ for CDC.
SCDC(1) = 0. No raw FADC bank CDD2 is output
        = 1. (default) CDD2 is output
SCDC(2) = 0. (default) No hits banks CDCH-DLYR-DSEC are output
        = 1. Hits banks output
SCDC(3) = 0. (default) No data banks DCDA are output
        = 1. Data banks output

SCDC(4:5) -> modify the FADC raw data output in the CDD2 bank.
SCDC(4) = 0. do not perform bilinear conversion for each FADC bin
        = 1. (default) Bilinear conversion is performed for each FADC bin in
             the raw data bank CDD2.
SCDC(5) = 0. (default) Perform zero_suppression on FADC raw data
        = 1. no zero_suppression on FADC raw data


SFDC(1:3) -> controls which banks are output in /ZEBCOM/ for FDC.
SFDC(1) = 0. No raw FADC bank CDD3 is output
        = 1. ( default) CDD3 is output
SFDC(2) = 0. (default) No hits banks are output
        = 1. Hits banks output
SFDC(3) = 0. (default) No data banks are output
        = 1. Data banks output

SFDC(4:5) -> modify the FADC raw data output in the CDD3 bank.
SFDC(4) = 0. do not perform bilinear conversion for each FADC bin
        = 1. (default) Bilinear conversion is performed for each FADC bin in
             the raw data bank CDD3.
SFDC(5) = 0. (default) Perform zero_suppression on FADC raw data
        = 1. no zero_suppression on FADC raw data


SVTX(1:3) -> controls which banks are output in /ZEBCOM/ for VTX.
SVTX(1) = 0. No raw FADC bank CDD1 is output
        = 1. ( default) CDD1 is output
SVTX(2) = 0. (default) No hits banks VTXH-VLAY-VSEC are output
        = 1. Hits banks output
SVTX(3) = 0. (default) No data banks VWDA are output
        = 1. Data banks output

SVTX(4:5) -> modify the FADC raw data output in the CDD1 bank.
SVTX(4) = 0. do not perform bilinear conversion for each FADC bin
        = 1. (default) Bilinear conversion is performed for each FADC bin in
             the raw data bank CDD1.
SVTX(5) = 0. (default) Perform zero_suppression on FADC raw data
        = 1. no zero_suppression on FADC raw data
\end{verbatim}

\subsection{THIRD LEVEL (USER DEFINED) SWITCHES}

The switches above are defined for the user and are stored in COMMON~/\D0LOG/
(\D0LOG.INC). In addition the user may define his own switches and store them
in /DxxLOG/ (DxxLOG.INC). The switches of this type that currently exist are:

\begin{verbatim}
LOECUT/HIECUT/X0STEP - controls shower parameterization  (in DCALOG.INC)
    LOECUT - minimum energy to parameterize, dumps remainder at end point
                (default 0.010 GEV)
    HIECUT - maximum energy to parameterize, Geant tracking above this energy
                (default 0.200 GEV)
    X0STEP - step size in radiation lengths for parameterization
                (default 1.00 X0)
\end{verbatim}


\subsection{GEANT COMMANDS}

In addition to the switches there are a number of commands that have been
provided by GEANT's authors. These are standard Geant switches and are
described in some detail in the Geant manual.

\begin{verbatim}
LIST   turn on echoing and checking of data cards.

SAVE 'INIT' ['KINE' ['HITS' ['DIGI']]]   save Geant Init, Kinematic, Hits
       and Digi banks, the [] indicate optional additions.  Output goes to
       file FOR002.DAT. NOTE: One might save DIGI and not HITS but it makes
                        no sense to save either without both INIT and KINE.

       INIT   saves the initialization banks, Geometry, Materials, Media etc.
              once per run, initialization takes a great deal of time if you
              are running with a complicated geometry. So doing this once and
              then GETting the results for additional runs can save a lot of
              time.

       KINE   saves the initial event banks, JVERT and JTRACK, saves reading in
              complicated ISAJET events for example.

       HITS   saves the Hit banks, useful if you want to redo digitization for
              different smearing etc.

       DIGI   saves the digitization banks, useful if you want to redo the
              analysis portion of the run.

GET 'INIT' ['HITS' ['DIGI']] inverse of SAVE.  Files are read from FOR001.DAT


HSTA 'TIME' ...  enable Geant built in (system type) histograms.

 ...   See Geant manual for more.

       Input is terminated by:

STOP
\end{verbatim}

\subsection{INTERACTIVE COMMANDS}

In interactive mode there are a large number of commands available to you. Some
are provided by the KUIP system itself and some by GEANT.
Below you will find a short list of useful interactive
commands. This list is by no means complete.  See the GXINT section of the
Geant manual for more.

\begin{verbatim}
*** GEANT Commands ***

QUIT         Exit GEANT without the normal exit processing
EXIT         Exit GEANT with the normal exit processing

ZOOM x y     allows one to zoom in on any portion of the detector.  One must
             first have drawn a view (DVIEW2 for example).  If one has a cursor
             equiped terminal, the cursor will now appear.  On a VT240 as with
             most of our other terminals, one moves the cursor with the "arrow"
             keys or the mouse.  You "mark" the desired center of the new
             drawing by hitting the SPACE BAR then RETURN, or with the left
             mouse button.  Follow this by reissueing your drawing command (ie.
             DVIEW2).  x and y are the zoom factors for the x and y axes
             respectively. If y is omitted ZOOM x x is assumed. If both are
             omitted a factor of 2 is used. ZOOM 1 1 allows you to pan the
             view.  ZOOM 0 resets the view to its original scale.

MEASURE      allows one to measure linear distances on a drawing.  Mark both
             ends of the line you want to measure. The distance between the
             points is printed after the second point is marked.

KINE ipart pl pt phi x0 y0 z0    active ONLY if RKIN 0 and no ZBIO was given
             in the first dialog.  Allows one to debug the program and to
             obtain single tracks (drawing or printing as desired).  See
             discussion of RKIN above.

DTREE vol    plots the geometry tree under vol, where vol is any given volume
             name.

DFSPC vol    draws the specs for all volumes under vol, produces 3 views of
             each volume and the immediate daughters with parameters and
             scales.  After it is done with one it will ask if you want the
             next one.  Try it you'll like it.

DSPEC vol    Same as above, but does only one volume.

DCUT vol iaxis cutval u0 v0 su sv    draws a cut view of vol and all daughters
             in plane perpendicular to axis iaxis (1=x etc.) cutting at cutval
             cm from the origin.  u0, v0 are the desired screen coordinates of
             the origin (10. 10. will put the interaction point at the center
             of the screen).  su, sv are the scale factors to be used
             (.015 .015 will display most of the detector).

MANY-MANY-MORE see the GXINT section of the Manual.
\end{verbatim}

\section{PROGRAM STRUCTURE}
\label{struct}

Geant is structured as both a framework and a library of utility routines. The
framework is not a ``pure'' framework however. It calls user routines at
strategic points. But the user routines must call other parts of the framework.
For example, while tracking an event, the user routine GUTRAK is called for
each track. GUTRAK {\em must} call GTRACK, a Geant routine to actually track each
track.

The \D0 version of Geant has cleaned up this situation somewhat by providing
routines to do all of the common functions, like calling GTRACK above, creating
the overall Mother Volumes, saving and writing out track banks created within
\D0GEANT, reading and writing the input and output banks etc. In addition, to
minimize conflicts between groups developing code independently, each detector
is treated as a separate, independent Program Builder package. The general
functions are performed by a separate package, called \D0.

From a users point of view, \D0GEANT, with the ``master'' package \D0
is a framework that calls user hooks and a library of
routines that allow access to the track and geometry parameters. It also
contains routines that allow one to calculate and store detector dependent
quantities for later use or output.

There are one or more user hooks into the program at each point that Geant
allows them. Some of these are for use only by the master package. The Geant
user routines, the Program Builder hooks and the user interface routines are
listed below.

\begin{verbatim}
\GEANT \PBD   \User  \ Description \
----------------------------------------------------

Initialization:
\UGINIT\LURSWT\FFIxxx\ Init run-time switches (FFREAD)\
\UGINIT\LUESWT\CLNxxx\ Edit run-time switches\
\UGINIT\LUBOOK\BOKxxx\ Book HBOOK histograms for ANLxxx\

Create Geometry:
\UGINIT\LURGEO\INIxxx\ Read Geometry Data\
\UGINIT\LUSGEO\GEOxxx\ Create Geometry\

Tracking:
\GUKINE\LUKINE\KIND0 \ Definition of event kinematics, D0 package only\
\GUTREV\LUTREV\TEVxxx\ Called BEFORE each event\
\GUSTEP\LUSTEP\STPxxx\ Called at each step, Store HIT data\
\GUTRAK\LUTRAK\TRKxxx\ Called AFTER each track\

Digitization and output:
\GUDIGI\LUDIGI\DIGxxx\ Digitization after each event\
\GUOUT \LUOUT \ANLxxx\ Analysis and output, after each event\

Finish up after run:
\UGLAST\LULAST\LSTxxx\ Termination, HSTOR etc\

Interactive only:  (THESE ARE NOT CURRENTLY USED IN THE 3.15 VERSION)
\GUINTI\LUINTI\INTxxx\ Initialize interactive user commands\
\GUIGET\LUHELP\HLPxxx\ Print interactive help information\
\GUIGET\LUMENU\MENxxx\ Switching routine for interactive user menu\
\end{verbatim}

\section{NAMING/NUMBERING CONVENTIONS }
\label{naming}

A number of conventions for naming and numbering various entities have been
defined to help prevent conflicts between detector packages. Reserved names for
volumes, detector sets and digitization quantities should start with the
letters:

\begin{verbatim}
     U,E,P,Q - Calorimeters
     A,B,C,M - Muon System
     V,T,F,D - Central Detectors

Reserved numbers for materials and media are:

     1-19  MVOL (Master volume) and pure materials
    20-29  MUON (Muon System)
    30-39  UCAL (Central Calorimeter)
    40-54  ECAL (End Cap Calorimeter)
    55-59  LEVO (Level 0 system)
    60-69  VTX  (Vertex Detector)
    70-79  TRD  (TRD system)
    80-89  CDC  (Central Drift Chamber)
    90-99  FDC  (Forward Drift Chamber)

Rotation matrices  are 100 times these.
\end{verbatim}

\section{MAINTAINING THE CODE}
\label{maint}

The DZERO version of GEANT is maintained on FNAL\D0 using DEC's CMS (Code
Management System) by the authors of the code. See \D0\$DOCS:CMS\_USER.DOC for
information on the CMS procedures. The libraries are formed from the CMS
libraries using MMS (Module Management System) by the \D0 librarian (logical
\D0\$LIBRARIAN on all \D0 clusters).

A new release (version) of the program can be declared at any time by the
\D0GEANT Librarian (Mike Diesburg). In general when a user wishes a release
to be made, s/he should request it from the detector coordinator. The librarian
then declares a new CMS class, does some simple checks to make sure it runs and
requests the release from FNAL\D0. A release may (usually does) involve more
than releasing \D0GEANT. All of the libraries that Geant uses must also be
updated. So the process can take a while.

An FNAL\D0 release results in several libraries plus source code for one being
put into the \D0Library under the rooted directory \D0\$\D0GEANT\$ROOT:. The
OLBs, MEMs, etc.\ are in \D0\$\D0GEANT. The source code can be found in
\D0\$\D0GEANT\$xxx. For more details on the organization of the \D0GEANT
portions of \D0LIBRARY see:  

\begin{verbatim}
        D0$D0GEANT:D0GEANT_LIB.MEM.
\end{verbatim}

For more details on the method of updating and the directory protocols, see:

\begin{verbatim}
        D0$DOCS:PROTOCOL.MEM.
\end{verbatim}

For details on the current, and earlier releases see:

\begin{verbatim}
        D0$D0GEANT:D0GEANT_REL.MEM       old releases
     or D0$D0GEANT:Vn.mm-REL_NOTES       newer ones
\end{verbatim}

For details on specific detectors/subsystems see the various .MEM's in
\D0\$\D0GEANT:.

\end{document}
