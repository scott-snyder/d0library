
%%  $Log$
%% Revision 1.3  1995/05/23  17:15:45  hobbs
%% Version released with file format version 1.6
%%
%% Revision 1.2  1995/01/04  18:40:15  hobbs
%% Release incorporating bad-file RCP parameters and TEXT_STREAM modes
%%
%% Revision 1.1  1994/08/04  16:11:06  hobbs
%% Initial revision
%%

\documentstyle[12pt,epsf]{article}
\textwidth 6.75 truein
\textheight 9.0 truein
\oddsidemargin -0.25 truein
\evensidemargin -0.25 truein
\topmargin -0.5 truein

\def\h4{\hskip 0.5truecm}
\def\vh3{\vphantom{} \hskip 0.325truecm}

\begin{document}

 \title{ D\O\ Dad User's Guide}
 \author{ John D. Hobbs }
 \date{\today  \\ \vskip -0.2truecm {\small $Revision$}\\ }

\maketitle
\setlength{\parindent}{0.75truecm}

This note describes the user's view of the D0DAD system providing 
\underline{d}irect \underline{a}ccess I/O to D\O\ \underline{d}ata.  
The User's Guide begins with instructions for reading data using the centrally
maintained D0DAD files. Section~\ref{ss-ug-rcp} describes optional RCP 
parameters used to select or reject runs based on run or date.  
Section~\ref{ss-ug-lumi} explains how to compute the luminosity corresponding
to the data in a given D0DAD file.  Sections \ref{ss-ug-personal} 
to \ref{ss-ug-utils} describe utilities for creating D0DAD files, 
creating DST's from D0DAD files and examining the contents of D0DAD files.
%%  Finally, appendix~\ref{app-proposal} is a copy of the
%% original note distributed when the preliminary implementation was presented
%% to the OCPB.

\vskip 1.0 truecm\par{\small
\begin{list}{ }{\leftmargin 0.25 in \rightmargin 0.25 in }
\item Original version: June  9, 1994 
\item {\bf Revision 1.1} (September 5, 1994): New format for 
  NEW\_RUNS\_xxxxxx.LIS files 
\item {\bf Revision 1.2} (December 22, 1994): Additional RCP parameters for 
  reporting/controlling errors from  unreadable Zebra files.
  OMNI\_FILTER and RECO version selection  (See section~\ref{ss-ug-rcp}).
\item {\bf Revision 1.3} (May 22, 1995): Modified d0dad file format for 
  increased streaming efficiency.  New mode, TEXT\_STREAM.  Centrally maintained
  trigger and filter d0dad files. (See section~\ref{ss-ug-tstream}.)
\end{list}
}

\newpage

\section{Using Standard D0DAD Streams \label{ss-ug-standard}}
\par
The primary use of D0DAD is to replicate standard D\O\ data streams streams
in a manner which does not require splitting data for different streams
into separate Zebra files.  The data in a given stream is read by 
specifying a D0DAD file as the input file to an analysis job in the 
place of a (set of) Zebra file(s).  The standard set of D0DAD files is created 
for the streams defined in omni\_filter and stored in the directory 
D0\$DATA\$D0DAD.  Table~\ref{t-stand} lists the standard stream names and the 
corresponding D0DAD filenames for the run 1b data.  The run 1a data will have
the D0DAD file names as for run 1b, with the string RUN1B replaced by the
string RUN1A.   An example of reading 
D0DAD-specified data within the D0USER frame is given in figure~\ref{f-easy}.

\par
D0DAD files are also created for all triggers and filters defined in normal
physics running.  These files are in the directory D0\$DATA\$D0DAD\$TRIGFILT.
The file names are RUN1{\it x}\_{\it cccc}\_{\it trigger/filter}.D0DADF.  
Here {\it x} is either {\it A} or {\it B} 
for runs Ia and Ib respectively, {\it cccc} is either {\it TRIG} or 
{\it FILT} for level 1 and level 2 respectivley  and {\it trigger/filter}
is the name defined in the trigger list.

\par
In order to be able to read only runs added after a certain
date, the file {\tt NEW\_RUNS\_YYMMDD.LIS} will be created
in the directory D0\$DATA\$D0DAD each day that
the master catalog is updated.  This file will list all runs entered into the
catalog on the date {\tt MM/DD/YY} in which {\tt YY}, {\tt MM} and {\tt DD}
are the year, month and day respectively.  These files can be used with the
RCP parameters described in section~\ref{ss-ug-rcp} to select specific runs 
either by run number or via a time stamp.
{\tiny
\begin{table}
  \begin{center}
  \begin{tabular}{|l|l|}\hline
     {\small \bf Stream} & {\small \bf D0DAD file name} \\
    \hline
      QCD\_GAMMA 	& RUN1B\_QCD\_GAMMA.D0DADF \\
      MU1\_B      	& RUN1B\_MU1\_B.D0DADF \\
      MU\_B      	& RUN1B\_MU\_B.D0DADF \\
      MU2\_B     	& RUN1B\_MU2\_B.D0DADF \\
      NP\_LQ\_2EM 	& RUN1B\_NP\_LQ\_2EM.D0DADF \\
      NP\_LQ\_ENU 	& RUN1B\_NP\_LQ\_ENU.D0DADF \\
      NP\_LSS\_SELECT 	& RUN1B\_NP\_LSS\_SELECT.D0DADF \\
      NP\_SCALAR     	& RUN1B\_NP\_SCALAR.D0DADF \\
      NP\_SQGL 		& RUN1B\_NP\_SQGL.D0DADF \\
      NP\_MSP 		& RUN1B\_NP\_MSP.D0DADF \\
      NP\_TAU	 	& RUN1B\_NP\_TAU.D0DADF \\
      NP\_MUNU 		& RUN1B\_NP\_MUNU.D0DADF \\
      NP\_MET 		& RUN1B\_NP\_MET.D0DADF \\
      NP\_WRIGHT 	& RUN1B\_NP\_WRIGHT.D0DADF \\
      TOP\_EE 		& RUN1B\_TOP\_EE.D0DADF \\
      TOP\_EJET 	& RUN1B\_TOP\_EJET.D0DADF \\
      TOP\_EMU 		& RUN1B\_TOP\_EMU.D0DADF \\
      TOP\_JETS 	& RUN1B\_TOP\_JETS.D0DADF \\
      TOP\_MUJET 	& RUN1B\_TOP\_MUJET.D0DADF \\
      TOP\_MUMU 	& RUN1B\_TOP\_MUMU.D0DADF \\
      TOP\_TOP\_EE 	& RUN1B\_TOP\_TOP\_EE.D0DADF \\
      TOP\_TOP\_EJET 	& RUN1B\_TOP\_TOP\_EJET.D0DADF \\
      TOP\_TOP\_EMU 	& RUN1B\_TOP\_TOP\_EMU.D0DADF \\
      TOP\_TOP\_JETS 	& RUN1B\_TOP\_TOP\_JETS.D0DADF \\
      TOP\_TOP\_MUJET 	& RUN1B\_TOP\_TOP\_MUJET.D0DADF \\
      TOP\_TOP\_MUMU 	& RUN1B\_TOP\_TOP\_MUMU.D0DADF \\
      FAKE\_E\_CANDIDATE & RUN1B\_FAKE\_E\_CANDIDATE.D0DADF \\
      ELF\_W 		& RUN1B\_ELF\_W.D0DADF \\
      ELF\_Z 		& RUN1B\_ELF\_Z.D0DADF \\
      MU1\_WZT 		& RUN1B\_MU1\_WZT.D0DADF \\
      MU2\_WZT 		& RUN1B\_MU2\_WZT.D0DADF \\
      WZ\_ELF\_MU 	& RUN1B\_WZ\_ELF\_MU.D0DADF \\
      ELF\_MED 		& RUN1B\_ELF\_MED.D0DADF \\
      QCD\_STA 		& RUN1B\_QCD\_STA.D0DADF \\
      QCD\_GAP 		& RUN1B\_QCD\_GAP.D0DADF \\
      QCD\_JJJ 		& RUN1B\_QCD\_JJJ.D0DADF \\
      QCD\_QJT 		& RUN1B\_QCD\_QJT.D0DADF \\
      QCD\_NTR 		& RUN1B\_QCD\_NTR.D0DADF \\
      TOP\_BKG 		& RUN1B\_TOP\_BKG.D0DADF \\
      NP\_CLEAN\_TAU 	& RUN1B\_NP\_CLEAN\_TAU.D0DADF \\
      NP\_SCALAR\_TIGHT & RUN1B\_NP\_SCALAR\_TIGHT.D0DADF \\
      NP\_SQGL\_TIGHT 	& RUN1B\_NP\_SQGL\_TIGHT.D0DADF \\
      NP\_LQ\_2EM\_TIGHT & RUN1B\_NP\_LQ\_2EM\_TIGHT.D0DADF \\
      LQNUE 		& RUN1B\_LQNUE.D0DADF \\
      LQNN 		& RUN1B\_LQNN.D0DADF \\
      NP\_LSS\_TIGHT 	& RUN1B\_NP\_LSS\_TIGHT.D0DADF \\
      NP\_WRIGHT\_TIGHT & RUN1B\_NP\_WRIGHT\_TIGHT.D0DADF \\
      NP\_MULQ\_TIGHT 	& RUN1B\_NP\_MULQ\_TIGHT.D0DADF \\
      ELF 		& RUN1B\_ELF.D0DADF \\
      MINBIAS 		& RUN1B\_MINBIAS.D0DADF \\
      MU1\_FILT 	& RUN1B\_MU1\_FILT.D0DADF \\
      PI0\_FILTER 	& RUN1B\_PI0\_FILTER.D0DADF \\
      TWOJET\_FILTER 	& RUN1B\_TWOJET\_FILTER.D0DADF \\
      THREEJET\_FILTER 	& RUN1B\_THREEJET\_FILTER.D0DADF \\
      EMAG\_JETS 	& RUN1B\_EMAG\_JETS.D0DADF \\
      NP\_MULQ          & RUN1B\_NP\_MULQ.D0DADF \\
      TAU\_FILTER       & RUN1B\_TAU\_FILTER.D0DADF \\
      TOP\_SGL          & RUN1B\_TOP\_SGL.D0DADF \\
      TOP\_TOP\_SGL     & RUN1B\_TOP\_TOP\_SGL.D0DADF \\
      NEW\_GAM\_JET     & RUN1B\_NEW\_GAM\_JET.D0DADF \\
      SPR8              & RUN1B\_SPR8.D0DADF \\
      SPR9              & RUN1B\_SPR9.D0DADF \\
      SPR0              & RUN1B\_SPR0.D0DADF \\
   \hline
  \end{tabular}
  \caption{Standard streams and the corresponding D0DAD filenames.  This set
     is maintained by the production group.  The filenames shown are
     for the {\bf run 1b} data. The {\bf run 1a} filenames have 
     the string RUN1B replaced by RUN1A with the rest of the file name 
     unchanged.  
     All files are in the directory D0\$DATA\$D0DAD. \label{t-stand}}
  \end{center}
\end{table}  }% End small

\begin{figure}
   \centerline{\line(1,0){300}}
   \begin{verbatim}
   Menu: D0USER SETUP 
   
      1: Manual Process                      2: Auto Process
      3: Input Data File                     4: Output Data Files
      5: Set Interrupt Menu                  6: User Dialog
      7: Event Displays                      8: Define dumps
      9: Zero Summaries                     10: D0HPLT package
     11: QUIT                               12: Start Logging to File
     13: Stop Logging to File               14: Run Command File
     15: Switches                           16: Control Menu
   \end{verbatim}
   {\tt 
   \h4 Select: \{command, \#, HELP (\#), MENU, EXIT\} > { \bf input} \\
   \vh3 Will you be reading files from tape? [N]> { \bf N}
   }
   \begin{verbatim}
   If you give wild card all files with given string are processed
   but option Manual Process will stop after N events requested.
   If you give generic name no wild card will be allowed
   If generic name points to a tape stage will be invoked
   If you give generic name no wild card will be allowed
   If generic name points to a tape stage will be invoked
   If name of input file is NONE no file will be open.
   If name of input file is FILE_NAMES expect logical FILE_NAMES
   to be assigned to an ASCII file with lists of files.
   \end{verbatim}
   {\tt
   \h4 Input File Name> { \bf D0\$DATA\$D0DAD:RUN1B\_NP\_MET.D0DADF } \\
   \vh3 Input file modes are X (exchange), G (special X), or N (native) \\
   \vh3 File mode X,G or N ? [N]:> {\bf X}
   }

   \centerline{\line(1,0){300}}
   \vskip 2.0 truecm

  \caption{An example of the input file specification portion of a D0USER job 
      which reads the NP\_MET stream data using a D0DAD file.  The user's
      typing appears in bold.\label{f-easy}}
\end{figure}

\par
For each Zebra file successfully opened by D0DAD, a message of the form
\begin{verbatim}
     D0DAD: ProcessingFile: <filename>
\end{verbatim}
is printed.
If a Zebra data file cannot be opened by d0dad, an error message of the form:
\begin{verbatim}
    D0DAD_READ_EVENT-W :OPEN_FAILURE
    FailedFile: D0$DATA$DST:all_072814m01.x_mds01reu1210_all41_nonex00_4070812
\end{verbatim}
is printed using the standard D\O\ error message utility.  All such occurances
are printed, so the log file can be searched to find the name of files to be
excluded from the luminosity calculation.  

\par
In addition, the error-handling RCP parameters 
described in section~\ref{ss-ug-rcp} allow an analysis job to be stopped with 
a fatal error message if any unreadable files are encountered 
(See parameter FILE\_ERROR\_IS\_FATAL)
and allow the writing of a d0dad file containing references to all unreadable 
events (See parameter ERROR\_LIST\_FILE).

\section{RCP Control \label{ss-ug-rcp}}
\par
In order to provide finer control over the data read via D0DAD, an RCP file
can be used to select data based on run number and date.
Figure~\ref{f-RCP} contains an 
example of the optional RCP file {\em d0dad.rcp}\footnote{A copy of this RCP
file can be found in D0\$D0DAD:D0DAD.RCP on the Vaxes or in \$d0d0dad/d0dad.rcp
on Unix systems.}.  The RCP parameters are defined in table~\ref{t-rcp}.
This RCP file and any of the parameters are optional.  If the RCP file is not
found or if any parameter is missing, the parameter(s) is(are) set to have no 
effect.  For all jobs, a message is printed (via the ERRMSG utility) indicating
whether or not the RCP file was found.  The parameter values are not printed.
\begin{table}{\small
  \begin{list}{ }{\leftmargin 0.75 in \rightmargin 0.75 in }
     \item {\bf START\_RUN} - The run number of the first run to be analyzed.
     \item {\bf STOP\_RUN}  - The run number of the last run to be analyzed.
     \item {\bf VETO\_RUNS} - A list (RCP ARRAY) of runs to skip
     \item {\bf SELECT\_RUNS } - A list of runs to read.
     \item {\bf SELECT\_EVENTS } - A list of run/event ranges to be read. An
       entry is of the form ``RUN\# FirstEvent\# LastEvent\#''.  All events in
       run RUN\# between FirstEvent\# and LastEvent\# will be read.  The
       RCP array can hold up to 1000 entries.  The NEW\_RUNS\_xxxxxx.LIS files
       described above are formatted to match this entry and provide a means
       for reading data added on a given day.
     \item {\bf START\_TIME} - The
       time given in the START\_TIME array is of the form {\tt YYMMDD HHMM} in
       which {\tt YY} is the year, {\tt MM} the month, {\tt DD} the day, 
       {\tt HH} the hour and {\tt MM} the time.  For example, 
       June 12, 1994 at 9:30 am is specified as {\tt 940612 0930}.  All
       runs whose catalog time stamp is later than the date specified in 
       START\_TIME will be read.  The catalog time stamp is the time the run was
       entered.
     \item {\bf NSKIP} - Number of events to skip.  The number of events 
       skipped counter 
       is incremented only if a given run/event passes all of the above 
       requirements.
     \item {\bf NEVENTS} - The maximum number of events to read.  The counter
        is incremented only if a given run/event passes all of the above 
        requirements.
     \item { \bf RECO\_VERSION} - Allowed RECO version range. This is derived
        from the file name.
     \item { \bf OMNI\_VERSION} - Allowed OMNI\_FILTER version range. This
        is also derived from the filename.
     \item { \bf FILE\_ERROR\_IS\_FATAL} - Logical which, if true, means that 
        open errors for Zebra data files are fatal errors (Default: FALSE)
     \item { \bf ERROR\_LIST\_FILE} - Name of a D0DAD file to which events in
        bad files are written.  This can be used as the input file to another
        run which would recover all lost data.
  \end{list}}
  \caption{Definitions of the RCP parameters in D0DAD.RCP \label{t-rcp}}
\end{table}
\begin{figure}
  \centerline{\line(1,0){300}}
  \tiny
  \begin{center}
   \begin{verbatim}
     \START    D0DAD_RCP
     \SIZE         67        39
     !----------------------------------------------------------------------
     !    Name:       D0DAD.RCP
     !    Package:    D0DAD
     !    Library:    GENERAL
     !    Purpose:    Optional control to select input runs/events
     !    Version
     !    Created:    11-Jul-1994 John D Hobbs
     !----------------------------------------------------------------------
     !
     NSKIP            0   ! Number of events to skip
     NEVENTS          0   ! Maximum number to read (0==>Read all)

     START_RUN   57712    ! Lowest run number for reading
     STOP_RUN   100000    ! Highest run for reading
     \ARRAY VETO_RUNS     ! Up to 100 runs to be vetoed.
      12345
      36405   
     \END
     \ARRAY SELECT_RUNS   ! If this array has any entries, only runs listed 
       72736              ! will be read.
       73980
       74263
     \END
     \ARRAY SELECT_EVENTS ! List of run/event ranges (See NEW_RUNS_xxxx)
       81372       2   11364
       81372   11365   22852
       81382   23986   34378
       81384       2   10568
       81461       2   11297
       81461   11298   15372
     \END 

     FILE_ERROR_IS_FATAL .TRUE. ! Stop the job for read or open errors for 
                                !   Zebra data files
     ERROR_LIST_FILE     'BAD_EVENTS.D0DADF'
                                ! D0DAD file containing references to unread
                                !   events
     \ARRAY START_TIME  
      940101 1200         ! Earliest date for runs to read: DATE TIME
     \END
     \ARRAY RECO_VERSION  
      1210  1215          ! Read data RECO'd with versions 12.10 - 12.15
     \END
     \ARRAY OMNI_VERSION 
      30   41             ! Read data from OMNI_FILTER versons 3.0 to 4.1
     \END
     \STOP    !   D0DAD_RCP
   \end{verbatim}
  \end{center}
  \centerline{\line(1,0){300}}
  \vskip 1.0 truecm
  \caption{An example of the optional RCP file {\em d0dad.rcp}\label{f-RCP}}
\end{figure}

\section{Computing Luminosity \label{ss-ug-lumi}}
\par
The luminosity corresponding to the data referenced in a D0DAD file is computed
using the standard luminosity utilities GET\_FILE\_LUM.  To use 
these utilities, the run numbers used in producing the
stream must be known.  The file { D0\$DATA\$D0DAD:RUN1B\_MDS.FILECAT} is an
ASCII file containing a list of the files used to produce the master catalog.
The first line of this file is a header, and the remaining lines are the
list of files.  This file can be copied and editted to produce the list needed
by the luminosity utilities.  Each file-entry line contains four fields.  The 
first field is the Zebra file name and is all that is needed to get the 
luminosity.  A file is put in the file catalog even if no events are selected
into D0DAD files, so the file catalog contains the complete list of Zebra 
files.

\section{Creating Personal D0DAD Streams \label{ss-ug-personal}}
\par
\subsection{Creating D0DAD Streams from Run/Event lists \label{sss-ug-personal}}
Given a file containing run/event number pairs, users can
create D0DAD files referencing the events.  This section describes the steps
needed to create the D0DAD file.  The two steps are:
\begin{center}
\begin{tabular}{l}
  \$ d0setup d0dad \\
  \$ d0dad {\em input\_text\_file output\_d0dad\_file event\_catalog\_file} \\
\end{tabular}
\end{center}
The D0DAD command on the second line takes three filenames as arguments.  The
first file is a file containing run/event pairs, one pair per line, in a format
readable by a fortran ``*'' format descriptor.  The second file is the name of
the new D0DAD output file.  The third file is the name of a master catalog 
containing the run/event information for the desired input run/event pairs.
For {\bf run 1a} data the file catalog is 
{ D0\$D0DAD\$CATALOGS:RUN1A\_MDS.EVTCAT}, and for {\bf run 1b} 
data the catalog is
{ D0\$D0DAD\$CATALOGS:RUN1B\_MDS.EVTCAT}.
These catalogs reference only microDST's.  There are currently no plans to 
build catalogs for DST's or STA's. 

\par
If the input text file name is any of ``--'', ``stdin'' or ``SYS\$INPUT'', 
the run/event
pairs will be read from the standard input until an end-of-file\footnote{
On VMS machines, the end-of-file character is ctrl-Z; On unix machines, ctrl-D.}
is reached.  

\par
Other catalogs may be created. For example, the catalog TOPMC.EVTCAT contains
the
standard decay mixture of $t\overline{t} $ events generated at masses of 120,
140, 160 and 180 GeV.  All catalogs will be in the D0\$D0DAD\$CATALOGS 
directory\footnote{On D0 Unix systems, the catalog directory is defined by the
environment variable d0d0dad\_\_catalogs.}.  D0DAD files are transportable 
across platforms, so a file written on (e.g.) a Vax can be copied via rcp
to an SGI and used as is.  The reverse also holds.

\subsection{Creating Streams Based on Bitmasks \label{ss-ug-tstream}}
\par
It can be useful to produce a set of streams based on different criteria.  
Although this can be done by creating a set of run/event text files as described
in the preceeding section, that method is rather inefficient.  There is a new
command invoked via: \\
\centerline{d0dad /mode=text\_stream \it 
  bitmask\_file bitmask\_defn event\_catalog}
which produces a set of output d0dad files based on a text file
defining the correspondance between streams and events. Three files are needed:
\begin{enumerate}
\item {\it bitmask\_file} -- This file contains a list of run/event pairs and 
  the bitmasks defining the streams to which the event belongs.  The format of
  each line in the list is: \
  
  \centerline{         run\# event\# bbbb}

  in which bbbb is a hex bitmask.  A  set bit implies that the event is written
  to the corresponding stream.  Up to 160 streams may be defined and the entries
  must be sorted by run/event. 


\item {\it bitmask\_defn} -- This is a file which contains one line for each bit
  used.  The line contains the d0dad file name.  If no output file is desired
  for a given bit, the corresponding line must be blank.  For example if
  bit \#1 is used to create a file called BIT1.D0DADF, bit 2 is unused, bit \#3
  is used to create/update a file called BIT3.D0DADF and bit \#4 is used to
  create/update file called BIT4.D0DAD, the bitmask definition text
  file would be:
\begin{verbatim}
         BIT1.D0DADF
                                          <-- Blank line in file
         BIT3.D0DADF
         BIT4.D0DADF
\end{verbatim}

\item {\it event\_catalog} --  The event catalog defined as for the preceeding
  section
\end{enumerate}

\section{Creating DST's from D0DAD Streams.\label{ss-ug-copy}}
\par
It is also possible to invert the process of creating D0DAD files from Zebra
files.  The following two commands\footnote{The D0SETUP D0DAD command is 
necessary only once for each login window.} will create a Zebra file called
{\em mydata.mds} from the d0dad file called {\em d0dad\_file.d0dadf}:
\begin{center}
\begin{tabular}{l}
   \$ d0setup d0dad \\
   \$ d0dad /mode=copy {\em d0dad\_file.d0dadf mydata.mds} \\
\end{tabular}
\end{center}

\par
The additional command line options ``/run=\#'' and ``/event=\#' can be added
to the d0dad command to select a single run or event or a single range of runs
or events within a run.  For example, the command
\begin{center}
\begin{tabular}{l}
    \$d0dad /mode=copy /run=58203 {\em all\_runs.d0dadf r58203}
\end{tabular}
\end{center}
will create a $\mu$DST file called {\em r58203} containing the events from 
run 58203 found in the d0dad file {\em all\_runs.d0dadf}.  The allowed formats 
of these options are:
\begin{center}
\begin{tabular}{ll}
  /run=r1      & Select events in run number r1  \\
  /run=r1--r2  & Select events in run numbers r1 to r2 inclusive \\
  /event=r1    & Select events having event number r1   \\
  /event=r1--r2 & Select events having events numbers r1 to r2 inclusive \\
\end{tabular}
\end{center}
These options may be used together to select a subset of events from a set
of runs.  The command
\begin{center}
\begin{tabular}{l}
\$ d0dad /mode=copy /run=58203 /event=4980 ...
\end{tabular}
\end{center}
will select only event 4980 in run 58203.

\section{Formatted Printout Utility\label{ss-ug-utils}}
\par
Because the D0DAD event catalogs and d0dad files are binary data,\footnote{The 
file catalogs are in ascii format.}
a formatted dump utility has been written.  The command
\begin{center}
 \$ d0dad /mode=dump {\em file\_name}
\end{center}
will print an ascii summary of the file {\em file\_name}.  Here 
{\em file\_name} is the name of either an event catalog or a d0dad file.  
In addition to the /event and /run options as
described above, this command also accepts 
the /event option without any event number.
In this case, the summary listing is changed to include information for every
event in the catalog or d0dad file.


\end{document}





