\tracingcommands=1
\font\ninerm=cmr9 \font\eightrm=cmr8 \font\sixrm=cmr6 %\font\csc=cmcsc

\magnification=1200 % This magnifies everything by 1.2.
\parskip 10pt plus 1pt % This puts some empty space between paragraphs
\parindent 20pt  % paragraphs are not to be indended

\nopagenumbers  % this page ia not to be numbered
\null\vskip-46pt % put the first line 46 points higher than normal
\hbox to 6.5truein {\hfil D0 Report No. 1021}
  % a convenient way to make a box of specific size.
\vskip  .1in 
\line {\hfill Yi Xia }
\line {\hfill Mich. State Univ.}
\line {\hfill APR. 10,  1991}
\line {\hfill L2\_electron.doc}
\line {\hfill Version 00\_01}
\vskip 0.8in
\centerline{\bf How to Run VMS\_L2\_ELECTRON Filter}
\vskip .8in
\centerline{\bf Abstract}
\vskip 0.5in
This note is designed for  the users who have never run 
L2\_ELECTRON package
and VMS\_FILTER before.
Examples of running L1 and L2 simulation in detail of step by step 
are given.
You may find this file in D0\$CALOR\_FILTER:L2\_ELECTRON.DOC.

Due to the frequently updateing of D0software, the L2\_ELECTRON 
package may be 
crash during your use. Please send any error messages and questions to 
                         D0TEST::XIAYI. 
It will help the author to find bugs and follow the trace of 
the development
of D0software.
\vfill\eject
\footline={\hss\tenrm-- \folio\ --\hss}
\vskip 0.2in
\centerline{\bf A. Introduction }
\vskip 0.2in
L2\_ELECTRON is a software package. It works either in a 
Level\_2 node
to filter events under Level\_2 environment, or in purpose 
of trigger simulation in conjunction
with the VMS\_FILTER 
package. 

Due to the limited time and memory budget in D0 L2 
environment, L2\_ELECTRON is a carefull designed filter which 
is not only
faster but also has a quite strong rejection 
power to reduce the background EM trigger rate. It may run togather
with the D0user CAHITS
package or run independently within its own self closed 
software package. In combination with CAHITS you may look the 
output of L2\_ELECTRON directly by adding your analysis package
just after the L2\_ELECTRON. It is faster when you run L2\_ELECTRON
under its own system. 

L2\_ELECTRON has a logic flag. 
The L2\_framework will consider to pass or reject this event
depending on the status of this logic flag.
When there is a electron (or more) which 
was found by L2\_ELECTRON in a 
event, it will set the logic flag to TRUE, otherwise a FALSE 
logic will be 
assigned to this flag. The value of this flag will be passed to 
L2\_framework
which will consider all results of all other releated filters to make
the final decision.

\vskip 0.3in
\centerline{\bf B. A Sketch of L2\_ELECTRON}
\vskip 0.1in
From the Level\_1 TRGR bank which is filled by Level\_1 trigger 
framework
or by LEVEL\_1 simulation process, L2\_ELECTRON obtains a 
list of electron
candidates with the address of the hot trigger towers for every event.  

L2\_ELECTRON cycles through this list of candidates (a list
 of hot trigger
towers which have the EMt showers above a certain threshold
 and the HDt lower
the veto threshold) to find the electrons.

If the number of candidates from Level\_1 TRGR is not zero, then 
L2\_ELECTRON
unpacks the CAD1 and CAD2 data bank of calorimetor data.  Combainning 
the above 
Level\_1 hot tower address and the calorimetor data, L2\_ELECTRON 
finds the positions of the possible EM 
shower first.
 
For separating the shower caused by $\gamma$s or $\pi^0$s
with electron,
L2\_ELECTRON uses track marching package to reject these 
fake candidates. Only
the candidates which passed this track marching check will 
meet the shower shape
check. Surrounding the shower axis of a candidate 
L2\_ELECTRON measures the 
longitudinal and transverse shape of this shower 
using the 3\_D energy 
deposition distribution in the calorimetor. L2\_ELECTRON uses 
energy information
in five layers (EM1,EM2,EM3,EM4 and HD1) and the total EM shower 
EMtot to check
the longitudinal shape which is represented by the distribution of the 
correlations of energy betweem the above six values. The transverse 
shower 
shape of the candidate is based on the calculation of its transverse 
shower 
radius and the shower diffusivity which is defined at 
page 8 of the D0 NOTE 869. 

The candidate which failed the above check will be rejected. 
If the number
of candidates which were satisfied as a electron reaches the quota 
that you ask 
for (you set this quota in the RCP file for L2\_ELECTRON) the 
event's flag will
be set TRUE, and this checking cycle stops. If there is none 
candidate which
was passed 
the check then the event's logic flag will be set FALSE.

\vskip 0.2in
\centerline{\bf  C. How to Run L2\_ELECTRON}
\vskip 0.2in
\+\bf I)  The RCP file to control L2\_ELECTRON\cr

L2\_ELECTRON uses a RCP file which is referred to by the logical
\item{name:} 
{\bf L2ELCT\_RCP}
and can be found in D0\$CALOR\_FILTER:L2ELCT.RCP. 

This RCP file contains whole cuts parameters and constants which are used
by the L2\_ELECTRON. The parameters which user releated are 
list and explained as
follow: (do not try to modify other parameters except bellow)

{\bf   ETCUT:} Level\_2 Et Cut on energy which 
deposited in the Hot Trigger Tower in CC region.

{\bf   EETCUT:} Level\_2 Et Cut on energy which deposited 
in the Hot Trigger Tower in EC region.

{\bf   ELIMIT:} Quota of the electrons passed the check
 should be reached, if the event's flag becomes TRUE.

{\bf   CCCUT:} If it is TRUE, the CC part longitudinal shape 
check is turn on.

{\bf   TRCUT:} If it is TRUE, the CC part transverse shape 
check is turn on.

{\bf   ECCUT:} If it is TRUE, the EC part longitudinal shape 
check is turn on.

{\bf   ECTRA:} If it is TRUE, the EC part transverse shape 
check is turn on.

There are more parameters, but they are not for modification 
by user right now.
Due to the differency of shower shape between data and Monte 
Calro, the above
RCP file which services for Monte Carlo will not good for the
 testbeam data. 
User should check the subtitle inside the RCP file, make sure
 it consistents
with the data type which you want to work on.

\vskip 0.4in
\vfill\eject
\+\bf II) How to run L1 simulation\cr 

Before you try to run VMS\_FILTER on a data set, you have to let this data
set pass through the Level\_1 simulation. A package called 
L1C which does this
job. Please see D0\$LEVEL1:LEVEL1\_SIMULATION.DOC to learn 
how to use it. Data
without TRGR bank will be not accepted by the VMS\_FILTER.


You may follow the steps which I give you here to run L1 trigger simulation
(EM part) without diging into the hell of L1C.

The instructions of step by step are :

\item{A.}  Create your working area in your account (right now it is
better that you work in 
D0SF cluster)

{\bf    [xxxx.L1]}

\item{B.}  type : \$LIBT ~~ LEVEL1

\item{C.}  type : \$COPY ~~ D0\$LEVEL1:L1C.RCP ~~ [xxxx.L1]*.*

\item{D.}  type : \$COPY ~~ D0\$LEVEL1:LEVEL1\_CONFIG\_0000.DAT ~~ [xxxx.L1]

\item{E.}  EVE your L1C.RCP,  change

      "D0\$LEVEL1:LEVEL1\_CONFIG\_0000.DAT"

\item{      to}

      "your\_disk:[xxxx.l1]LEVEL1\_CONFIG\_0000.DAT"

\item{F.}  Modify the part of EMt Hot Tower Reference Set 
Definition in your

    {\bf [xxxx.l1]LEVEL1\_CONFIG\_0000.DAT}  file. 

There are four references which
have been setup. You may change the Threshold and Veto. The maximum
number of refrences is 4 , so if you want to test more then four 
Threshold, you have to run the simulation more then one time.

\item{G.}  Modify the part of Specific Trigger Definition in above file too. It 
    should set some trigger bits to corresponding to your Emt Hot Tower
    Reference Set definition. Later on, when the trig bit fired it will
    mean this event satisfied this trigger definition.  Do not forget 
that you should active these triggers which you have defined here at 
the end of this configuration file.


\item{H.}  type:  {\bf    \$PBD/D0USER/PACKEDGE=L1C}

\item{It}  creates {\bf L1C\_D0USER.LNK and L1C\_D0USER.COM} file .
You should modify the  .COM file, 

replace:

     \$define L1C\_RCP D0\$LEVEL1:L1C.RCP
\item{   by}

     \$define L1C\_RCP your\_disk:[xxxx.l1]L1C.RCP

\item{I.}  type : \$@L1C\_D0USER.COM

           \$@L1C\_D0USER.LNK ~~~ (p1 is DEBUG feature, if you want)

    Now you should have{\bf L1C\_D0USER.EXE} in your area.

\item{J.}  You can run interative mode :
    now by  type:   \$D0USER/FULL  (answer the question ,choose proper menu
    which will let you input your data file and choice the output file name.
    But a lot of item in the menu are not ready yet, so be ware do not 
    attend to run a fancy job which uses many features in the menu)

\item{K.}  For Batch job, you may copy L1C\_BATCH.COM and
     SETUP\$L1.INP  from

     D0SFXX::USR\$ROOT1:[XIAYI.L1\_EXAMPLE] to your [xxxx.L1] area.

    Modify the SETUP\$L1.INP file (data file name of the input and output)

    TYPE \$SUBMIT/QUE=qqqq   L1C\_BATCH

    The batch job L1C\_BATCH should now run in the que :qqqq

\vskip 0.4in
\+\bf III) How to Use L2\_ELECTRON with VMS\_FILTER Package\cr

The LEVEL\_2 FRAMEWORK runs under VAXELN environment instead of VMS. For 
offline simulation of the Level\_2 trigger, a VMS\_FILTER package 
was wrotten by
Jan Hoftun which allows the user to simulate Level\_2 trigger 
without COOR and
VAXELN environment. Please read D0\$L2CONTROL:LEVEL2\_MANUAL.MEM 
for more  information.

For first time user of the VMS\_FILTER, a area in my account 
at D0SF cluster
in which several files that will help you to run the L2\_ELECTRON has been 
created. So please follow these instructions in your first try.

\item{  A.} Create a working ares [xxxx.L2] in your account.

\item{  B.} copy 

    D0SFXX::USR\$ROOT1:[XIAYI.L2\_example]D0\_FILTER.OPT and
          
    D0SFXX::USR\$ROOT1:[XIAYI.L2\_example]SETUP.COM and
    
    D0SFXX::USR\$ROOT1:[XIAYI.L2\_example]L2TOOL.DAT and

     D0SFXX::USR\$ROOT1:[XIAYI.L2\_example]FILTER\_MAKER.EXAMPLE and

     D0SFXX::USR\$ROOT1:[XIAYI.L2\_example]SETUP\$L2\_ELECTRON.INP and

     D0SFXX::USR\$ROOT1:[XIAYI.L2\_example]MAKE\_RUN\_FILE.COM and

     D0SFXX::USR\$ROOT1:[XIAYI.L2\_example]SETUP\$L2OUT.INP and

     D0SFXX::USR\$ROOT1:[XIAYI.L2\_example]L2ELCT.RCP 
    
   into your [xxxx.L2] area .

       You should modify the DIRECTORIES which used in above file to your
       local directories. For example , replace the {\bf[XIAYI.L2]} 
by {\bf[xxxx.L2].}

\item{  C.} According to your need to modify the L2\_ELECTRON 
run control file L2ELCT.RCP.

\item{  D.} Print out the FILTER\_MAKER.EXAMPLE. 

Step by step follow this example,
       you should obtain 

{\bf VMS\_FILTER\_D0USER.LNK} and
                         
{\bf VMS\_FILTER\_D0USER.COM} and
                         
{\bf VMS\_FILTER\_D0USER.EXE}  

     The procedure in this example shows you how to create a 
VMS\_FILTER which uses the CAHITS package.

\item{  E.} EVE the SETUP\$L2\_ELECTRON.INP file to replace the 
trigger bit, now it is
       set to 18 in that file, to the trigger bit which you are 
interested in.

\item{  F.} Submit batch job MAKE\_RUN\_FILE.COM

\item{  G.} Copy CAHITS.RCP from D0\$calor\_off into your area. You 
should modify
       this file make sure the Flag for Monte Carlo data is turned on,
if you are working on Monte Carlo data.
  Of course you also should
       modify the VMS\_FILTER\_D0USER.COM file, let it define the CAHITS\_RCP
       as your CAHITS.RCP file.

\item{  H.} Now you may run the L2\_ELECTRON by type :

       \$@SETUP.COM

       \$@VMS\_FILTER\_D0USER.COM

       \$ D0USER/FULL

\vskip 0.1in
\centerline{ Choose the menu which you want, and now all are yours.}

\vskip 0.8in
\centerline{\bf D. Acknowledgement}
\vskip 0.1in
I would like to thank Maris Abolins and Jan Hoften for their helps
on understanding the L1C and VMS\_FILTER software .
\vfill\eject\end %\eject, only, would be used if other pages were to follow.


