\newpage
\part{CPC INFOCACHE Routines Reference}
\vspace*{1cm}
\begin{verbatim}
  List of INFOCACHE routines documented in this section:

  IC_BOOK       book a cache
  IC_BOOKED     logical, true if a cache is booked

  IC_READ       INFOCACHE mode: read cache
  IC_WRITE      INFOCACHE mode: write cache

  IC_SIGNAL     MAILBOX mode:   send signal to master
  IC_SEND       MAILBOX mode:   send reply to client

  IC_ALARM      ALARM mode:     send an alarm

  IC_BROADCAST  BROADCAST mode: send a broadcast

  IC_WATCH      WATCHER mode:   set up a watch on the lockholder
  IC_STATUS     WATCHER mode:   load status info into lock cache

  IC_TAKE_LOCK  RESOURCE LK mode: lock the resource              NEW
  IC_DROP_LOCK  RESOURCE LK mode: drop lock on the resource      NEW
  IC_CHECK_LOCK RESOURCE LK mode: check resource lock status     NEW

  IC_ASSOCIATE  general. Associate an AST routine with a cache
  IC_NUM_ID     general. Convert cache number to name in AST
  IC_DELIVER    general. Read message in AST

  IC_WAIT       general. Wait for an interrupt from a cache

  IC_INFO       general. To obtain information on caches

  IC_MATCH      general. To obtain lists of cache groups

  IC_RELEASE    general. To un-book a cache
\end{verbatim}

\begin{routine} %RRRRRRRRRRRRRRRRRRRRRRRRRRRRRRRRRRRRRRRRRRRRRRRRRRRRR
\subroutine
   {SUBROUTINE}{IC\_ALARM}{(CANAME, CADAT, CHOPT, IERR)}
\begin{overview}
An  \alarm mode routine.

Sends an alarm.
\end{overview}
\begin{argdeflist}
\argitem{CANAME}{character/readonly}{The name assigned to the cache}
\argitem{CADAT}{16 bytes/readonly}{The cache contents, \ie the alarm
  message}
\argitem{CHOPT}{character/readonly}{Option string, upper case}
\argitem{IERR}{integer/modify}{Error code. Error if nonzero}
\end{argdeflist}
\begin{describe}

An alarm is sent to all
processes which have booked the cache in `alarm server' mode.

The contents of CADAT are sent with the alarm as an alarm
message, and can
be retrieved by processes receiving the alarm interrupt in the same
way that broadcasts are retrieved (see routines
\verb|IC_WAIT| and \verb|IC_DELIVER|).

The alarm message might contain, for instance
\begin{itemize}
\item a message code
  as generated by the VMS MESSAGE utility (4 bytes) 
\item an identifier
  of the sending process (1 to 4 bytes)
\item additional alarm information in the remaining 8 to 11 bytes.
\end{itemize}

\end{describe}
\begin{options}
\opt{none}{}
\end{options}
\begin{returncodes}
\rcode{any {\tt IC\_BROADCAST} error code}{}
\end{returncodes}
\end{routine}

\begin{routine} %RRRRRRRRRRRRRRRRRRRRRRRRRRRRRRRRRRRRRRRRRRRRRRRRRRRRR
\subroutine
   {SUBROUTINE}{IC\_ASSOCIATE}{(CANAME, CADAT, ASTSUB, CHOPT, IERR)}
\begin{overview}
A routine used by interrupt receiving processes in \broadcast,
\alarm, or \mailbox mode. It associates an AST routine \verb|ASTSUB|
with the specified cache.

\end{overview}
\begin{argdeflist}
\argitem{CANAME}{character/readonly}{The name assigned to the cache}
\argitem{CADAT}{16 bytes/readonly}{Not used at present}
\argitem{ASTSUB}{external routine}{The AST routine name}
\argitem{CHOPT}{character/readonly}{Option string, upper case}
\argitem{IERR}{integer/modify}{Error code. Error if nonzero}
\end{argdeflist}
\begin{describe}

This routine can be used to associate an AST routine with a specified
cache through which interrupts are received. When an interrupt is
received, the AST routine is called and user code in the AST routine
receives and processes the message. The AST might process
alarms in \alarm mode, broadcasts in \broadcast mode, or client messages
in \mailbox mode.

Inside the user's AST routine, the routine \verb|IC_DELIVER| {\em must} be
called. This routine reads the message from the sender and completes
the message handshaking. In the case of \mailbox mode, a master must call
\verb|IC_SEND| after \verb|IC_DELIVER| in the AST routine to send the
reply to the client (and complete the handshaking).

Using this routine, a single process can handle interrupts from many
sources at the same time. The coding example below illustrates how
one AST routine processes interrupts from more than one cache.

While the process is waiting for interrupts
from any of its caches it can do other things.

For example, a display program might have one AST routine associated with
a broadcast cache, to receive and display general broadcast messages;
several alarm caches associated with another AST routine to receive and
signal alarms; and another AST routine associated to
a \mailbox type cache in master mode to receive requests from other
processes or interactive users. The main program in this example might
be a loop with a 30 second hibernate, which refreshes the
screen displays when it wakes up every 30 secs.

\end{describe}
\begin{options}
\opt{none}{}
\end{options}
\begin{returncodes}
\rcode{IC\_E\_INVMODE}{cache is not an appropriate usage mode.}
\end{returncodes}
\begin{examplecode}\begin{verbatim}
*
        EXTERNAL ASTALARM, ASTCMD
        INTEGER ICACHE(4)
        ...
        CALL IC_BOOK('HVALARM',ICACHE,'S',IERR)
        IF (IERR.NE.0) GOTO 999
        CALL IC_ASSOCIATE('HVALARM',ICACHE,ASTALARM,' ',IERR)
        IF (IERR.NE.0) GOTO 999
        CALL IC_BOOK('FBALARM',ICACHE,'S',IERR)
        IF (IERR.NE.0) GOTO 999
        CALL IC_ASSOCIATE('FBALARM',ICACHE,ASTALARM,' ',IERR)
        IF (IERR.NE.0) GOTO 999
        CALL IC_BOOK('MYCMDS',ICACHE,'M',IERR)
        IF (IERR.NE.0) GOTO 999
        CALL IC_ASSOCIATE('MYCMDS',ICACHE,ASTCMDS,' ',IERR)
        IF (IERR.NE.0) GOTO 999
        DO WHILE (.TRUE.)
          CALL REFRESH_DIPLAY
          CALL LIB$WAIT(30.)
        ENDDO
        ...
*
        SUBROUTINE ASTALARM(NUM)
*         the NUM parameter passed to the AST routine is the
*         cache identifier, which is converted to the cache name
*         by the routine IC_NUM_ID below.
        CHARACTER NAME*30
        INTEGER ICACHE(4)
        CALL IC_NUM_ID(NUM,NAME,IERR)
        IF (IERR.NE.0) GOTO 999
*         receive message. Mandatory!
        CALL IC_DELIVER(NAME,ICACHE,' ',IERR)
        IF (IERR.NE.0) GOTO 999
        IF (NAME(1:LENOCC(NAME)).EQ.'HVALARM') THEN
          ...
        ELSE IF (NAME(1:LENOCC(NAME)).EQ.'FBALARM') THEN
          ...
        ENDIF
        END
*
        SUBROUTINE ASTCMDS(NUM)
        CHARACTER NAME*30
        INTEGER ICACHE(4)
        CALL IC_NUM_ID(NUM,NAME,IERR)
        IF (IERR.NE.0) GOTO 999
*         receive message from the client. Mandatory!
        CALL IC_DELIVER(NAME,ICACHE,' ',IERR)
        IF (IERR.NE.0) GOTO 999
        <act, based on contents of ICACHE, and fill ICACHE with a reply>
*         send reply to the client. Mandatory!
        CALL IC_SEND(NAME,ICACHE,' ',IERR)
        IF (IERR.NE.0) GOTO 999
        END
\end{verbatim}\end{examplecode}
\end{routine}

\begin{routine} %RRRRRRRRRRRRRRRRRRRRRRRRRRRRRRRRRRRRRRRRRRRRRRRRRRRRR
\subroutine
   {SUBROUTINE}{IC\_BOOK}{(CANAME, CADAT, CHOPT, IERR)}
\begin{overview}
This procedure is called to book a cache. The usage mode is determined
by the option selected.
\end{overview}
\begin{argdeflist}
\argitem{CANAME}{character/readonly}{The name assigned to the cache.
  Maximum length is 20 characters.}
\argitem{CADAT}{16 bytes/modify}{For some usage modes, the existing value
of the cache is returned, or an initialization value is passed.}
\argitem{CHOPT}{character/readonly}{Option string, upper case}
\argitem{IERR}{integer/modify}{Error code. Error if nonzero}
\end{argdeflist}
\begin{describe}
This routine books the cache of the specified name for use in the program,
with the usage mode determined by the option selected.

This routine must be called before any other IC routines are called for
a particular cache.

Only \em{one} mode option can be passed for a given cache (a cache can
have only one usage mode!).
\end{describe}
\begin{options}
\opt{none}{ Default usage mode is \info. Any number of processes can book
  a cache in this mode. All can use {\tt IC\_WRITE}, {\tt IC\_READ} to
  write and read the cache.}
\opt{M}{ \mailbox mode: master process. The calling process becomes the
  master (message receiver) of the cache. {\tt IC\_ASSOCIATE} must be
  called after {\tt IC\_BOOK}. Only {\em one} process can book a given
  cache as a master.}
\opt{C}{ \mailbox mode: client process. Any number of clients can book
   a given cache. Any client can use {\tt IC\_SIGNAL} to communicate with
   the master.}
\opt{B}{ \broadcast mode. Any number of processes can book a broadcast
  cache. All must call {\tt IC\_ASSOCIATE} or {\tt IC\_WAIT} after
  {\tt IC\_SETUP}.}
\opt{L}{ \watcher mode: program lock. The process becomes the lockholder
  of the cache. The lock is relenquished (to a watcher) when the process
  crashes or when it deliberately releases the lock. The lockholder can
  use {\tt IC\_STATUS} to load process status info into the cache.}
\opt{P}{ \watcher mode: program status watcher. Any number of processes
  who want to monitor the status of the lockholder and receive notification
  when it crashes can book the cache in this mode and call {\tt IC\_WATCH}.}
\opt{T}{ \alarm mode: task which sends out alarms. Any number of processes
   can book an alarm cache with this option. These processes can use
  {\tt IC\_ALARM} to send alarms to listening servers.}
\opt{S}{ \alarm mode: alarm server which receives and processes alarms
  sent by alarm tasks (cache holders with option {\tt T}).
  Any number of processes can book an alarm cache as a server. These
  processes must call {\tt IC\_ASSOCIATE} after {\tt IC\_BOOK}.}
\opt{R}{ \reslock mode: Book the resource lock. After booking the resource lock,
  {\tt IC\_TAKE\_LOCK} can be called to take out the lock.}
\opt{X}{ cache should exist. Error returned if it does not.}
\opt{Y}{ system mode. Cache is recognised across all UICs. See the
  technical section.}
\end{options}
\begin{returncodes}
\rcode{IC\_E\_BADNAME}{ cache name too long}
\rcode{IC\_E\_MAXEX}{ maximum number of caches exceeded}
\end{returncodes}
\end{routine}

\begin{routine} %RRRRRRRRRRRRRRRRRRRRRRRRRRRRRRRRRRRRRRRRRRRRRRRRRRRRR
\subroutine
   {LOGICAL FUNCTION}{IC\_BOOKED}{(CANAME)}
\begin{overview}
Checks whether a cache is booked.

\end{overview}
\begin{argdeflist}
\argitem{CANAME}{character/readonly}{The name assigned to the cache}
\end{argdeflist}
\begin{describe}

Returns TRUE if the cache is booked.

\end{describe}
\begin{options}
\opt{none}{}
\end{options}
\end{routine}

\begin{routine} %RRRRRRRRRRRRRRRRRRRRRRRRRRRRRRRRRRRRRRRRRRRRRRRRRRRRR
\subroutine
   {SUBROUTINE}{IC\_BROADCAST}{(CANAME, CADAT, CHOPT, IERR)}
\begin{overview}
A  \broadcast mode routine.

Sends a broadcast.
\end{overview}
\begin{argdeflist}
\argitem{CANAME}{character/readonly}{The name assigned to the cache}
\argitem{CADAT}{16 bytes/readonly}{The cache contents written.}
\argitem{CHOPT}{character/readonly}{Option string, upper case}
\argitem{IERR}{integer/modify}{Error code. Error if nonzero}
\end{argdeflist}
\begin{describe}

A broadcast is sent to all processes which have the specified cache
booked (and have enabled interrupt reception with a call to
{\tt IC\_ASSOCIATE} or {\tt IC\_WAIT}). 
The contents of CADAT are sent with the broadcast, and are
retrieved by processes receiving the broadcast interrupt via
{\tt IC\_WAIT} or (if {\tt IC\_ASSOCIATE} is used) \verb|IC_DELIVER|.

This routine has for technical reasons a delay of 100 msec built into
it at present. This delay will be reduced if possible in a future
version. There will always be a delay equal to the longest time taken
to receive the broadcast by any of the processes that has the cache booked.

\end{describe}
\begin{options}
\opt{none}{}
\end{options}
\begin{returncodes}
\rcode{IC\_E\_NOBCAST}{cache type is not alarm or broadcast}
\end{returncodes}
\end{routine}

\begin{routine} %RRRRRRRRRRRRRRRRRRRRRRRRRRRRRRRRRRRRRRRRRRRRRRRRRRRRR
\subroutine
   {SUBROUTINE}{IC\_CHECK\_LOCK}{NAME,HELD,IU,IP,IC,PRC,CPU,CHOPT,IERR)}
\begin{overview}
Resource lock mode. Check the status of a resource lock.

\end{overview}
\begin{argdeflist}
\argitem{NAME}{character/readonly}{The resource lock name}
\argitem{HELD}{logical/modify}{whether the lock is held by another process}
\argitem{IU}{8 bytes/modify}{user-defined component of the lock's cache}
\argitem{IP}{integer/modify}{process ID of the lock holder}
\argitem{IC}{integer/modify}{CSID of the lock holder, for experts playing with
 GETSYI}
\argitem{PRC}{character/modify}{process name of the lock holder}
\argitem{CPU}{character/modify}{node name of the lock holder}
\argitem{CHOPT}{character/readonly}{Option string, upper case}
\argitem{IERR}{integer/modify}{Error code. Error if nonzero}
\end{argdeflist}
\begin{describe}

This routine enables a process to determine who is holding a resource lock.
Note that in VMS 5.1 the process ID is unknown when the lockholder is
running on a remote node. This may be rectified in VMS 5.2.

The HELD logical is true if a process anywhere on the cluster currently
holds the lock; false otherwise.

A user-defined code identifying the lockholder can be loaded into the
lock's cache in the \verb|IC_TAKE_LOCK| call and can then be read in
the call to this routine.

\end{describe}
\begin{options}
\opt{none}{}
\end{options}
\begin{returncodes}
\rcode{IC\_E\_INVMODE}{cache is not the correct usage mode.}
\end{returncodes}
\end{routine}

\begin{routine} %RRRRRRRRRRRRRRRRRRRRRRRRRRRRRRRRRRRRRRRRRRRRRRRRRRRRR
\subroutine
   {SUBROUTINE}{IC\_DELIVER}{(CANAME, CADAT, CHOPT, IERR)}
\begin{overview}
A routine used within AST routines to receive a message. Used in
\mailbox, \broadcast, and \alarm modes.

\end{overview}
\begin{argdeflist}
\argitem{CANAME}{character/readonly}{The name assigned to the cache}
\argitem{CADAT}{16 bytes/modify}{The cache contents at time of interrupt,
  \ie the message from the sender}
\argitem{CHOPT}{character/readonly}{Option string, upper case}
\argitem{IERR}{integer/modify}{Error code. Error if nonzero}
\end{argdeflist}
\begin{describe}

This routine is called inside an AST routine to read the message coming
from the signaller (whether the signaller is a broadcaster, a client,
or an alarm task, depending on the usage mode).

This routine {\em must} be called, even in the event that the
message itself is not of interest. It is mandatory because the handshaking
between sender and receiver is completed in this routine (except for
\mailbox mode, in which the handshaking is completed by \verb|IC_SEND|).

The message is returned in CADAT.

\end{describe}
\begin{options}
\opt{C}{read the cache value from the data structure. Value is {\em not}
  read from the cache itself. An rarely needed option.}
\end{options}
\begin{returncodes}
\rcode{IC\_E\_INVMODE}{cache is not the correct usage mode.}
\end{returncodes}
\begin{examplecode}\begin{verbatim}
*
**        see example for IC_ASSOCIATE.
*
\end{verbatim}\end{examplecode}
\end{routine}

\begin{routine} %RRRRRRRRRRRRRRRRRRRRRRRRRRRRRRRRRRRRRRRRRRRRRRRRRRRRR
\subroutine
   {SUBROUTINE}{IC\_DROP\_LOCK}{NAME, CHOPT, IERR)}
\begin{overview}
Resource lock mode. Drops the lock.

\end{overview}
\begin{argdeflist}
\argitem{NAME}{character/readonly}{The resource lock name}
\argitem{CHOPT}{character/readonly}{Option string, upper case}
\argitem{IERR}{integer/modify}{Error code. Error if nonzero}
\end{argdeflist}
\begin{describe}

This routine drops the specified lock, if it is held. If it is not held,
the routine returns quietly. It loads the lock's cache with zeros before
dropping.

\end{describe}
\begin{options}
\opt{none}{}
\end{options}
\begin{returncodes}
\rcode{IC\_E\_INVMODE}{cache is not the correct usage mode.}
\rcode{IC\_E\_INACTIVE}{AST delivered during releasing of the cache,
  and should be ignored.}
\end{returncodes}
\end{routine}

\begin{routine} %RRRRRRRRRRRRRRRRRRRRRRRRRRRRRRRRRRRRRRRRRRRRRRRRRRRRR
\subroutine
   {SUBROUTINE}{IC\_INFO}{(CANAME, IMODE, LDATA, CHOPT, IERR)}
\begin{overview}
A general routine. Not restricted to a particular usage mode.

Used to obtain information on caches.
\end{overview}
\begin{argdeflist}
\argitem{CANAME}{character/readonly}{The name assigned to the cache}
\argitem{IMODE}{integer/readonly}{The usage mode of the cache. Only needed
  if the cache for which information is requested has not been booked. If the
  cache has been booked, IMODE can be 0. For info on usage mode parameters
  see the technical section.}
\argitem{LDATA}{int array/modify}{The cache info returned. Contents
depend on selected option.}
\argitem{CHOPT}{character/readonly}{Option string, upper case}
\argitem{IERR}{integer/modify}{Error code. Error if nonzero}
\end{argdeflist}
\begin{describe}

The various options of this routine permit the retrieval of information
associated with the cache CANAME (except option {\tt I} for which CANAME
is a returned string). See the descriptions of the options below.

The option {\tt `LP'} with the name specification {`*'} provides a
dump to the terminal of known caches, useful for debugging.

\end{describe}
\begin{options}
\opt{C}{return cache contents in LDATA; 16 bytes}
\opt{N}{number of processes with cache booked returned in LDATA(1).
Can be used to check cache existence.}
\opt{L}{list of processes holding this cache dumped to screen. Nothing
returned in LDATA (yet)}
\opt{M}{return usage mode of cache in IMODE. See discussion on 
\INFOCACHE parameters in the technical section for the usage mode codes.}
\opt{S}{return ID numbers of all caches containing the string
NAME. LDATA(1) is number of caches returned;
thereafter follow ID numbers. After ID numbers,
in LDATA(1+LDATA(1)+I), are mode codes. Should not be called directly by
user; called from {\tt IC\_MATCH}.}
\opt{I}{a cache ID is passed in LDATA. The name of the cache is
returned in CANAME.}
\opt{P}{Print. Can be used in conjunction with {\tt C,N,M,S} options to
send info to screen as well as return it in LDATA.}
\end{options}
\begin{returncodes}
\rcode{IC\_E\_UNKNOWN}{ cache has unknown usage mode. System error.}
\rcode{IC\_E\_NOMATCH}{ cache name is unknown}
\rcode{IC\_E\_BADMODE}{ unknown lock mode. System error.}
\end{returncodes}
\end{routine}


\begin{routine} %RRRRRRRRRRRRRRRRRRRRRRRRRRRRRRRRRRRRRRRRRRRRRRRRRRRRR
\subroutine
   {SUBROUTINE}{IC\_MATCH}{(MATCH, NMATCH, IDNUM, IMODE, CHOPT, IERR)}
\begin{overview}
A general routine for information retrieval. 
Not restricted to a particular usage mode.

Retrieves groups of caches satisfying a character string match.

\end{overview}
\begin{argdeflist}
\argitem{MATCH}{character/readonly}{The string to match}
\argitem{NMATCH}{integer/modify}{The number of caches with names containing
  the specified string}
\argitem{IDNUM}{int array/modify}{The array of cache IDs returned. There
  are {\tt NMATCH} elements. {\tt IC\_INFO} can be used to convert the ID
  to a cache name (see coding example below)}
\argitem{IMODE}{int array/modify}{The array of cache usage modes returned.
  There are {\tt NMATCH} elements.}
\argitem{CHOPT}{character/readonly}{Option string, upper case}
\argitem{IERR}{integer/modify}{Error code. Error if nonzero}
\end{argdeflist}
\begin{describe}

Given a string to match, {\tt MATCH}, this routine returns
 in array IDNUM a list of
caches whose names contain the string. Array {\tt IMODE}
contains usage mode codes for the different caches. See the technical
section for more info on usage mode codes.
NMATCH is the number of matches found.

This routine provides a means of implementing cache groups.
Cache groups can be established by using part
of each cache name as a unique group identifier. This routine can
then be used to retrieve the members of a group and perform an
operation on some or all of the group member (see the example below).

NOTE that only those caches which a process {\em on the local node}
has booked are returned in the list! That is, if cache 
\verb|/MU/HV| is booked
by a process on VXL3ON but not by any processes on VSL3RC, a
process on VSL3RC which executes the example below will not find
any matches.

\end{describe}
\begin{options}
\opt{none}{}
\end{options}
\begin{returncodes}
\rcode{any {\tt IC\_INFO} return code}{}
\end{returncodes}
% $$$ test
\begin{examplecode}\begin{verbatim}
*    obtain a list of all MUon caches (caches with /MU/ in the name).
   CALL IC_MATCH('/MU/',NCACHE,ID,IM,' ',IERR)
*    process each of the caches in the retrieved group
   DO I=1,NCACHE
     CALL IC_INFO(CNAME,0,ID(I),'I',IERR) ! cache name is returned
*      one might for example want to send a signal to each
*      mailbox cache in the group.
     IF (IM(I).EQ.IC__MAILBOX) CALL IC_SIGNAL(CNAME,...)
   ENDDO
\end{verbatim}\end{examplecode}
\end{routine}


\begin{routine} %RRRRRRRRRRRRRRRRRRRRRRRRRRRRRRRRRRRRRRRRRRRRRRRRRRRRR
\subroutine
   {SUBROUTINE}{IC\_NUM\_ID}{(NUM, CANAME, CHOPT, IERR)}
\begin{overview}
Converts a cache numeric identifier to a string identifier.
\end{overview}
\begin{argdeflist}
\argitem{NUM}{integer/readonly}{numeric identifier of cache}
\argitem{CANAME}{character/modify}{cache name; allow up to 30 characters}
\argitem{CHOPT}{character/readonly}{Option string, upper case}
\argitem{IERR}{integer/modify}{Error code. Error if nonzero}
\end{argdeflist}
\begin{describe}

This routine is used inside AST routines to decode the name of the
cache that has triggered the interrupt. The numeric identifier NUM is
available as the single parameter passed to the AST routine. This
routine converts NUM to the name string. The string is padded with
blanks.

\end{describe}
\begin{options}
\opt{none}{}
\end{options}
\begin{returncodes}
\rcode{IC\_E\_OUTRNG}{cache number out of range. An invalid cache number
  has been passed.}
\rcode{IC\_E\_INACTIVE}{cache is inactive. System error.}
\end{returncodes}
\begin{examplecode}\begin{verbatim}
*
**        see example for IC_ASSOCIATE.
*
\end{verbatim}\end{examplecode}
\end{routine}


\begin{routine} %RRRRRRRRRRRRRRRRRRRRRRRRRRRRRRRRRRRRRRRRRRRRRRRRRRRRR
\subroutine
   {SUBROUTINE}{IC\_READ}{(CANAME, CADAT, CHOPT, IERR)}
\begin{overview}
An \info mode routine.

Used to read an information cache.
\end{overview}
\begin{argdeflist}
\argitem{CANAME}{character/readonly}{The name assigned to the cache}
\argitem{CADAT}{16 bytes/modify}{The cache contents returned.}
\argitem{CHOPT}{character/readonly}{Option string, upper case}
\argitem{IERR}{integer/modify}{Error code. Error if nonzero}
\end{argdeflist}
\begin{describe}

Returns the contents of the specified information cache in CADAT.

Can be used at any time by any process that has the cache booked.

The contents of the cache as written by the last process to use
\verb|IC_WRITE| will be returned in CADAT.

\end{describe}
\begin{options}
\opt{none}{}
\end{options}
\begin{returncodes}
\rcode{IC\_E\_EXREAD}{attempted read while in EX mode. System error.}
\rcode{IC\_E\_INVMODE}{cache is not the correct usage mode.}
\end{returncodes}
\end{routine}


\begin{routine} %RRRRRRRRRRRRRRRRRRRRRRRRRRRRRRRRRRRRRRRRRRRRRRRRRRRRR
\subroutine
   {SUBROUTINE}{IC\_RELEASE}{(CANAME, CADAT, CHOPT, IERR)}
\begin{overview}
A general routine. Releases the specified cache.
\end{overview}
\begin{argdeflist}
\argitem{CANAME}{character/readonly}{The name assigned to the cache}
\argitem{CADAT}{16 bytes/readonly}{The cache contents written, if
  the {\tt W} option is used}
\argitem{CHOPT}{character/readonly}{Option string, upper case}
\argitem{IERR}{integer/modify}{Error code. Error if nonzero}
\end{argdeflist}
\begin{describe}

This routine removes the specified cache from the active caches of
the current process. No calls referencing the specified cache will
work after calling this routine (except \verb|IC_BOOK| of course),
and no interrupts will be received from the released cache.

A lockholder (\watcher mode) can use this routine in order to release
the lock in an orderly manner when the program exits normally without
crashing; the {\tt W} option can be used to write to the cache a
code indicating to watchers that the lock was released as a result of a
normal exit and not a crash.

\end{describe}
\begin{options}
\opt{W}{write to the cache the contents of CADAT before releasing. Can
  be used only by masters in \mailbox mode and lockholders in \watcher
  mode.}
\end{options}
\begin{returncodes}
\rcode{none}{}
\end{returncodes}
\end{routine}


\begin{routine} %RRRRRRRRRRRRRRRRRRRRRRRRRRRRRRRRRRRRRRRRRRRRRRRRRRRRR
\subroutine
   {SUBROUTINE}{IC\_SEND}{(CANAME, CADAT, CHOPT, IERR)}
\begin{overview}
A routine used within \mailbox master mode AST routines to send a
reply to a signalling client.

\end{overview}
\begin{argdeflist}
\argitem{CANAME}{character/readonly}{The name assigned to the cache}
\argitem{CADAT}{16 bytes/readonly}{The reply to the client}
\argitem{CHOPT}{character/readonly}{Option string, upper case}
\argitem{IERR}{integer/modify}{Error code. Error if nonzero}
\end{argdeflist}
\begin{describe}

When a \mailbox master processes a client's signal inside an AST routine,
after receiving the message with \verb|IC_DELIVER|  a reply {\em must}
be sent using \verb|IC_SEND|. It is mandatory because this routine also
finishes the handshaking between master and client. The reply should
be loaded into CADAT.

\end{describe}
\begin{options}
\opt{none}{}
\end{options}
\begin{returncodes}
\rcode{IC\_E\_NONMASTER}{caller is not master of the cache}
\end{returncodes}
\begin{examplecode}\begin{verbatim}
*
**        see example for IC_ASSOCIATE.
*
\end{verbatim}\end{examplecode}
\end{routine}


\begin{routine} %RRRRRRRRRRRRRRRRRRRRRRRRRRRRRRRRRRRRRRRRRRRRRRRRRRRRR
\subroutine
   {SUBROUTINE}{IC\_SIGNAL}{(CANAME, CADAT, TIMEOUT, CHOPT, IERR)}
\begin{overview}
A \mailbox mode routine.

Called by a client to signal the master.

\end{overview}
\begin{argdeflist}
\argitem{CANAME}{character/readonly}{The name assigned to the cache}
\argitem{CADAT}{16 bytes/modify}{The message to be sent and (on return)
  the reply received}
\argitem{TIMEOUT}{real/readonly}{Timeout in seconds. {\tt IC\_SIGNAL} exits
  after this time if a reply has not yet been received from the master.
  If zero, there is no timeout (infinite wait).}
\argitem{CHOPT}{character/readonly}{Option string, upper case}
\argitem{IERR}{integer/modify}{Error code. Error if nonzero}
\end{argdeflist}
\begin{describe}

A \mailbox mode client calls \verb|IC_SIGNAL| to send a message to the master of
the specified cache and receive a reply. 

\verb|IC_SIGNAL| is called
with the message to be delivered to the master in CADAT. The master
receives an interrupt, reads the message from the client with a call
to \verb!IC_DELIVER!, and then sends a reply with \verb!IC_SEND!.
After the reply is received by the client, \verb|IC_SIGNAL| returns
with the reply in CADAT.

The TIMEOUT parameter if non-zero establishes the time after which
control is returned to the calling process even if the master has not yet
sent a reply.

\end{describe}
\begin{options}
\opt{A}{autobooking. Not yet implemented.}
\end{options}
\begin{returncodes}
\rcode{IC\_E\_INVMODE}{cache is not the correct usage mode.}
\rcode{IC\_E\_EXWRITE}{attempted write while in EX mode (attempt by master to
  signal itself)}
\rcode{IC\_E\_INVTIMOUT}{invalid timeout. System error.}
\end{returncodes}
\end{routine}


\begin{routine} %RRRRRRRRRRRRRRRRRRRRRRRRRRRRRRRRRRRRRRRRRRRRRRRRRRRRR
\subroutine
   {SUBROUTINE}{IC\_STATUS}{(CANAME, CADAT, CHOPT, IERR)}
\begin{overview}
A \watcher mode routine.

Used by lockholders to store status information.
\end{overview}
\begin{argdeflist}
\argitem{CANAME}{character/readonly}{The name assigned to the cache}
\argitem{CADAT}{16 bytes/readonly}{The status information}
\argitem{CHOPT}{character/readonly}{Option string, upper case}
\argitem{IERR}{integer/modify}{Error code. Error if nonzero}
\end{argdeflist}
\begin{describe}

This routine permits a lockholder to store status information in the
lock cache.

This information can be read by interested processes
via \verb|IC_INFO|, and is passed to watcher processes via
\verb|IC_WATCH| when the lockholder crashes or releases the lock.

Note that any process, not just watcher processes,
can by using \verb|IC_INFO| read the status of the lockholder.

The lockholder passes the status info in CADAT.

\end{describe}
\begin{options}
\opt{none}{}
\end{options}
\begin{returncodes}
\rcode{IC\_E\_NONLKHOL}{ calling process is not the lockholder}
\end{returncodes}
\end{routine}

\begin{routine} %RRRRRRRRRRRRRRRRRRRRRRRRRRRRRRRRRRRRRRRRRRRRRRRRRRRRR
\subroutine
   {SUBROUTINE}{IC\_TAKE\_LOCK}{(CANAME, CADAT, IDUM, TIMEOUT, CHOPT, IERR)}
\begin{overview}
Resource lock mode. Take out the specified resource lock.

\end{overview}
\begin{argdeflist}
\argitem{CANAME}{character/readonly}{The resource lock name}
\argitem{CADAT}{8 bytes/readonly}{User defined component of the lock's
    cache}
\argitem{IDUM}{integer/readonly}{Unused. For future use.}
\argitem{TIMEOUT}{real/readonly}{Timeout period, seconds}
\argitem{CHOPT}{character/readonly}{Option string, upper case}
\argitem{IERR}{integer/modify}{Error code. Error if nonzero}
\end{argdeflist}
\begin{describe}

This routine takes out the specified resource lock. If the lock is currently
held by another process, the routine waits for a period TIMEOUT seconds
before returning with a timeout error.

\end{describe}
\begin{options}
\opt{none}{}
\end{options}
\begin{returncodes}
\rcode{IC\_E\_INVMODE}{cache is not the correct usage mode.}
\rcode{IC\_E\_LKHELD}{lock is held by another process, and has not become
   available within the timeout interval.}
\end{returncodes}
\end{routine}

\begin{routine} %RRRRRRRRRRRRRRRRRRRRRRRRRRRRRRRRRRRRRRRRRRRRRRRRRRRRR
\subroutine
   {SUBROUTINE}{IC\_WAIT}{(CANAME, CADAT, TIMEOUT, CHOPT, IERR)}
\begin{overview}
A routine used by interrupt receiving processes in \broadcast,
\alarm, or \mailbox mode. It hibernates the process until an
interrupt is received.

\end{overview}
\begin{argdeflist}
\argitem{CANAME}{character/readonly}{The name assigned to the cache}
\argitem{CADAT}{16 bytes/modify}{The cache contents when the interrupt
  is delivered (the message from the interrupting process)}
\argitem{TIMEOUT}{real/readonly}{Timeout in seconds. {\tt IC\_WAIT} exits
  after this time if a message has not yet been received.
  If zero, there is no timeout (infinite wait).}
\argitem{CHOPT}{character/readonly}{Option string, upper case}
\argitem{IERR}{integer/modify}{Error code. Error if nonzero}
\end{argdeflist}
\begin{describe}

This routine can be used to suspend a process until an interrupt comes
in from the specified cache. The calling process might be waiting for
an alarm in \alarm mode, a broadcast in \broadcast mode, or a message
from a client in \mailbox mode.

When the interrupt is delivered and the process wakes, the contents
of the specified cache at the time of the interrupt (broadcast message,
alarm code, etc.) is returned in CADAT.

The routine \verb|IC_ASSOCIATE| is used more often than \verb|IC_WAIT|,
because \verb|IC_WAIT| can only handle interrupts from one cache, and
it prevents the process from doing anything else while it waits for
an interrupt.

\end{describe}
\begin{options}
\opt{none}{}
\end{options}
\begin{returncodes}
\rcode{IC\_E\_INVMODE}{cache is not the correct usage mode.}
\end{returncodes}
\end{routine}

\begin{routine} %RRRRRRRRRRRRRRRRRRRRRRRRRRRRRRRRRRRRRRRRRRRRRRRRRRRRR
\subroutine
   {SUBROUTINE}{IC\_WATCH}{(CANAME, CADAT, AST, CHOPT, IERR)}
\begin{overview}
A \watcher mode routine.

Establishes a watch on a \watcher cache, such that the caller is
notified when the lockholder process crashes.
\end{overview}
\begin{argdeflist}
\argitem{CANAME}{character/readonly}{The name assigned to the cache.
  Maximum length is 20 characters.}
\argitem{CADAT}{16 bytes/modify}{The cache contents on return, for
option 'W'}
\argitem{AST}{external routine}{AST routine. Used in conjunction with
option {\tt A}.}
\argitem{CHOPT}{character/readonly}{Option string, upper case}
\argitem{IERR}{integer/modify}{Error code. Error if nonzero}
\end{argdeflist}
\begin{describe}

This routine can be used by processes which have a \watcher mode cache
booked in {\tt P} (process watcher) mode. It enables the process to
take action when the cache lock is dropped by the lockholder process
(either because the lockholder has crashed or has released the
lock deliberately at normal termination). The action can either be to
hibernate and wake when the lock is dropped (option {\tt W}),
or establish an AST
routine that is called when the lock is dropped (option {\tt A})
so the watcher process
can do other things in the meantime and can watch several locks.

If the {\tt X} option has been selected, once the wake or AST
is delivered the watcher process becomes the new lockholder. If there are
several watcher processes, either all of them or none of them should use
the {\tt X} option.

Either the {\tt W} or {\tt A} option must be specified or an error will
be returned.

In {\tt A} mode, \verb|IC_DELIVER| can be called inside the AST routine
to read the status stored by the lockholder. In {\tt W} mode, the
status information is available in CADAT upon return.

\end{describe}
\begin{options}
\opt{W}{ Wait for lock release. Process hibernates until
lock is released. Cache at time of release is returned to caller in CADAT}
\opt{A}{ AST routine is established. AST is the name of the routine,
declared EXTERNAL. When lock is released, the AST routine is called.}
\opt{X}{ Exclusive notification. If several processes watch a lock in
X mode, only one of them will be activated when the lock is released.}
\end{options}
\begin{returncodes}
\rcode{IC\_E\_BADWOPT}{ missing an {\tt A} or {\tt W} option selection}
\rcode{IC\_E\_BADWATCH}{ cache mode is not valid for {\tt IC\_WATCH}}
\rcode{IC\_E\_DBLWATCH}{ a call to WATCH from the watched. Illegal.}
\rcode{IC\_E\_NOEXIST}{ cache does not exist (returned only if {\tt X}
  option was selected). No error message is printed.}
\end{returncodes}
\end{routine}

\begin{routine} %RRRRRRRRRRRRRRRRRRRRRRRRRRRRRRRRRRRRRRRRRRRRRRRRRRRRR
\subroutine
   {SUBROUTINE}{IC\_WRITE}{(CANAME, CADAT, CHOPT, IERR)}
\begin{overview}
An \info mode routine.

Writes an information cache.
\end{overview}
\begin{argdeflist}
\argitem{CANAME}{character/readonly}{The name assigned to the cache}
\argitem{CADAT}{16 bytes/readonly}{The cache contents written.}
\argitem{CHOPT}{character/readonly}{Option string, upper case}
\argitem{IERR}{integer/modify}{Error code. Error if nonzero}
\end{argdeflist}
\begin{describe}

Writes CADAT to the specified information cache.

Can be used at any time by any process that has the cache booked.

\end{describe}
\begin{options}
\opt{none}{}
\end{options}
\begin{returncodes}
\rcode{IC\_E\_INVMODE}{cache is not the correct usage mode.}
\end{returncodes}
\end{routine}
