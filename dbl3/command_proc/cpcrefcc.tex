\newpage
\part{CPC CLUSCOM Routines Reference}
\vspace*{1cm}
\begin{verbatim}
  List of CLUSCOM routines documented in this section:

  CC_SETUP      to initially set up the CLUSCOM section

  CC_SAVE       to save individual variables in a section
                + CC_SAVEV, CC_SAVEC

  CC_WRITELOCK  to write lock a section
                + CC_WRITELOCKV, CC_WRITELOCKC
  CC_RELEASE    to release a write lock
                + CC_RELEASEV, CC_RELEASEC

  CC_UPDATE     to update memory from disk
                + CC_UPDATEV, CC_UPDATEC

  CC_FLUSH      to flush memory to disk
                + CC_FLUSHV, CC_FLUSHC

  CC_SET_LAST   to control partial read/write of the section     NEW
\end{verbatim}

\begin{routine} %RRRRRRRRRRRRRRRRRRRRRRRRRRRRRRRRRRRRRRRRRRRRRRRRRRRRR
\subroutine
   {SUBROUTINE}{CC\_FLUSH}{(COMID, CHOPT, IERR)}
\begin{overview}
Flushes the common in memory to disk.
\end{overview}
\begin{argdeflist}
\argitem{COMID}{character/readonly}{The name of the CLUSCOM section}
\argitem{CHOPT}{character/readonly}{Option string, upper case}
\argitem{IERR}{integer/modify}{Error code. Error if nonzero}
\end{argdeflist}
\begin{describe}

This routine flushes memory to disk. If a write lock is in place, it remains in
place. If there is no write lock on the common on entry, the
common is write locked during the flush and then unlocked
again. This keeps to a minimum the time during which the common is write
locked and inaccessible to other processes.

This routine is only safe to use when there is only one process which writes
to the common. See the discussion on \verb|CC_FLUSH| in the technical section.

Related routines are \verb|SUBROUTINE CC_FLUSHV(VARNAME,CHOPT,IERR)|
and \verb|SUBROUTINE CC_FLUSHC(STRING,CHOPT,IERR)| for specifying
variable or string names instead of the section name.

\end{describe}
\begin{options}
\opt{none}{}
\end{options}
\begin{returncodes}
\rcode{no error return codes originate from this routine}{}
\end{returncodes}
\begin{examplecode}\begin{verbatim}
*
        COMMON /CCTEST/ IFIRST, <variables>, ILAST(128)
        CHARACTER STRING*10
* ...
        CALL CC_SETUP('CCTEST','CCTEST.MAPCOM',IFIRST,ILAST,' ',IERR)
        IF (IERR.NE.0) GOTO 999
* ...
        DO WHILE (.TRUE.)
          <assignment statements which load the common>
          CALL CC_FLUSH('CCTEST',' ',IERR)
          IF (IERR.NE.0) GOTO 999
          CALL LIB$WAIT(UPDATE_INTERVAL)
        ENDDO
\end{verbatim}\end{examplecode}
\end{routine}


\begin{routine} %RRRRRRRRRRRRRRRRRRRRRRRRRRRRRRRRRRRRRRRRRRRRRRRRRRRRR
\subroutine
   {SUBROUTINE}{CC\_RELEASE}{(COMID, CHOPT, IERR)}
\begin{overview}
Releases a write locked section.
\end{overview}
\begin{argdeflist}
\argitem{COMID}{character/readonly}{The name of the CLUSCOM section}
\argitem{CHOPT}{character/readonly}{Option string, upper case}
\argitem{IERR}{integer/modify}{Error code. Error if nonzero}
\end{argdeflist}
\begin{describe}
This routine releases the write lock on a section, after flushing
memory to disk to save any changes. The revision number of the
section is incremented.

If the section is not write locked when this routine is called, it
does nothing.

This routine should always be used after \verb!CC_WRITELOCK!.

Related routines are \verb|SUBROUTINE CC_RELEASEV(VARNAME,CHOPT,IERR)|
and \verb|SUBROUTINE CC_RELEASEC(STRING,CHOPT,IERR)| for specifying
variable or string names instead of the section name.

\end{describe}
\begin{options}
\opt{none}{}
\end{options}
\begin{returncodes}
\rcode{no error return codes originate from this routine}{}
\end{returncodes}
\begin{examplecode}\begin{verbatim}
*
*   see the example for CC_WRITELOCK
*
\end{verbatim}\end{examplecode}
\end{routine}


\begin{routine} %RRRRRRRRRRRRRRRRRRRRRRRRRRRRRRRRRRRRRRRRRRRRRRRRRRRRR
\subroutine
   {SUBROUTINE}{CC\_SAVE}{(VARNAM, VALUE, CHOPT, IERR)}
\begin{overview}
A new value VALUE for a variable VARNAM in a mapped common is written
to the section on disk
\end{overview}
\begin{argdeflist}
\argitem{VARNAM}{non-char/readonly}{Variable to be updated}
\argitem{VALUE}{non-char/readonly}{New value of variable}
\argitem{CHOPT}{character/readonly}{Option string, upper case}
\argitem{IERR}{integer/modify}{Error code. Error if nonzero}
\end{argdeflist}
\begin{describe}
This routine finds which mapped common VARNAM is contained in, updates
the common from disk if necessary, writes the new value VALUE into VARNAM and
flushes the common to disk. The revision number is incremented.

Related routines are \verb|SUBROUTINE CC_SAVEV(ANAME,AVAL,LNG,CHOPT,IERR)|
which saves new values AVAL in a vector (array) ANAME of length LNG,
and \verb|SUBROUTINE CC_SAVEC(STRNAM,STRVAL,CHOPT,IERR)| to save a new
value STRVAL in a string STRNAM.

If several variables in a given section have to be updated at the same
time, it is more efficient to use \verb|CC_WRITELOCK| and \verb|CC_RELEASE|.

\end{describe}
\begin{options}
\opt{F}{ Force the update from disk to take place, even if the
revision number is unchanged.}
\end{options}
\begin{returncodes}
\rcode{no error return codes originate from this routine}{}
\end{returncodes}
\begin{examplecode}\begin{verbatim}
*
        COMMON /CCTEST/ IFIRST, X, IARRAY(10), STRING, ILAST(128)
        CHARACTER STRING*10
* ...
        CALL CC_SETUP('CCTEST','CCTEST.MAPCOM',IFIRST,ILAST,' ',IERR)
        IF (IERR.NE.0) GOTO 999
* ...
        CALL CC_SAVE(X,7.,' ',IERR)
        IF (IERR.NE.0) GOTO 999
        CALL CC_SAVEC(STRING,'NEWSTRING',' ',IERR)
        IF (IERR.NE.0) GOTO 999
\end{verbatim}\end{examplecode}
\end{routine}

\begin{routine} %RRRRRRRRRRRRRRRRRRRRRRRRRRRRRRRRRRRRRRRRRRRRRRRRRRRRR
\subroutine
   {SUBROUTINE}{CC\_SETUP}{(COMID, FNAME, IFIRST, ILAST, CHOPT, IERR)}
\begin{overview}
This procedure is called to set up the mapping of a
CLUSCOM section. It creates the
mapping between a common block in memory and a disk file.
\end{overview}
\begin{argdeflist}
\argitem{COMID}{character/readonly}{The name assigned to the CLUSCOM
section by the user. Maximum length is 20 characters.}
\argitem{FNAME}{character/readonly}{The filename to contain the mapped
section. The filename convention is {\tt <comid>.MAPCOM} }
\argitem{IFIRST}{integer/readonly}{The first word of the common block
to be mapped}
\argitem{ILAST}{integer/readonly}{The last word of the common block to
be mapped}
\argitem{CHOPT}{character/readonly}{Option string, upper case}
\argitem{IERR}{integer/modify}{Error code. Error if nonzero}
\end{argdeflist}
\begin{describe}
This routine sets up the mapping between a common block in memory and
a disk file (referred to as a CLUSCOM section).
The specified file (referred to as a section file)
is opened and mapped to the memory
region IFIRST to ILAST. If the file already exists, it is opened and
mapped; if it does not exist it is created. Section files are always
created with group and world write access, so that privileges are
not needed to map them.

There are two modes of access -- using QIOs (block I/O in earlier
CLUSCOM versions) or global sections.
These two modes are discussed in the technical section.
They are functionally equivalent from the user's point of view (but see
the technical section).
The default mode is QIOs. Global section mode is usually faster,
but {\em the SYSGBL privilege is required to use the global section
 mode}. No special privileges are required to use QIO mode.

Note that the ILAST location should mark the beginning of 512 bytes
of unused space. This is most easily achieved by making ILAST a 128 element
array. See the discussion in the technical part.

\end{describe}
\begin{options}
\opt{G}{ Use global section (rather than the default QIO) mode.}
\opt{R}{ File is opened READONLY. The file is opened with only read
access required. Attempts to write to the file are refused.}
\opt{W}{ File is opened with a write lock. This option should be used
carefully. No other process can gain read or write access to the file until the
file is released by the write lock holder.}
\opt{C}{ Force creation of a new map file, even if a version exists.}
\opt{O}{ Open old map file. If no file exists, an error is returned.}
\opt{M}{ by default, when an existing map file is opened,
memory is initialized from disk rather than vice
versa. To initialize disk from memory, use this
option. NOTE that in global section mode, this option has
the effect of {\em zeroing} both memory and disk.}
\opt{Y}{ system mode. Cache is recognised across all UICs. See the
  technical section. }
\end{options}
\begin{returncodes}
\rcode{CC\_E\_BADNAME}{ identifier string too long}
\rcode{CC\_E\_BADFILE}{ filename too long}
\rcode{CC\_E\_MAXEX}{ maximum number of mapped commons exceeded}
\rcode{CC\_E\_NOFILE}{ specified section file does not exist}
\end{returncodes}
\begin{examplecode}\begin{verbatim}
*
**        the common to map
        COMMON /CCTEST/ IFIRST, IARRAY(10), STRING, ILAST(128)
        CHARACTER STRING*10
*
        CALL CC_SETUP('CCTEST','CCTEST.MAPCOM',IFIRST,ILAST,' ',IERR)
        IF (IERR.NE.0) GOTO 999
* ...
\end{verbatim}\end{examplecode}
\end{routine}


\begin{routine} %RRRRRRRRRRRRRRRRRRRRRRRRRRRRRRRRRRRRRRRRRRRRRRRRRRRRR
\subroutine
   {SUBROUTINE}{CC\_SET\_LAST}{(COMID, ILAST, CHOPT, IERR)}
\begin{overview}
This procedure tells CLUSCOM to read/write only part of the section.

\end{overview}
\begin{argdeflist}
\argitem{COMID}{character/readonly}{The name assigned to the CLUSCOM
section by the user. Maximum length is 20 characters.}
\argitem{ILAST}{integer/readonly}{The last word of the common block to
read/write}
\argitem{CHOPT}{character/readonly}{Option string, upper case}
\argitem{IERR}{integer/modify}{Error code. Error if nonzero}
\end{argdeflist}
\begin{describe}
This procedure sets the final word of a mapped common to 
be written to or read from the section file. It tells CLUSCOM to
to write/read only part of a common, to save time when only a small part
of a large common, the region up to ILAST, is used.

To reset reading/writing to use the entire section, call this routine with
the {\tt R} option.

Note that no I/O is actually performed by this routine.
\end{describe}
\begin{options}
\opt{R}{Reset to read/write the entire common. ILAST ignored.}
\end{options}
\begin{returncodes}
\rcode{CC\_E\_MISMATCH}{ name/variable mismatch. The ILAST location is not
   in the indicated section.}
\end{returncodes}
\begin{examplecode}\begin{verbatim}
*
*       eg. if the mapped common is
   COMMON /MY/ IFIRST, IARRAY(10000), ILAST(128)
* ...
*       then
   CALL CC_SET_LAST(COMID,IARRAY(100),' ',IERR)
*       will cause /MY/ to be read/written only up to the first page
*       boundary past IARRAY(100).
\end{verbatim}\end{examplecode}
\end{routine}


\begin{routine} %RRRRRRRRRRRRRRRRRRRRRRRRRRRRRRRRRRRRRRRRRRRRRRRRRRRRR
\subroutine
   {SUBROUTINE}{CC\_UPDATE}{(COMID, CHOPT, IERR)}
\begin{overview}
Updates the common block in memory from the disk file.
\end{overview}
\begin{argdeflist}
\argitem{COMID}{character/readonly}{The name of the CLUSCOM section}
\argitem{CHOPT}{character/readonly}{Option string, upper case}
\argitem{IERR}{integer/modify}{Error code. Error if nonzero}
\end{argdeflist}
\begin{describe}
This routine checks the revision number
\footnote{see the discussion on revision numbers in the technical section}
of the section and
updates from disk if and only if changes have been made by other
processes since the last update. Checking the revision number is very
fast, so that it is not inefficient to use \verb|CC_UPDATE| frequently,
for example in a loop which refreshes a screen display every few
seconds based on information in a CLUSCOM section.

Related routines are \verb|SUBROUTINE CC_UPDATEV(VARNAME,CHOPT,IERR)|
and \verb|SUBROUTINE CC_UPDATEC(STRING,CHOPT,IERR)| to update a
variable VARNAME or character string STRING respectively, where
VARNAME or STRING are contained in mapped commons. Note that the
whole common is updated, not just the specified variable; these
routines just provide a means of identifying a mapped common by a variable
contained within it rather than by its name.

\end{describe}
\begin{options}
\opt{F}{ Force the update from disk to take place, even if the
revision number is unchanged.}
\end{options}
\begin{returncodes}
\rcode{no error return codes originate from this routine}{}
\end{returncodes}
\begin{examplecode}\begin{verbatim}
*
        COMMON /CCTEST/ IFIRST, IARRAY(10), STRING, ILAST(128)
        CHARACTER STRING*10
* ...
        CALL CC_SETUP('CCTEST','CCTEST.MAPCOM',IFIRST,ILAST,' ',IERR)
        IF (IERR.NE.0) GOTO 999
* ...
        CALL CC_UPDATE('CCTEST',' ',IERR)
        IF (IERR.NE.0) GOTO 999
        CALL CC_UPDATEV(IARRAY(3),' ',IERR)
        IF (IERR.NE.0) GOTO 999
        CALL CC_UPDATEC(STRING,' ',IERR)
        IF (IERR.NE.0) GOTO 999
\end{verbatim}\end{examplecode}
\end{routine}

\begin{routine} %RRRRRRRRRRRRRRRRRRRRRRRRRRRRRRRRRRRRRRRRRRRRRRRRRRRRR
\subroutine
   {SUBROUTINE}{CC\_WRITELOCK}{(COMID, CHOPT, IERR)}
\begin{overview}
Locks a section in write mode.
\end{overview}
\begin{argdeflist}
\argitem{COMID}{character/readonly}{The name of the CLUSCOM section}
\argitem{CHOPT}{character/readonly}{Option string, upper case}
\argitem{IERR}{integer/modify}{Error code. Error if nonzero}
\end{argdeflist}
\begin{describe}
This routine updates a section from disk (if necessary) and locks it
in write mode, so that {\em no other process has any access to the
section -- read or write --
until the write lock is released with \verb|CC_RELEASE|}.
It should always be used in conjunction with \verb|CC_RELEASE|
as in the example below.

It updates from disk if and only if changes have been made by other
processes since the last update.

Related routines are \verb|SUBROUTINE CC_WRITELOCKV(VARNAME,CHOPT,IERR)|
and \verb|SUBROUTINE CC_WRITELOCKC(STRING,CHOPT,IERR)| for specifying
variable or string names instead of the section name.

\end{describe}
\begin{options}
\opt{F}{ Force the update from disk to take place, even if the
revision number is unchanged.}
\end{options}
\begin{returncodes}
\rcode{CC\_E\_NOWRTLOCK}{ attempt to write lock a readonly section}
\end{returncodes}
\begin{examplecode}\begin{verbatim}
*
        COMMON /CCTEST/ IFIRST, X, IARRAY(10), STRING, ILAST(128)
        CHARACTER STRING*10
* ...
        CALL CC_SETUP('CCTEST','CCTEST.MAPCOM',IFIRST,ILAST,' ',IERR)
        IF (IERR.NE.0) GOTO 999
* ...
        CALL CC_WRITELOCK('CCTEST',' ',IERR)
        IF (IERR.NE.0) GOTO 999
        X=X+2.
        STRING='NEWSTRING'
        CALL CC_RELEASE('CCTEST',' ',IERR)
        IF (IERR.NE.0) GOTO 999
\end{verbatim}\end{examplecode}
\end{routine}
