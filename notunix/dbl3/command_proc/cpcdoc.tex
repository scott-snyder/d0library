\input{[abachi.tex]arteuro}             %article style with A4 format
\input{[abachi.tex]verb}
%\newcommand{\showcode}[1]{\type{#1}}
\def\showcode#1{\type{#1}}
\newenvironment{code}{\begin{verbatim}}{\end{verbatim}}
% see also tabbing, p. 62   and verbatim, p 65
% loading fonts, p. 200

\newcommand{\eg}{{\em eg. }}
\newcommand{\ie}{{\em ie. }}
\newcommand{\mailbox}{ {\bf mailbox }}
\newcommand{\info}{{\bf infocache }}
\newcommand{\alarm}{{\bf alarm }}
\newcommand{\broadcast}{{\bf broadcast }}
\newcommand{\watcher}{{\bf process watcher }}
\newcommand{\reslock}{{\bf resource lock }}


%\pagestyle{empty}          %no page numbering
\begin{document}           % Beginning of text
\large                     % standard \large and 11 points

\begin{center}
{\LARGE The L3 Cluster Process Communication Facility (CPC)}\\
\vspace*{.7cm}
Torre Wenaus \ \ MIT\\ 
{\tt L3ONCL::WENAUS \ \ WENAUS@CERNVM.BITNET}
\\ \vspace*{.7cm}
\today
\\ \vspace*{.7cm}
Version 1.3
\end{center}
     
%=========================================
\begin{abstract}
\normalsize \em
This document describes the Cluster Process Communication (CPC)
package of routines written to
provide interprocess communication facilities for the online data acquisition
VAXcluster of the L3 experiment.

CPC currently consists of two components, CLUSCOM and INFOCACHE.
They provide, respectively, interprocess information sharing across
the cluster (FORTRAN common blocks shared cluster-wide) and
fast interprocess signalling (mailbox-like message handling with a
reply to the sender, broadcasts, process status monitoring and
crash signalling, alarm distribution, resource locking).

The first part introduces CPC and provides an
overview of what CPC provides and how it can be used. The second
part provides a more technical description of how the
components of CPC are implemented. Subsequent parts document in detail
all CPC routines callable from application programs.

Please tell me of problems or curious behaviour.
\end{abstract}

\newpage
\tableofcontents

\newpage

%=========================================
\part{Overview of CPC}

The L3 Cluster Process Communication package (CPC) enables fast interprocess
information exchange and signalling within a VAXcluster.

CPC was designed for the L3 online data acquisition system. The L3 online
computer system consists
of a single large mixed VAXcluster of about 30 nodes.
CPC aims to provide the communication
tools needed to handle interaction and
information exchange between the  processes of the L3 DAQ
which will be distributed across all nodes of the cluster. 
Some of the requirements are listed below.

\begin{itemize}

\item{ \bf Cluster-wide databases: }
Processes must be able to concurrently access
information pools (databases) from any node in the cluster,
with maximum efficiency
and, for the programmer, in as transparent a way as possible.

\item{\bf Interprocess signalling: }
Processes must be able to define signals (with associated symbolic names)
which are
known across the cluster so that messages can be exchanged between
processes on arbitrary nodes in the cluster (\eg for alarm handling,
command delivery, interprocess synchronisation). Signalling must be
fast to accommodate potentially high signalling rates. A handshake mode
in which an acknowledge is received by the sender should be available
for maximum safety.

\item{\bf Speed and efficiency: }
All communication tools should be as fast and efficient as possible,
in order to make full use of available resources.
The single-cluster environment makes possible the use of highly efficient
VMS tools which work only within a cluster, rather than more common
(but slower) tools based on DECnet.

\item{\bf Node independence: }
All communication tools should be node independent. That is, no
communication package should ever need to know what node a process is
running on. 
This makes possible a flexible and, if needed, a dynamically changing
configuration, and simplifies failover when a node goes down.


\end{itemize}

CPC has been designed with these requirements in mind.
CPC currently consists of two components, CLUSCOM and INFOCACHE.
They provide, respectively, interprocess information sharing across
the cluster (FORTRAN common blocks shared cluster-wide) and
fast interprocess signalling (mailbox-like message handling with a
reply to the sender, broadcasts, process status monitoring and
crash signalling, alarm distribution).

%=========================================
\section{Changes, New features}

Changes and additions are described in more detail in the appropriate
sections.

\begin{itemize}

\item{ \bf Infocache: new 'resource lock' usage mode}

\item{ \bf Infocache: {\tt IC\_INFO} cache list output improved}

\item{ \bf Cluscom: new, faster QIO-based I/O from Mike Capell. }
In the 1.3 release all I/O
is done using QIOs. The next release will include an improved, faster
global section mode.

\item{ \bf Cluscom: {\tt CC\_SET\_LAST} routine for partial section updating}

\item{ \bf Timeouts: built-in default timeouts increased}
In both CLUSCOM and INFOCACHE, timeouts were increased to accommodate the
slowness of VAXstations. Also, the timeout error message was changed to
make it more comprehensible, and the timeout error was changed from
ERROR level to WARNING level. Some bugs were fixed.

\item{ \bf AST level execution: protections}
Protections were incorporated against the CPC lock routines being entered
at AST level while executing at normal level. Now separate routines
are used for the two levels, preventing corruption of
parameters at normal level by the AST execution.

\end{itemize}

%=========================================
\section{CLUSCOM}

\subsection{CLUSCOM overview}
CLUSCOM is a set of routines which implement cluster-wide databases
in shared FORTRAN common blocks (sometimes referred to as CLUSCOM
sections or mapped commons).
Processes anywhere in the cluster
can obtain read/write access to a CLUSCOM section, which resides on
a disk file, through which a common block
within the program is shared with all other processes also mapped to
that section. Routines provide the interlocking necessary to ensure
data integrity in read/write operations.

Cluscom has two components, the data sharing mechanism and the interlock
mechanism. Two modes of data sharing are available: one based on 
sections, and one based on QIOs, added by Mike Capell
(RMS block I/O in earlier CLUSCOM versions). The two modes are
functionally equivalent; they differ in their speed and in when disk updates
take place (which can be very important, as discussed below).
More information on the two modes is in the technical section.

There always exist two representations of a CLUSCOM section: one on the
disk file, and one in memory. Routines are provided to keep both these
representations up to date, by flushing memory to disk after making
changes and updating memory from disk to obtain the changes of
other processes. Routines also coordinate access to the disk file so that
when one process is writing to disk, all other processes are locked out.
The interlocking is achieved using the Distributed Lock Manager. More
details are given in the technical section.

\subsection{Using CLUSCOM}

To create a shared database with CLUSCOM, the data to be shared must
first be gathered into a FORTRAN common block. Several CLUSCOM sections
can be created within a single program; each must be associated with
a separate common block. 

 The first and last words of the common are
used to delineate the beginning and end of the shared memory region,
as follows:

\begin{verbatim}
   COMMON /MYCOM/ IFIRST, ... ,MY_VARIABLES, ... ,ILAST(128)
CDEC$ PSECT /MYCOM/ ALIGN=9
\end{verbatim}

Note that the final word ILAST is actually an array of length 128
(512 bytes, or one VMS page). This array must be present and must
not be used for data storage. Its purpose is explained in the
technical section.

The \verb|CDEC$| declaration page aligns the common. This is essential.

The CLUSCOM section is created (or an existing section created by
another process is mapped to) by a call to \verb|CC_SETUP|:

\begin{verbatim}
   CALL CC_SETUP(COMID,FNAME,IFIRST,ILAST,CHOPT,IERR)
\end{verbatim}

\noindent
where COMID is the name associated with the section (often, but not
necessarily, the name of the common block), FNAME is the filename
of the disk file which contains the shared data, IFIRST and
ILAST are the first and last variables of the common (they should
be INTEGER or REAL), and CHOPT specifies options (\eg the data sharing
mechanism is selected via an option). Any error condition
is returned in IERR. If no errors occurred, IERR=0 upon return.
For more detailed information on \verb|CC_SETUP| see the reference section.

Note that \verb|CC_SETUP| options and options on all other CPC routines
{\em must be in upper case}.

Once the call to \verb|CC_SETUP| has been made, routines to update memory from
disk, flush memory to disk, write lock and unlock the common, and
modify individual variables can be called. These calls are summarised
below and are discussed in more detail in the reference section.

\begin{verbatim}
   CALL CC_SAVE(VARNAME,VALUE,CHOPT,IERR)
\end{verbatim}
   to save a new value VALUE of the variable VARNAME, where
   VARNAME is a REAL*4 or INTEGER*4 (non-array) value in any
   common currently mapped. Similar routines \verb|CC_SAVEV|,
   \verb|CC_SAVEC| save arrays and character strings respectively.

\begin{verbatim}
   CALL CC_WRITELOCK(COMID,CHOPT,IERR)
\end{verbatim}
   to lock the common COMID in WRITE mode. If you are changing
   more than one variable in the common it is more efficient
   to use this routine than \verb|CC_SAVE| calls.
   Following this call no other process will have any access to
   the common until \verb|CC_RELEASE| is called. Changes made to the
   common will not take effect until \verb|CC_RELEASE|
   or \verb|CC_FLUSH| is
   called, and will be lost if \verb|CC_UPDATE|
   or \verb|CC_SAVE| is called
   before \verb|CC_RELEASE| or \verb|CC_FLUSH|.
   Similar routines \verb|CC_WRITELOCKC| (for character strings),
   \verb|CC_WRITELOCKV| (for all non-string variables) permit one to
   specify a variable or character string within the common
   rather than the common identifier.


\begin{verbatim}
   CALL CC_RELEASE(COMID,CHOPT,IERR)
\end{verbatim}
   to release the write lock on common COMID. NO OTHER PROCESS
   CAN READ FROM OR WRITE TO THE COMMON WHILE YOU HAVE IT IN
   WRITE LOCK. So always use \verb|IC_WRITELOCK/IC_RELEASE| as follows, to
   minimise the time the common is held with a write lock:
\begin{verbatim}
   !   obtain a write lock
   CALL CC_WRITELOCK(...)
   !   set new values with normal assignment statements
   VARIABLE = NEWVAL
   !   immediately update the disk and release the write lock
   CALL CC_RELEASE(...)
\end{verbatim}
   Routines \verb|CC_RELEASEV|, \verb|CC_RELEASEC| also available.

\begin{verbatim}
   CALL CC_UPDATE(COMID,CHOPT,IERR)
\end{verbatim}
   to update the common from disk (ie get any modifications made
   by other processes since the last update). A read
   from disk is done only if the common has been changed by
   another process since the last update, so this is an
   inexpensive call.
   Routines \verb|CC_UPDATEV|, \verb|CC_UPDATEC| also available.

\begin{verbatim}
   CALL CC_FLUSH(COMID,CHOPT,IERR)
\end{verbatim}
   to flush memory to disk. Can only be used if there is a
   {\em single\/} process which writes to the common.
   This routine must be used with care
   if global section mode is used. See the technical section.
   Routines \verb|CC_FLUSHV|, \verb|CC_FLUSHC| also available.


When you change the size of a common block that is mapped by CLUSCOM,
you must delete the old section file or use the
{\tt C} option of \verb|CC_SETUP| to ensure a new copy is created.

%=========================================
\section{INFOCACHE}

\subsection{INFOCACHE overview}
The INFOCACHE package provides communication facilities complementary
to those of CLUSCOM: fast data sharing and various types of
interprocess signalling.

INFOCACHE is based on the {\sl cache\/},
an entity associated with a symbolic
name and accessible across the cluster, which can store information and
act as a signalling mechanism.
This basic entity is used in INFOCACHE to implement several
distinct utilities.
The different facilities, or usage modes, are outlined below:

\begin{itemize}
\item{\bf Information exchange: }
Referred to as \info mode.
A cache can be used to make a small (16 byte) packet of information
available read/write to processes clusterwide. This is similar to a
small CLUSCOM section, but has a much faster access time.
This is especially
appropriate for frequently changing information (\eg a current event number)
or information which must be checked frequently.
No interprocess signalling is involved.

\item{\bf Broadcast messages: }
    Referred to as \broadcast mode.
    Within a group of processes each of which has a given broadcast
    cache booked, any process can broadcast a message to
    all other processes in the group via the cache, without
    needing to know what other processes are in the group, where they
    are running, etc.

\item{\bf Master/client interprocess signalling: }
    Referred to as \mailbox mode.
    A `master' process receives signals from any number of `client'
    processes and sends replies.
    The cache is similar to a mailbox for the master process.
    Each signal transaction, initiated
    by a single subroutine call by the client, proceeds as follows:
     \begin{itemize}
     \item
       client sends a message to the master
     \item
       master receives the message via an interrupt,
       interprets it and acts on it, and
       then sends a message back to the client
     \item
       control is returned to the client process when the master's reply
       is received or a timeout expires.
     \end{itemize}
    The master's reply to the client sender provides acknowledgement
    handshaking that makes this a secure form of communication.
       
\item{\bf Process status watching and crash alarm: }
    Referred to as \watcher mode.
    A fast and reliable way of making available cluster-wide status
    information for a process. If the watched
    process crashes, notification is
    available cluster-wide, and a mechanism exists whereby a new copy of
    the program can be started automatically, possibly on another node.
    This crash alarm works even
    if the process crash was due to the crash of the node; no local server
    (which would of course also crash if the node crashed)
    is used to disseminate the alarm
    to processes on other nodes in the cluster.

\item{\bf Alarm distribution system: }
    Referred to as \alarm mode.
    A flexible and simple way of distributing alarms from any number of
    alarm-generating tasks (monitors, producers, etc.) to a dynamically
    variable assortment of alarm-receiving tasks (alarm loggers,
    display programs, etc.)

    An alarm generating program routes its alarms through one or more
    alarm channels identified by symbolic names (\eg a high voltage
    monitor might create an alarm channel 'HV' for its alarms), where each
    alarm channel corresponds to a cache.
    Any other process may declare that it wants to receive the alarms 
    of one or more alarm channels (\eg a status display program books
    the `HV' channel and displays HV alarms). These `alarm server'
    processes receive an interrupt and a message when an alarm
    generating task logs an alarm. 
    
\item{\bf Resource locks: }
    Referred to as \reslock mode.
    A means of controlling access to 'resources' defined by the user. A
    resource might be a subdetector Fastbus system, a communication channel,
    anything which should only be accessed by one user at a time. A
    processing trying to take out a resource lock will succeed only if the
    lock is not held by another process on the cluster. If the lock is
    held, the identity of the holder is available. To use a resource, the
    resource lock is taken out, the resource is used, and the resource lock
    is released when the resource is no longer needed.


\end{itemize}

The different INFOCACHE facilities are implemented as usage modes,
with each cache associated with a particular usage mode at booking time via
an option selection.

INFOCACHE utilises the Distributed Lock Manager. A cache is a lock, and
the information repository is the lock value block. More details are given
in the technical section.

\subsection{Using INFOCACHE}

A guide to how INFOCACHE is used in its various usage modes is given
here. For more information on the individual routines, their usage and
their calling parameters, see the reference section.

To use a cache, a program first books it with a call to \verb|IC_BOOK|.
The cache may already exist through a previous booking by another
process or may be created by the booking call; the user sees no
difference. The booking call is the same for all usage modes:

\begin{verbatim}
   CALL IC_BOOK(NAME,CACHE,CHOPT,IERR)
\end{verbatim}
where NAME is the name the user assigns to the cache, CACHE is
16 bytes of storage (can be I*4 CACHE(4), I*2 CACHE(8), REAL CACHE(4),
etc.) for the cache, CHOPT is an options string, and IERR is
a returned error code. The usage mode is determined by the option
string.

Note that \verb|IC_BOOK| options and options on all other CPC routines
{\em must be in upper case}.

After the \verb|IC_BOOK| call, the available calls depend on the usage
mode. The routines available for each mode are summarised below.

\subsubsection{Infocache mode}
   This is the default usage mode. The 16 byte data store of the cache
   can be read (\verb|IC_READ|) or written (\verb|IC_WRITE|)
   at any time by any process that has it booked.
   No signalling is involved.
   Reads/writes are very fast (.5 millisec up, depending on CPU speed
   and network access).

\subsubsection{Broadcast mode}
  Any process that has a broadcast cache booked can
  \begin{itemize}
     \item broadcast a message to all other processes that have
       the cache booked by calling \verb|IC_BROADCAST|
     \item wait until another process has sent out a broadcast, by calling
       \verb|IC_WAIT| (with a timeout if desired), and then read the message
     \item associate an AST-routine with the cache by calling
       \verb|IC_ASSOCIATE|;
       the AST routine is called when a broadcast is received (sent by
       another process).
       Inside the AST routine, \verb|IC_DELIVER| {\em must\/}
       be called to receive the broadcast message.
   \end{itemize}
   A process that has booked a broadcast cache {\em must\/} call either
   \verb|IC_ASSOCIATE| or \verb|IC_WAIT| immediately after the booking call.
   A process which fails to do so will be unable to receive broadcasts.

\subsubsection{Process watcher mode}
   For program crash detection and status monitoring.
   The process to be watched books a cache in `lock' mode when it starts up.
   The process can maintain in the cache information on its status by
   calling \verb|IC_STATUS|.

   Other processes by booking the cache in `watcher' mode and calling
   \verb|IC_WATCH| can detect if the lock-holder crashes or
   drops the lock.
   Watcher processes can call \verb|IC_WATCH| to
     \begin{itemize}
     \item
       hibernate until the lock is released, \ie until the process holding
       the lock crashes. For example, a backup copy of a critical program
       running on node $A$
       could be started on node $B$. The node $B$ copy would watch a
       lock taken out
       by the primary copy running on node $A$; if the node $A$ copy ever
       crashed, the lock would be released and the copy on node $B$
       could take over instantly. If the primary copy exits normally, it
       can inform the backup via a flag in the cache, which the backup
       program can check before taking over.
     \item
       establish an AST routine which is called if the lock is released.
       This way one process can watch the locks of any number of other
       processes and take action if any program crashes.
     \end{itemize}

     The lock-holder process can store status information in the
     cache associated with its lock by calling \verb|IC_STATUS|.
     Any other process (including but not limited to `watcher' processes)
     can read this status information
     from the cache by a call to \verb|IC_INFO|.

A lockholder can use the \verb|IC_RELEASE| routine in order to release
the lock in an orderly manner when the program exits normally without
crashing; the {\tt W} option can be used to write to the cache a
code indicating to watchers that the lock was released as a result of a
normal exit and not a crash.

\subsubsection{Mailbox mode}
  One process books a
  cache in `master' mode and receives signals from and sends replies
  to any number of
  other processes which book the same cache in `client' mode.

  The master process services client signals in an AST routine
  defined by a call to \verb|IC_ASSOCIATE|, which should be called
  immediately after booking the cache.
  The AST routine must call \verb|IC_DELIVER|
  to receive the message from the client, followed
  (after formulating a reply) by a call to
  \verb|IC_SEND| to send the response.

  The client calls \verb|IC_SIGNAL| to initiate the communication. This
  triggers the master's AST routine. The AST routine
  receives the message the client has stored in the cache, interprets
  it and formulates a reply, and sends the reply to the client.
  The client process returns from \verb|IC_SIGNAL| when the reply is
  received or when a timeout expires (optional).

  The cache in this case acts much like a mailbox for the master
  process. It has several advantages with respect to VMS mailboxes:
  it works cluster-wide, it is faster, and it permits the master (mailbox
  holder) to send an acknowledgement/reply containing information
  to the sender as part of the message transaction. Its limitation
  is the 16-byte maximum on the information sent as a message and
  returned as a reply. This can be overcome by using a CLUSCOM section
  to pass additional information.

\subsubsection{Alarm mode}
Alarms are divided into channels, with each channel corresponding to a
cache. One channel can be used for a whole class of alarms, \eg high
voltage alarms. The cache contents can contain a code for the particular
alarm being sent plus supplementary information (such as an identification
of the alarm sender). The VMS MESSAGE utility provides a convenient means of
defining alarms which can be identified by a 4 byte word. This leaves
12 bytes of the 16-byte cache remaining for supplementary information.

A task wishing to send alarms through a channel MYALARMS calls
\verb|IC_BOOK| with MYALARMS as cache name and the {\tt T} option.
Alarms are then sent by calls to \verb|IC_ALARM|.
An `alarm server' task wishing to receive alarms from a particular channel
calls \verb|IC_BOOK| with the {\tt S} option. 
Receipt of alarms works identically to receipt of broadcasts, \ie 
\verb|IC_ASSOCIATE| (or \verb|IC_WAIT|) must be called. See the
discussion on \broadcast mode.

\subsubsection{Resource lock mode}
A 'resource' can be anything to which access must be controlled; anything
which can accommodate only one user at a time. For example, in the online
database server application, the server can only process requests from one
client at a time. Other clients must not be permitted to interrupt a
'request in progress'. A client wishing to communicate with the server first
takes out a resource lock (\verb|IC_TAKE_LOCK|),
giving the client exclusive access. The client
communicates its request to the server, then drops the resource lock
(\verb|IC_DROP_LOCK|) to
let other clients communicate. Another example is the Fastbus local mode/
global mode interlock. One resource lock can be allocated to each subdetector.
When run control wants to enter global mode, it takes out the resource locks
on all the subdetectors. When the muon chambers want to enter local mode,
the MUCH server takes out the MUCH resource lock (which is only available
if global mode is not in effect). If the lock is held by someone else, the
\verb|IC_CHECK_LOCK| routine can tell the requesting process who the current
holder is.

%=========================================
\newpage
\part{Technical Information on CPC}

This section should be read by anyone writing applications using CPC.
It is assumed that application writers reading this
are knowledgeable about VMS,
and know a bit about (or are willing to read up on) the
Distributed Lock Manager. Chapter 12 of {\sl VAX/VMS Introduction to
System Services} describes the lock manager in detail.

CPC requires VAX/VMS version 5.

%=========================================
\section{CLUSCOM}

\subsection{The 512 byte ILAST array}
When CLUSCOM maps a common block to a disk file, it must map an exact
multiple of 512 bytes (1 page) of memory to the file. Because of this,
there will in general be a piece of memory {\em beyond\/} the end of the last
word of the common ILAST which is mapped to the common. This space
should be occupied by {\em unused\/} space, because information there can be
overwritten by garbage from the mapped file. For this reason, the word
ILAST of the common which is passed to \verb|CC_SETUP| as the last word of
the common should be followed by 512 bytes (128 I*4 or R*4 words) of
unused space. Most simply, just declare ILAST as a 128 element array.

\subsection{Disk to memory mapping: global sections vs QIOs}
In this section the operation of the two modes of CLUSCOM operation,
QIOs and global sections, are discussed in more detail.
Some cautionary advice is given.

In QIO mode, the two representations of the common, one on disk
and one in memory, are completely independent. There is no communication
between the two except when CLUSCOM calls are made to update memory
from disk or vice versa. When a program modifies a variable in
a mapped common, no other process sees the change at the time of the
modification. If the next CLUSCOM call is a memory-to-disk update,
other processes will see the change after this call; if it is a
disk-to-memory update, the change will be obliterated when it is
overwritten by the disk copy of the common.

In global section mode, the situation is very different. On a single
machine, the VMS operating system ensures that all processes mapped to
a global section see exactly the same data at all times. If one process
changes a variable, all other processes on that machine 
see the change immediately.
Processes on other machines will see the change the next time memory
is flushed to disk, which is under the control of VMS and
may happen at any time.

An application which uses the routines \verb|CC_SAVE|, \verb|CC_UPDATE|,
\verb|CC_WRITELOCK| and
\verb|CC_RELEASE| properly will work reliably and in exactly the same
manner in either global section mode or QIO mode.

{\em Note that the SYSGBL privilege is required to use global section mode.}

\subsubsection{{\tt CC\_FLUSH}  usage}
The \verb|CC_FLUSH| routine must however be treated much more carefully in
global section mode. In either mode, this routine can only be used
for commons with only a single writing program.
In QIO mode, reader processes see the changes
of the writer only when \verb|CC_FLUSH| is called, ie when the changes are
completed. In global section mode, changes are visible as they are
made (or at an indeterminate time, for readers running on machines other
than the writer's machine). To avoid the danger of a reader trying to
interpret a partially updated common, the writer should use flags as
follows:
\begin{verbatim}
         CHANGES_IN_PROGRESS = .TRUE.
         <writer makes the necessary changes to common variables>
         CHANGES_IN_PROGRESS = .FALSE.
         CALL CC_FLUSH(...)
\end{verbatim}
Readers must test \verb|CHANGES_IN_PROGRESS|, and must also lock out the writer
when they are reading with another flag. Clearly, it is easier just to use
QIO mode if \verb|CC_FLUSH| is to be used.

\subsection{The Distributed Lock Manager in CLUSCOM}
CLUSCOM uses the VMS Distributed Lock Manager to coordinate access
to the section file and maintain the revision number (kept in the first
word of the lock value block). One lock is used per section. It is
normally held in NL mode. When a write lock is obtained on the file,
the lock is converted to EX mode. EX is dropped when the write lock is
released, and a new revision number is written to the LVB.
The lock name is the cache name prefixed by `\verb|L3%CC_|'.


\subsection{The Revision Number}
A revision number is maintained in the CCOM structure
(a structure internal to the program containing information for each
section) and in the cache
associated with a particular common. When a common is updated to disk,
the revision number is incremented in the cache and CCOM. When an update
from disk is requested, first the revision number is read from the cache.
If it matches the revision number in CCOM, no changes have been made by
other processes since the last update, so no read from disk is performed
(unless the FORCE option 'F' is in effect). If the cache revision number
is different (better be greater!) then the read is performed and the internal
CCOM rev number is set to the cache value.

\subsection{Page alignment}
CLUSCOM sections must be page aligned. To achieve this, the \verb|CDEC$|
directive in VAX Fortran version 5 can be used. See the CLUSCOM example
near the end of this document.

%=========================================
\section{INFOCACHE}


\subsection{The Distributed Lock Manager in INFOCACHE}
Infocache is built over the VMS Distributed Lock Manager.
Two locks are used for each cache in all usage modes except \info
mode, which uses one lock. One lock, the primary lock, is named
with the name of the cache prefixed by `\verb|L3%c|' where {\tt c} is
a character identifying the usage mode. The other lock, the system
lock, has the prefix `\verb|L3%cS_|'. The two-lock scheme is essential
for modes involving signalling in order to ensure that there is no
interference between different process attempting a signalling
operation at nearly the same time. Generally, one of the locks is
held in EX mode throughout the transaction to lock out all other
would-be signallers, while the other lock is used for the signalling.

The signalling mechanism is the BAST, triggered by the signaller
attempting to take out a lock blocked by the signal receivers. The
receivers drop the blocking lock -- inside the \verb|IC_DELIVER| routine
called within the user's AST routine if \verb|IC_ASSOCIATE| is used --
thus enabling the sender to obtain the lock and load the message in
the cache, which is why it is essential that \verb|IC_DELIVER| is called.

\subsection{INFOCACHE usage mode parameters}

In the file INFOCACHE.PARS parameters are defined which identify each
of the \info usage modes (\verb|IC_INFO| can be used to obtain the usage
mode code of a cache). The parameter names are
\begin{verbatim}
   IC__MAILBOX
   IC__BROADCAST
   IC__WATCHER
   IC__ALARM
   IC__INFOCACHE
\end{verbatim}

\section{Scope of CPC -- system-wide or UIC-wide}

The Distributed Lock Manager used by both CLUSCOM and INFOCACHE can
create either system locks or less privileged locks which operate
only within a UIC group. By default, CLUSCOM and INFOCACHE use the
less privileged locks so that special privileges are not required
to use CPC. {\em In this mode, the CLUSCOM sections and caches are
visible only to processes running in the same UIC group.} (On the
L3 cluster, all users are under the same UIC group.) In both 
INFOCACHE and CLUSCOM the option {\tt Y} is available to create a
section or cache which is system-wide, accessible under all UICs. This
mode requires the SYSLCK privilege.

%\input{[abachi.tex]arteuro}             %article style with A4 format
%\input{[abachi.tex]verb}

%\newcommand{\mailbox}{ {\bf mailbox }}
%\newcommand{\info}{{\bf infocache }}
%\newcommand{\alarm}{{\bf alarm }}
%\newcommand{\broadcast}{{\bf broadcast }}
%\newcommand{\watcher}{{\bf process watcher }}

\input{[abachi.tex]softdoc}


%\pagestyle{empty}          %no page numbering
%\begin{document}           % Beginning of text
%\large                     % standard \large and 11 points
%\newpage

%=========================================

\newpage
\part{CPC CLUSCOM Routines Reference}
\vspace*{1cm}
\begin{verbatim}
  List of CLUSCOM routines documented in this section:

  CC_SETUP      to initially set up the CLUSCOM section

  CC_SAVE       to save individual variables in a section
                + CC_SAVEV, CC_SAVEC

  CC_WRITELOCK  to write lock a section
                + CC_WRITELOCKV, CC_WRITELOCKC
  CC_RELEASE    to release a write lock
                + CC_RELEASEV, CC_RELEASEC

  CC_UPDATE     to update memory from disk
                + CC_UPDATEV, CC_UPDATEC

  CC_FLUSH      to flush memory to disk
                + CC_FLUSHV, CC_FLUSHC

  CC_SET_LAST   to control partial read/write of the section     NEW
\end{verbatim}

\begin{routine} %RRRRRRRRRRRRRRRRRRRRRRRRRRRRRRRRRRRRRRRRRRRRRRRRRRRRR
\subroutine
   {SUBROUTINE}{CC\_FLUSH}{(COMID, CHOPT, IERR)}
\begin{overview}
Flushes the common in memory to disk.
\end{overview}
\begin{argdeflist}
\argitem{COMID}{character/readonly}{The name of the CLUSCOM section}
\argitem{CHOPT}{character/readonly}{Option string, upper case}
\argitem{IERR}{integer/modify}{Error code. Error if nonzero}
\end{argdeflist}
\begin{describe}

This routine flushes memory to disk. If a write lock is in place, it remains in
place. If there is no write lock on the common on entry, the
common is write locked during the flush and then unlocked
again. This keeps to a minimum the time during which the common is write
locked and inaccessible to other processes.

This routine is only safe to use when there is only one process which writes
to the common. See the discussion on \verb|CC_FLUSH| in the technical section.

Related routines are \verb|SUBROUTINE CC_FLUSHV(VARNAME,CHOPT,IERR)|
and \verb|SUBROUTINE CC_FLUSHC(STRING,CHOPT,IERR)| for specifying
variable or string names instead of the section name.

\end{describe}
\begin{options}
\opt{none}{}
\end{options}
\begin{returncodes}
\rcode{no error return codes originate from this routine}{}
\end{returncodes}
\begin{examplecode}\begin{verbatim}
*
        COMMON /CCTEST/ IFIRST, <variables>, ILAST(128)
        CHARACTER STRING*10
* ...
        CALL CC_SETUP('CCTEST','CCTEST.MAPCOM',IFIRST,ILAST,' ',IERR)
        IF (IERR.NE.0) GOTO 999
* ...
        DO WHILE (.TRUE.)
          <assignment statements which load the common>
          CALL CC_FLUSH('CCTEST',' ',IERR)
          IF (IERR.NE.0) GOTO 999
          CALL LIB$WAIT(UPDATE_INTERVAL)
        ENDDO
\end{verbatim}\end{examplecode}
\end{routine}


\begin{routine} %RRRRRRRRRRRRRRRRRRRRRRRRRRRRRRRRRRRRRRRRRRRRRRRRRRRRR
\subroutine
   {SUBROUTINE}{CC\_RELEASE}{(COMID, CHOPT, IERR)}
\begin{overview}
Releases a write locked section.
\end{overview}
\begin{argdeflist}
\argitem{COMID}{character/readonly}{The name of the CLUSCOM section}
\argitem{CHOPT}{character/readonly}{Option string, upper case}
\argitem{IERR}{integer/modify}{Error code. Error if nonzero}
\end{argdeflist}
\begin{describe}
This routine releases the write lock on a section, after flushing
memory to disk to save any changes. The revision number of the
section is incremented.

If the section is not write locked when this routine is called, it
does nothing.

This routine should always be used after \verb!CC_WRITELOCK!.

Related routines are \verb|SUBROUTINE CC_RELEASEV(VARNAME,CHOPT,IERR)|
and \verb|SUBROUTINE CC_RELEASEC(STRING,CHOPT,IERR)| for specifying
variable or string names instead of the section name.

\end{describe}
\begin{options}
\opt{none}{}
\end{options}
\begin{returncodes}
\rcode{no error return codes originate from this routine}{}
\end{returncodes}
\begin{examplecode}\begin{verbatim}
*
*   see the example for CC_WRITELOCK
*
\end{verbatim}\end{examplecode}
\end{routine}


\begin{routine} %RRRRRRRRRRRRRRRRRRRRRRRRRRRRRRRRRRRRRRRRRRRRRRRRRRRRR
\subroutine
   {SUBROUTINE}{CC\_SAVE}{(VARNAM, VALUE, CHOPT, IERR)}
\begin{overview}
A new value VALUE for a variable VARNAM in a mapped common is written
to the section on disk
\end{overview}
\begin{argdeflist}
\argitem{VARNAM}{non-char/readonly}{Variable to be updated}
\argitem{VALUE}{non-char/readonly}{New value of variable}
\argitem{CHOPT}{character/readonly}{Option string, upper case}
\argitem{IERR}{integer/modify}{Error code. Error if nonzero}
\end{argdeflist}
\begin{describe}
This routine finds which mapped common VARNAM is contained in, updates
the common from disk if necessary, writes the new value VALUE into VARNAM and
flushes the common to disk. The revision number is incremented.

Related routines are \verb|SUBROUTINE CC_SAVEV(ANAME,AVAL,LNG,CHOPT,IERR)|
which saves new values AVAL in a vector (array) ANAME of length LNG,
and \verb|SUBROUTINE CC_SAVEC(STRNAM,STRVAL,CHOPT,IERR)| to save a new
value STRVAL in a string STRNAM.

If several variables in a given section have to be updated at the same
time, it is more efficient to use \verb|CC_WRITELOCK| and \verb|CC_RELEASE|.

\end{describe}
\begin{options}
\opt{F}{ Force the update from disk to take place, even if the
revision number is unchanged.}
\end{options}
\begin{returncodes}
\rcode{no error return codes originate from this routine}{}
\end{returncodes}
\begin{examplecode}\begin{verbatim}
*
        COMMON /CCTEST/ IFIRST, X, IARRAY(10), STRING, ILAST(128)
        CHARACTER STRING*10
* ...
        CALL CC_SETUP('CCTEST','CCTEST.MAPCOM',IFIRST,ILAST,' ',IERR)
        IF (IERR.NE.0) GOTO 999
* ...
        CALL CC_SAVE(X,7.,' ',IERR)
        IF (IERR.NE.0) GOTO 999
        CALL CC_SAVEC(STRING,'NEWSTRING',' ',IERR)
        IF (IERR.NE.0) GOTO 999
\end{verbatim}\end{examplecode}
\end{routine}

\begin{routine} %RRRRRRRRRRRRRRRRRRRRRRRRRRRRRRRRRRRRRRRRRRRRRRRRRRRRR
\subroutine
   {SUBROUTINE}{CC\_SETUP}{(COMID, FNAME, IFIRST, ILAST, CHOPT, IERR)}
\begin{overview}
This procedure is called to set up the mapping of a
CLUSCOM section. It creates the
mapping between a common block in memory and a disk file.
\end{overview}
\begin{argdeflist}
\argitem{COMID}{character/readonly}{The name assigned to the CLUSCOM
section by the user. Maximum length is 20 characters.}
\argitem{FNAME}{character/readonly}{The filename to contain the mapped
section. The filename convention is {\tt <comid>.MAPCOM} }
\argitem{IFIRST}{integer/readonly}{The first word of the common block
to be mapped}
\argitem{ILAST}{integer/readonly}{The last word of the common block to
be mapped}
\argitem{CHOPT}{character/readonly}{Option string, upper case}
\argitem{IERR}{integer/modify}{Error code. Error if nonzero}
\end{argdeflist}
\begin{describe}
This routine sets up the mapping between a common block in memory and
a disk file (referred to as a CLUSCOM section).
The specified file (referred to as a section file)
is opened and mapped to the memory
region IFIRST to ILAST. If the file already exists, it is opened and
mapped; if it does not exist it is created. Section files are always
created with group and world write access, so that privileges are
not needed to map them.

There are two modes of access -- using QIOs (block I/O in earlier
CLUSCOM versions) or global sections.
These two modes are discussed in the technical section.
They are functionally equivalent from the user's point of view (but see
the technical section).
The default mode is QIOs. Global section mode is usually faster,
but {\em the SYSGBL privilege is required to use the global section
 mode}. No special privileges are required to use QIO mode.

Note that the ILAST location should mark the beginning of 512 bytes
of unused space. This is most easily achieved by making ILAST a 128 element
array. See the discussion in the technical part.

\end{describe}
\begin{options}
\opt{G}{ Use global section (rather than the default QIO) mode.}
\opt{R}{ File is opened READONLY. The file is opened with only read
access required. Attempts to write to the file are refused.}
\opt{W}{ File is opened with a write lock. This option should be used
carefully. No other process can gain read or write access to the file until the
file is released by the write lock holder.}
\opt{C}{ Force creation of a new map file, even if a version exists.}
\opt{O}{ Open old map file. If no file exists, an error is returned.}
\opt{M}{ by default, when an existing map file is opened,
memory is initialized from disk rather than vice
versa. To initialize disk from memory, use this
option. NOTE that in global section mode, this option has
the effect of {\em zeroing} both memory and disk.}
\opt{Y}{ system mode. Cache is recognised across all UICs. See the
  technical section. }
\end{options}
\begin{returncodes}
\rcode{CC\_E\_BADNAME}{ identifier string too long}
\rcode{CC\_E\_BADFILE}{ filename too long}
\rcode{CC\_E\_MAXEX}{ maximum number of mapped commons exceeded}
\rcode{CC\_E\_NOFILE}{ specified section file does not exist}
\end{returncodes}
\begin{examplecode}\begin{verbatim}
*
**        the common to map
        COMMON /CCTEST/ IFIRST, IARRAY(10), STRING, ILAST(128)
        CHARACTER STRING*10
*
        CALL CC_SETUP('CCTEST','CCTEST.MAPCOM',IFIRST,ILAST,' ',IERR)
        IF (IERR.NE.0) GOTO 999
* ...
\end{verbatim}\end{examplecode}
\end{routine}


\begin{routine} %RRRRRRRRRRRRRRRRRRRRRRRRRRRRRRRRRRRRRRRRRRRRRRRRRRRRR
\subroutine
   {SUBROUTINE}{CC\_SET\_LAST}{(COMID, ILAST, CHOPT, IERR)}
\begin{overview}
This procedure tells CLUSCOM to read/write only part of the section.

\end{overview}
\begin{argdeflist}
\argitem{COMID}{character/readonly}{The name assigned to the CLUSCOM
section by the user. Maximum length is 20 characters.}
\argitem{ILAST}{integer/readonly}{The last word of the common block to
read/write}
\argitem{CHOPT}{character/readonly}{Option string, upper case}
\argitem{IERR}{integer/modify}{Error code. Error if nonzero}
\end{argdeflist}
\begin{describe}
This procedure sets the final word of a mapped common to 
be written to or read from the section file. It tells CLUSCOM to
to write/read only part of a common, to save time when only a small part
of a large common, the region up to ILAST, is used.

To reset reading/writing to use the entire section, call this routine with
the {\tt R} option.

Note that no I/O is actually performed by this routine.
\end{describe}
\begin{options}
\opt{R}{Reset to read/write the entire common. ILAST ignored.}
\end{options}
\begin{returncodes}
\rcode{CC\_E\_MISMATCH}{ name/variable mismatch. The ILAST location is not
   in the indicated section.}
\end{returncodes}
\begin{examplecode}\begin{verbatim}
*
*       eg. if the mapped common is
   COMMON /MY/ IFIRST, IARRAY(10000), ILAST(128)
* ...
*       then
   CALL CC_SET_LAST(COMID,IARRAY(100),' ',IERR)
*       will cause /MY/ to be read/written only up to the first page
*       boundary past IARRAY(100).
\end{verbatim}\end{examplecode}
\end{routine}


\begin{routine} %RRRRRRRRRRRRRRRRRRRRRRRRRRRRRRRRRRRRRRRRRRRRRRRRRRRRR
\subroutine
   {SUBROUTINE}{CC\_UPDATE}{(COMID, CHOPT, IERR)}
\begin{overview}
Updates the common block in memory from the disk file.
\end{overview}
\begin{argdeflist}
\argitem{COMID}{character/readonly}{The name of the CLUSCOM section}
\argitem{CHOPT}{character/readonly}{Option string, upper case}
\argitem{IERR}{integer/modify}{Error code. Error if nonzero}
\end{argdeflist}
\begin{describe}
This routine checks the revision number
\footnote{see the discussion on revision numbers in the technical section}
of the section and
updates from disk if and only if changes have been made by other
processes since the last update. Checking the revision number is very
fast, so that it is not inefficient to use \verb|CC_UPDATE| frequently,
for example in a loop which refreshes a screen display every few
seconds based on information in a CLUSCOM section.

Related routines are \verb|SUBROUTINE CC_UPDATEV(VARNAME,CHOPT,IERR)|
and \verb|SUBROUTINE CC_UPDATEC(STRING,CHOPT,IERR)| to update a
variable VARNAME or character string STRING respectively, where
VARNAME or STRING are contained in mapped commons. Note that the
whole common is updated, not just the specified variable; these
routines just provide a means of identifying a mapped common by a variable
contained within it rather than by its name.

\end{describe}
\begin{options}
\opt{F}{ Force the update from disk to take place, even if the
revision number is unchanged.}
\end{options}
\begin{returncodes}
\rcode{no error return codes originate from this routine}{}
\end{returncodes}
\begin{examplecode}\begin{verbatim}
*
        COMMON /CCTEST/ IFIRST, IARRAY(10), STRING, ILAST(128)
        CHARACTER STRING*10
* ...
        CALL CC_SETUP('CCTEST','CCTEST.MAPCOM',IFIRST,ILAST,' ',IERR)
        IF (IERR.NE.0) GOTO 999
* ...
        CALL CC_UPDATE('CCTEST',' ',IERR)
        IF (IERR.NE.0) GOTO 999
        CALL CC_UPDATEV(IARRAY(3),' ',IERR)
        IF (IERR.NE.0) GOTO 999
        CALL CC_UPDATEC(STRING,' ',IERR)
        IF (IERR.NE.0) GOTO 999
\end{verbatim}\end{examplecode}
\end{routine}

\begin{routine} %RRRRRRRRRRRRRRRRRRRRRRRRRRRRRRRRRRRRRRRRRRRRRRRRRRRRR
\subroutine
   {SUBROUTINE}{CC\_WRITELOCK}{(COMID, CHOPT, IERR)}
\begin{overview}
Locks a section in write mode.
\end{overview}
\begin{argdeflist}
\argitem{COMID}{character/readonly}{The name of the CLUSCOM section}
\argitem{CHOPT}{character/readonly}{Option string, upper case}
\argitem{IERR}{integer/modify}{Error code. Error if nonzero}
\end{argdeflist}
\begin{describe}
This routine updates a section from disk (if necessary) and locks it
in write mode, so that {\em no other process has any access to the
section -- read or write --
until the write lock is released with \verb|CC_RELEASE|}.
It should always be used in conjunction with \verb|CC_RELEASE|
as in the example below.

It updates from disk if and only if changes have been made by other
processes since the last update.

Related routines are \verb|SUBROUTINE CC_WRITELOCKV(VARNAME,CHOPT,IERR)|
and \verb|SUBROUTINE CC_WRITELOCKC(STRING,CHOPT,IERR)| for specifying
variable or string names instead of the section name.

\end{describe}
\begin{options}
\opt{F}{ Force the update from disk to take place, even if the
revision number is unchanged.}
\end{options}
\begin{returncodes}
\rcode{CC\_E\_NOWRTLOCK}{ attempt to write lock a readonly section}
\end{returncodes}
\begin{examplecode}\begin{verbatim}
*
        COMMON /CCTEST/ IFIRST, X, IARRAY(10), STRING, ILAST(128)
        CHARACTER STRING*10
* ...
        CALL CC_SETUP('CCTEST','CCTEST.MAPCOM',IFIRST,ILAST,' ',IERR)
        IF (IERR.NE.0) GOTO 999
* ...
        CALL CC_WRITELOCK('CCTEST',' ',IERR)
        IF (IERR.NE.0) GOTO 999
        X=X+2.
        STRING='NEWSTRING'
        CALL CC_RELEASE('CCTEST',' ',IERR)
        IF (IERR.NE.0) GOTO 999
\end{verbatim}\end{examplecode}
\end{routine}


%=========================================

\newpage
\part{CPC INFOCACHE Routines Reference}
\vspace*{1cm}
\begin{verbatim}
  List of INFOCACHE routines documented in this section:

  IC_BOOK       book a cache
  IC_BOOKED     logical, true if a cache is booked

  IC_READ       INFOCACHE mode: read cache
  IC_WRITE      INFOCACHE mode: write cache

  IC_SIGNAL     MAILBOX mode:   send signal to master
  IC_SEND       MAILBOX mode:   send reply to client

  IC_ALARM      ALARM mode:     send an alarm

  IC_BROADCAST  BROADCAST mode: send a broadcast

  IC_WATCH      WATCHER mode:   set up a watch on the lockholder
  IC_STATUS     WATCHER mode:   load status info into lock cache

  IC_TAKE_LOCK  RESOURCE LK mode: lock the resource              NEW
  IC_DROP_LOCK  RESOURCE LK mode: drop lock on the resource      NEW
  IC_CHECK_LOCK RESOURCE LK mode: check resource lock status     NEW

  IC_ASSOCIATE  general. Associate an AST routine with a cache
  IC_NUM_ID     general. Convert cache number to name in AST
  IC_DELIVER    general. Read message in AST

  IC_WAIT       general. Wait for an interrupt from a cache

  IC_INFO       general. To obtain information on caches

  IC_MATCH      general. To obtain lists of cache groups

  IC_RELEASE    general. To un-book a cache
\end{verbatim}

\begin{routine} %RRRRRRRRRRRRRRRRRRRRRRRRRRRRRRRRRRRRRRRRRRRRRRRRRRRRR
\subroutine
   {SUBROUTINE}{IC\_ALARM}{(CANAME, CADAT, CHOPT, IERR)}
\begin{overview}
An  \alarm mode routine.

Sends an alarm.
\end{overview}
\begin{argdeflist}
\argitem{CANAME}{character/readonly}{The name assigned to the cache}
\argitem{CADAT}{16 bytes/readonly}{The cache contents, \ie the alarm
  message}
\argitem{CHOPT}{character/readonly}{Option string, upper case}
\argitem{IERR}{integer/modify}{Error code. Error if nonzero}
\end{argdeflist}
\begin{describe}

An alarm is sent to all
processes which have booked the cache in `alarm server' mode.

The contents of CADAT are sent with the alarm as an alarm
message, and can
be retrieved by processes receiving the alarm interrupt in the same
way that broadcasts are retrieved (see routines
\verb|IC_WAIT| and \verb|IC_DELIVER|).

The alarm message might contain, for instance
\begin{itemize}
\item a message code
  as generated by the VMS MESSAGE utility (4 bytes) 
\item an identifier
  of the sending process (1 to 4 bytes)
\item additional alarm information in the remaining 8 to 11 bytes.
\end{itemize}

\end{describe}
\begin{options}
\opt{none}{}
\end{options}
\begin{returncodes}
\rcode{any {\tt IC\_BROADCAST} error code}{}
\end{returncodes}
\end{routine}

\begin{routine} %RRRRRRRRRRRRRRRRRRRRRRRRRRRRRRRRRRRRRRRRRRRRRRRRRRRRR
\subroutine
   {SUBROUTINE}{IC\_ASSOCIATE}{(CANAME, CADAT, ASTSUB, CHOPT, IERR)}
\begin{overview}
A routine used by interrupt receiving processes in \broadcast,
\alarm, or \mailbox mode. It associates an AST routine \verb|ASTSUB|
with the specified cache.

\end{overview}
\begin{argdeflist}
\argitem{CANAME}{character/readonly}{The name assigned to the cache}
\argitem{CADAT}{16 bytes/readonly}{Not used at present}
\argitem{ASTSUB}{external routine}{The AST routine name}
\argitem{CHOPT}{character/readonly}{Option string, upper case}
\argitem{IERR}{integer/modify}{Error code. Error if nonzero}
\end{argdeflist}
\begin{describe}

This routine can be used to associate an AST routine with a specified
cache through which interrupts are received. When an interrupt is
received, the AST routine is called and user code in the AST routine
receives and processes the message. The AST might process
alarms in \alarm mode, broadcasts in \broadcast mode, or client messages
in \mailbox mode.

Inside the user's AST routine, the routine \verb|IC_DELIVER| {\em must} be
called. This routine reads the message from the sender and completes
the message handshaking. In the case of \mailbox mode, a master must call
\verb|IC_SEND| after \verb|IC_DELIVER| in the AST routine to send the
reply to the client (and complete the handshaking).

Using this routine, a single process can handle interrupts from many
sources at the same time. The coding example below illustrates how
one AST routine processes interrupts from more than one cache.

While the process is waiting for interrupts
from any of its caches it can do other things.

For example, a display program might have one AST routine associated with
a broadcast cache, to receive and display general broadcast messages;
several alarm caches associated with another AST routine to receive and
signal alarms; and another AST routine associated to
a \mailbox type cache in master mode to receive requests from other
processes or interactive users. The main program in this example might
be a loop with a 30 second hibernate, which refreshes the
screen displays when it wakes up every 30 secs.

\end{describe}
\begin{options}
\opt{none}{}
\end{options}
\begin{returncodes}
\rcode{IC\_E\_INVMODE}{cache is not an appropriate usage mode.}
\end{returncodes}
\begin{examplecode}\begin{verbatim}
*
        EXTERNAL ASTALARM, ASTCMD
        INTEGER ICACHE(4)
        ...
        CALL IC_BOOK('HVALARM',ICACHE,'S',IERR)
        IF (IERR.NE.0) GOTO 999
        CALL IC_ASSOCIATE('HVALARM',ICACHE,ASTALARM,' ',IERR)
        IF (IERR.NE.0) GOTO 999
        CALL IC_BOOK('FBALARM',ICACHE,'S',IERR)
        IF (IERR.NE.0) GOTO 999
        CALL IC_ASSOCIATE('FBALARM',ICACHE,ASTALARM,' ',IERR)
        IF (IERR.NE.0) GOTO 999
        CALL IC_BOOK('MYCMDS',ICACHE,'M',IERR)
        IF (IERR.NE.0) GOTO 999
        CALL IC_ASSOCIATE('MYCMDS',ICACHE,ASTCMDS,' ',IERR)
        IF (IERR.NE.0) GOTO 999
        DO WHILE (.TRUE.)
          CALL REFRESH_DIPLAY
          CALL LIB$WAIT(30.)
        ENDDO
        ...
*
        SUBROUTINE ASTALARM(NUM)
*         the NUM parameter passed to the AST routine is the
*         cache identifier, which is converted to the cache name
*         by the routine IC_NUM_ID below.
        CHARACTER NAME*30
        INTEGER ICACHE(4)
        CALL IC_NUM_ID(NUM,NAME,IERR)
        IF (IERR.NE.0) GOTO 999
*         receive message. Mandatory!
        CALL IC_DELIVER(NAME,ICACHE,' ',IERR)
        IF (IERR.NE.0) GOTO 999
        IF (NAME(1:LENOCC(NAME)).EQ.'HVALARM') THEN
          ...
        ELSE IF (NAME(1:LENOCC(NAME)).EQ.'FBALARM') THEN
          ...
        ENDIF
        END
*
        SUBROUTINE ASTCMDS(NUM)
        CHARACTER NAME*30
        INTEGER ICACHE(4)
        CALL IC_NUM_ID(NUM,NAME,IERR)
        IF (IERR.NE.0) GOTO 999
*         receive message from the client. Mandatory!
        CALL IC_DELIVER(NAME,ICACHE,' ',IERR)
        IF (IERR.NE.0) GOTO 999
        <act, based on contents of ICACHE, and fill ICACHE with a reply>
*         send reply to the client. Mandatory!
        CALL IC_SEND(NAME,ICACHE,' ',IERR)
        IF (IERR.NE.0) GOTO 999
        END
\end{verbatim}\end{examplecode}
\end{routine}

\begin{routine} %RRRRRRRRRRRRRRRRRRRRRRRRRRRRRRRRRRRRRRRRRRRRRRRRRRRRR
\subroutine
   {SUBROUTINE}{IC\_BOOK}{(CANAME, CADAT, CHOPT, IERR)}
\begin{overview}
This procedure is called to book a cache. The usage mode is determined
by the option selected.
\end{overview}
\begin{argdeflist}
\argitem{CANAME}{character/readonly}{The name assigned to the cache.
  Maximum length is 20 characters.}
\argitem{CADAT}{16 bytes/modify}{For some usage modes, the existing value
of the cache is returned, or an initialization value is passed.}
\argitem{CHOPT}{character/readonly}{Option string, upper case}
\argitem{IERR}{integer/modify}{Error code. Error if nonzero}
\end{argdeflist}
\begin{describe}
This routine books the cache of the specified name for use in the program,
with the usage mode determined by the option selected.

This routine must be called before any other IC routines are called for
a particular cache.

Only \em{one} mode option can be passed for a given cache (a cache can
have only one usage mode!).
\end{describe}
\begin{options}
\opt{none}{ Default usage mode is \info. Any number of processes can book
  a cache in this mode. All can use {\tt IC\_WRITE}, {\tt IC\_READ} to
  write and read the cache.}
\opt{M}{ \mailbox mode: master process. The calling process becomes the
  master (message receiver) of the cache. {\tt IC\_ASSOCIATE} must be
  called after {\tt IC\_BOOK}. Only {\em one} process can book a given
  cache as a master.}
\opt{C}{ \mailbox mode: client process. Any number of clients can book
   a given cache. Any client can use {\tt IC\_SIGNAL} to communicate with
   the master.}
\opt{B}{ \broadcast mode. Any number of processes can book a broadcast
  cache. All must call {\tt IC\_ASSOCIATE} or {\tt IC\_WAIT} after
  {\tt IC\_SETUP}.}
\opt{L}{ \watcher mode: program lock. The process becomes the lockholder
  of the cache. The lock is relenquished (to a watcher) when the process
  crashes or when it deliberately releases the lock. The lockholder can
  use {\tt IC\_STATUS} to load process status info into the cache.}
\opt{P}{ \watcher mode: program status watcher. Any number of processes
  who want to monitor the status of the lockholder and receive notification
  when it crashes can book the cache in this mode and call {\tt IC\_WATCH}.}
\opt{T}{ \alarm mode: task which sends out alarms. Any number of processes
   can book an alarm cache with this option. These processes can use
  {\tt IC\_ALARM} to send alarms to listening servers.}
\opt{S}{ \alarm mode: alarm server which receives and processes alarms
  sent by alarm tasks (cache holders with option {\tt T}).
  Any number of processes can book an alarm cache as a server. These
  processes must call {\tt IC\_ASSOCIATE} after {\tt IC\_BOOK}.}
\opt{R}{ \reslock mode: Book the resource lock. After booking the resource lock,
  {\tt IC\_TAKE\_LOCK} can be called to take out the lock.}
\opt{X}{ cache should exist. Error returned if it does not.}
\opt{Y}{ system mode. Cache is recognised across all UICs. See the
  technical section.}
\end{options}
\begin{returncodes}
\rcode{IC\_E\_BADNAME}{ cache name too long}
\rcode{IC\_E\_MAXEX}{ maximum number of caches exceeded}
\end{returncodes}
\end{routine}

\begin{routine} %RRRRRRRRRRRRRRRRRRRRRRRRRRRRRRRRRRRRRRRRRRRRRRRRRRRRR
\subroutine
   {LOGICAL FUNCTION}{IC\_BOOKED}{(CANAME)}
\begin{overview}
Checks whether a cache is booked.

\end{overview}
\begin{argdeflist}
\argitem{CANAME}{character/readonly}{The name assigned to the cache}
\end{argdeflist}
\begin{describe}

Returns TRUE if the cache is booked.

\end{describe}
\begin{options}
\opt{none}{}
\end{options}
\end{routine}

\begin{routine} %RRRRRRRRRRRRRRRRRRRRRRRRRRRRRRRRRRRRRRRRRRRRRRRRRRRRR
\subroutine
   {SUBROUTINE}{IC\_BROADCAST}{(CANAME, CADAT, CHOPT, IERR)}
\begin{overview}
A  \broadcast mode routine.

Sends a broadcast.
\end{overview}
\begin{argdeflist}
\argitem{CANAME}{character/readonly}{The name assigned to the cache}
\argitem{CADAT}{16 bytes/readonly}{The cache contents written.}
\argitem{CHOPT}{character/readonly}{Option string, upper case}
\argitem{IERR}{integer/modify}{Error code. Error if nonzero}
\end{argdeflist}
\begin{describe}

A broadcast is sent to all processes which have the specified cache
booked (and have enabled interrupt reception with a call to
{\tt IC\_ASSOCIATE} or {\tt IC\_WAIT}). 
The contents of CADAT are sent with the broadcast, and are
retrieved by processes receiving the broadcast interrupt via
{\tt IC\_WAIT} or (if {\tt IC\_ASSOCIATE} is used) \verb|IC_DELIVER|.

This routine has for technical reasons a delay of 100 msec built into
it at present. This delay will be reduced if possible in a future
version. There will always be a delay equal to the longest time taken
to receive the broadcast by any of the processes that has the cache booked.

\end{describe}
\begin{options}
\opt{none}{}
\end{options}
\begin{returncodes}
\rcode{IC\_E\_NOBCAST}{cache type is not alarm or broadcast}
\end{returncodes}
\end{routine}

\begin{routine} %RRRRRRRRRRRRRRRRRRRRRRRRRRRRRRRRRRRRRRRRRRRRRRRRRRRRR
\subroutine
   {SUBROUTINE}{IC\_CHECK\_LOCK}{NAME,HELD,IU,IP,IC,PRC,CPU,CHOPT,IERR)}
\begin{overview}
Resource lock mode. Check the status of a resource lock.

\end{overview}
\begin{argdeflist}
\argitem{NAME}{character/readonly}{The resource lock name}
\argitem{HELD}{logical/modify}{whether the lock is held by another process}
\argitem{IU}{8 bytes/modify}{user-defined component of the lock's cache}
\argitem{IP}{integer/modify}{process ID of the lock holder}
\argitem{IC}{integer/modify}{CSID of the lock holder, for experts playing with
 GETSYI}
\argitem{PRC}{character/modify}{process name of the lock holder}
\argitem{CPU}{character/modify}{node name of the lock holder}
\argitem{CHOPT}{character/readonly}{Option string, upper case}
\argitem{IERR}{integer/modify}{Error code. Error if nonzero}
\end{argdeflist}
\begin{describe}

This routine enables a process to determine who is holding a resource lock.
Note that in VMS 5.1 the process ID is unknown when the lockholder is
running on a remote node. This may be rectified in VMS 5.2.

The HELD logical is true if a process anywhere on the cluster currently
holds the lock; false otherwise.

A user-defined code identifying the lockholder can be loaded into the
lock's cache in the \verb|IC_TAKE_LOCK| call and can then be read in
the call to this routine.

\end{describe}
\begin{options}
\opt{none}{}
\end{options}
\begin{returncodes}
\rcode{IC\_E\_INVMODE}{cache is not the correct usage mode.}
\end{returncodes}
\end{routine}

\begin{routine} %RRRRRRRRRRRRRRRRRRRRRRRRRRRRRRRRRRRRRRRRRRRRRRRRRRRRR
\subroutine
   {SUBROUTINE}{IC\_DELIVER}{(CANAME, CADAT, CHOPT, IERR)}
\begin{overview}
A routine used within AST routines to receive a message. Used in
\mailbox, \broadcast, and \alarm modes.

\end{overview}
\begin{argdeflist}
\argitem{CANAME}{character/readonly}{The name assigned to the cache}
\argitem{CADAT}{16 bytes/modify}{The cache contents at time of interrupt,
  \ie the message from the sender}
\argitem{CHOPT}{character/readonly}{Option string, upper case}
\argitem{IERR}{integer/modify}{Error code. Error if nonzero}
\end{argdeflist}
\begin{describe}

This routine is called inside an AST routine to read the message coming
from the signaller (whether the signaller is a broadcaster, a client,
or an alarm task, depending on the usage mode).

This routine {\em must} be called, even in the event that the
message itself is not of interest. It is mandatory because the handshaking
between sender and receiver is completed in this routine (except for
\mailbox mode, in which the handshaking is completed by \verb|IC_SEND|).

The message is returned in CADAT.

\end{describe}
\begin{options}
\opt{C}{read the cache value from the data structure. Value is {\em not}
  read from the cache itself. An rarely needed option.}
\end{options}
\begin{returncodes}
\rcode{IC\_E\_INVMODE}{cache is not the correct usage mode.}
\end{returncodes}
\begin{examplecode}\begin{verbatim}
*
**        see example for IC_ASSOCIATE.
*
\end{verbatim}\end{examplecode}
\end{routine}

\begin{routine} %RRRRRRRRRRRRRRRRRRRRRRRRRRRRRRRRRRRRRRRRRRRRRRRRRRRRR
\subroutine
   {SUBROUTINE}{IC\_DROP\_LOCK}{NAME, CHOPT, IERR)}
\begin{overview}
Resource lock mode. Drops the lock.

\end{overview}
\begin{argdeflist}
\argitem{NAME}{character/readonly}{The resource lock name}
\argitem{CHOPT}{character/readonly}{Option string, upper case}
\argitem{IERR}{integer/modify}{Error code. Error if nonzero}
\end{argdeflist}
\begin{describe}

This routine drops the specified lock, if it is held. If it is not held,
the routine returns quietly. It loads the lock's cache with zeros before
dropping.

\end{describe}
\begin{options}
\opt{none}{}
\end{options}
\begin{returncodes}
\rcode{IC\_E\_INVMODE}{cache is not the correct usage mode.}
\rcode{IC\_E\_INACTIVE}{AST delivered during releasing of the cache,
  and should be ignored.}
\end{returncodes}
\end{routine}

\begin{routine} %RRRRRRRRRRRRRRRRRRRRRRRRRRRRRRRRRRRRRRRRRRRRRRRRRRRRR
\subroutine
   {SUBROUTINE}{IC\_INFO}{(CANAME, IMODE, LDATA, CHOPT, IERR)}
\begin{overview}
A general routine. Not restricted to a particular usage mode.

Used to obtain information on caches.
\end{overview}
\begin{argdeflist}
\argitem{CANAME}{character/readonly}{The name assigned to the cache}
\argitem{IMODE}{integer/readonly}{The usage mode of the cache. Only needed
  if the cache for which information is requested has not been booked. If the
  cache has been booked, IMODE can be 0. For info on usage mode parameters
  see the technical section.}
\argitem{LDATA}{int array/modify}{The cache info returned. Contents
depend on selected option.}
\argitem{CHOPT}{character/readonly}{Option string, upper case}
\argitem{IERR}{integer/modify}{Error code. Error if nonzero}
\end{argdeflist}
\begin{describe}

The various options of this routine permit the retrieval of information
associated with the cache CANAME (except option {\tt I} for which CANAME
is a returned string). See the descriptions of the options below.

The option {\tt `LP'} with the name specification {`*'} provides a
dump to the terminal of known caches, useful for debugging.

\end{describe}
\begin{options}
\opt{C}{return cache contents in LDATA; 16 bytes}
\opt{N}{number of processes with cache booked returned in LDATA(1).
Can be used to check cache existence.}
\opt{L}{list of processes holding this cache dumped to screen. Nothing
returned in LDATA (yet)}
\opt{M}{return usage mode of cache in IMODE. See discussion on 
\INFOCACHE parameters in the technical section for the usage mode codes.}
\opt{S}{return ID numbers of all caches containing the string
NAME. LDATA(1) is number of caches returned;
thereafter follow ID numbers. After ID numbers,
in LDATA(1+LDATA(1)+I), are mode codes. Should not be called directly by
user; called from {\tt IC\_MATCH}.}
\opt{I}{a cache ID is passed in LDATA. The name of the cache is
returned in CANAME.}
\opt{P}{Print. Can be used in conjunction with {\tt C,N,M,S} options to
send info to screen as well as return it in LDATA.}
\end{options}
\begin{returncodes}
\rcode{IC\_E\_UNKNOWN}{ cache has unknown usage mode. System error.}
\rcode{IC\_E\_NOMATCH}{ cache name is unknown}
\rcode{IC\_E\_BADMODE}{ unknown lock mode. System error.}
\end{returncodes}
\end{routine}


\begin{routine} %RRRRRRRRRRRRRRRRRRRRRRRRRRRRRRRRRRRRRRRRRRRRRRRRRRRRR
\subroutine
   {SUBROUTINE}{IC\_MATCH}{(MATCH, NMATCH, IDNUM, IMODE, CHOPT, IERR)}
\begin{overview}
A general routine for information retrieval. 
Not restricted to a particular usage mode.

Retrieves groups of caches satisfying a character string match.

\end{overview}
\begin{argdeflist}
\argitem{MATCH}{character/readonly}{The string to match}
\argitem{NMATCH}{integer/modify}{The number of caches with names containing
  the specified string}
\argitem{IDNUM}{int array/modify}{The array of cache IDs returned. There
  are {\tt NMATCH} elements. {\tt IC\_INFO} can be used to convert the ID
  to a cache name (see coding example below)}
\argitem{IMODE}{int array/modify}{The array of cache usage modes returned.
  There are {\tt NMATCH} elements.}
\argitem{CHOPT}{character/readonly}{Option string, upper case}
\argitem{IERR}{integer/modify}{Error code. Error if nonzero}
\end{argdeflist}
\begin{describe}

Given a string to match, {\tt MATCH}, this routine returns
 in array IDNUM a list of
caches whose names contain the string. Array {\tt IMODE}
contains usage mode codes for the different caches. See the technical
section for more info on usage mode codes.
NMATCH is the number of matches found.

This routine provides a means of implementing cache groups.
Cache groups can be established by using part
of each cache name as a unique group identifier. This routine can
then be used to retrieve the members of a group and perform an
operation on some or all of the group member (see the example below).

NOTE that only those caches which a process {\em on the local node}
has booked are returned in the list! That is, if cache 
\verb|/MU/HV| is booked
by a process on VXL3ON but not by any processes on VSL3RC, a
process on VSL3RC which executes the example below will not find
any matches.

\end{describe}
\begin{options}
\opt{none}{}
\end{options}
\begin{returncodes}
\rcode{any {\tt IC\_INFO} return code}{}
\end{returncodes}
% $$$ test
\begin{examplecode}\begin{verbatim}
*    obtain a list of all MUon caches (caches with /MU/ in the name).
   CALL IC_MATCH('/MU/',NCACHE,ID,IM,' ',IERR)
*    process each of the caches in the retrieved group
   DO I=1,NCACHE
     CALL IC_INFO(CNAME,0,ID(I),'I',IERR) ! cache name is returned
*      one might for example want to send a signal to each
*      mailbox cache in the group.
     IF (IM(I).EQ.IC__MAILBOX) CALL IC_SIGNAL(CNAME,...)
   ENDDO
\end{verbatim}\end{examplecode}
\end{routine}


\begin{routine} %RRRRRRRRRRRRRRRRRRRRRRRRRRRRRRRRRRRRRRRRRRRRRRRRRRRRR
\subroutine
   {SUBROUTINE}{IC\_NUM\_ID}{(NUM, CANAME, CHOPT, IERR)}
\begin{overview}
Converts a cache numeric identifier to a string identifier.
\end{overview}
\begin{argdeflist}
\argitem{NUM}{integer/readonly}{numeric identifier of cache}
\argitem{CANAME}{character/modify}{cache name; allow up to 30 characters}
\argitem{CHOPT}{character/readonly}{Option string, upper case}
\argitem{IERR}{integer/modify}{Error code. Error if nonzero}
\end{argdeflist}
\begin{describe}

This routine is used inside AST routines to decode the name of the
cache that has triggered the interrupt. The numeric identifier NUM is
available as the single parameter passed to the AST routine. This
routine converts NUM to the name string. The string is padded with
blanks.

\end{describe}
\begin{options}
\opt{none}{}
\end{options}
\begin{returncodes}
\rcode{IC\_E\_OUTRNG}{cache number out of range. An invalid cache number
  has been passed.}
\rcode{IC\_E\_INACTIVE}{cache is inactive. System error.}
\end{returncodes}
\begin{examplecode}\begin{verbatim}
*
**        see example for IC_ASSOCIATE.
*
\end{verbatim}\end{examplecode}
\end{routine}


\begin{routine} %RRRRRRRRRRRRRRRRRRRRRRRRRRRRRRRRRRRRRRRRRRRRRRRRRRRRR
\subroutine
   {SUBROUTINE}{IC\_READ}{(CANAME, CADAT, CHOPT, IERR)}
\begin{overview}
An \info mode routine.

Used to read an information cache.
\end{overview}
\begin{argdeflist}
\argitem{CANAME}{character/readonly}{The name assigned to the cache}
\argitem{CADAT}{16 bytes/modify}{The cache contents returned.}
\argitem{CHOPT}{character/readonly}{Option string, upper case}
\argitem{IERR}{integer/modify}{Error code. Error if nonzero}
\end{argdeflist}
\begin{describe}

Returns the contents of the specified information cache in CADAT.

Can be used at any time by any process that has the cache booked.

The contents of the cache as written by the last process to use
\verb|IC_WRITE| will be returned in CADAT.

\end{describe}
\begin{options}
\opt{none}{}
\end{options}
\begin{returncodes}
\rcode{IC\_E\_EXREAD}{attempted read while in EX mode. System error.}
\rcode{IC\_E\_INVMODE}{cache is not the correct usage mode.}
\end{returncodes}
\end{routine}


\begin{routine} %RRRRRRRRRRRRRRRRRRRRRRRRRRRRRRRRRRRRRRRRRRRRRRRRRRRRR
\subroutine
   {SUBROUTINE}{IC\_RELEASE}{(CANAME, CADAT, CHOPT, IERR)}
\begin{overview}
A general routine. Releases the specified cache.
\end{overview}
\begin{argdeflist}
\argitem{CANAME}{character/readonly}{The name assigned to the cache}
\argitem{CADAT}{16 bytes/readonly}{The cache contents written, if
  the {\tt W} option is used}
\argitem{CHOPT}{character/readonly}{Option string, upper case}
\argitem{IERR}{integer/modify}{Error code. Error if nonzero}
\end{argdeflist}
\begin{describe}

This routine removes the specified cache from the active caches of
the current process. No calls referencing the specified cache will
work after calling this routine (except \verb|IC_BOOK| of course),
and no interrupts will be received from the released cache.

A lockholder (\watcher mode) can use this routine in order to release
the lock in an orderly manner when the program exits normally without
crashing; the {\tt W} option can be used to write to the cache a
code indicating to watchers that the lock was released as a result of a
normal exit and not a crash.

\end{describe}
\begin{options}
\opt{W}{write to the cache the contents of CADAT before releasing. Can
  be used only by masters in \mailbox mode and lockholders in \watcher
  mode.}
\end{options}
\begin{returncodes}
\rcode{none}{}
\end{returncodes}
\end{routine}


\begin{routine} %RRRRRRRRRRRRRRRRRRRRRRRRRRRRRRRRRRRRRRRRRRRRRRRRRRRRR
\subroutine
   {SUBROUTINE}{IC\_SEND}{(CANAME, CADAT, CHOPT, IERR)}
\begin{overview}
A routine used within \mailbox master mode AST routines to send a
reply to a signalling client.

\end{overview}
\begin{argdeflist}
\argitem{CANAME}{character/readonly}{The name assigned to the cache}
\argitem{CADAT}{16 bytes/readonly}{The reply to the client}
\argitem{CHOPT}{character/readonly}{Option string, upper case}
\argitem{IERR}{integer/modify}{Error code. Error if nonzero}
\end{argdeflist}
\begin{describe}

When a \mailbox master processes a client's signal inside an AST routine,
after receiving the message with \verb|IC_DELIVER|  a reply {\em must}
be sent using \verb|IC_SEND|. It is mandatory because this routine also
finishes the handshaking between master and client. The reply should
be loaded into CADAT.

\end{describe}
\begin{options}
\opt{none}{}
\end{options}
\begin{returncodes}
\rcode{IC\_E\_NONMASTER}{caller is not master of the cache}
\end{returncodes}
\begin{examplecode}\begin{verbatim}
*
**        see example for IC_ASSOCIATE.
*
\end{verbatim}\end{examplecode}
\end{routine}


\begin{routine} %RRRRRRRRRRRRRRRRRRRRRRRRRRRRRRRRRRRRRRRRRRRRRRRRRRRRR
\subroutine
   {SUBROUTINE}{IC\_SIGNAL}{(CANAME, CADAT, TIMEOUT, CHOPT, IERR)}
\begin{overview}
A \mailbox mode routine.

Called by a client to signal the master.

\end{overview}
\begin{argdeflist}
\argitem{CANAME}{character/readonly}{The name assigned to the cache}
\argitem{CADAT}{16 bytes/modify}{The message to be sent and (on return)
  the reply received}
\argitem{TIMEOUT}{real/readonly}{Timeout in seconds. {\tt IC\_SIGNAL} exits
  after this time if a reply has not yet been received from the master.
  If zero, there is no timeout (infinite wait).}
\argitem{CHOPT}{character/readonly}{Option string, upper case}
\argitem{IERR}{integer/modify}{Error code. Error if nonzero}
\end{argdeflist}
\begin{describe}

A \mailbox mode client calls \verb|IC_SIGNAL| to send a message to the master of
the specified cache and receive a reply. 

\verb|IC_SIGNAL| is called
with the message to be delivered to the master in CADAT. The master
receives an interrupt, reads the message from the client with a call
to \verb!IC_DELIVER!, and then sends a reply with \verb!IC_SEND!.
After the reply is received by the client, \verb|IC_SIGNAL| returns
with the reply in CADAT.

The TIMEOUT parameter if non-zero establishes the time after which
control is returned to the calling process even if the master has not yet
sent a reply.

\end{describe}
\begin{options}
\opt{A}{autobooking. Not yet implemented.}
\end{options}
\begin{returncodes}
\rcode{IC\_E\_INVMODE}{cache is not the correct usage mode.}
\rcode{IC\_E\_EXWRITE}{attempted write while in EX mode (attempt by master to
  signal itself)}
\rcode{IC\_E\_INVTIMOUT}{invalid timeout. System error.}
\end{returncodes}
\end{routine}


\begin{routine} %RRRRRRRRRRRRRRRRRRRRRRRRRRRRRRRRRRRRRRRRRRRRRRRRRRRRR
\subroutine
   {SUBROUTINE}{IC\_STATUS}{(CANAME, CADAT, CHOPT, IERR)}
\begin{overview}
A \watcher mode routine.

Used by lockholders to store status information.
\end{overview}
\begin{argdeflist}
\argitem{CANAME}{character/readonly}{The name assigned to the cache}
\argitem{CADAT}{16 bytes/readonly}{The status information}
\argitem{CHOPT}{character/readonly}{Option string, upper case}
\argitem{IERR}{integer/modify}{Error code. Error if nonzero}
\end{argdeflist}
\begin{describe}

This routine permits a lockholder to store status information in the
lock cache.

This information can be read by interested processes
via \verb|IC_INFO|, and is passed to watcher processes via
\verb|IC_WATCH| when the lockholder crashes or releases the lock.

Note that any process, not just watcher processes,
can by using \verb|IC_INFO| read the status of the lockholder.

The lockholder passes the status info in CADAT.

\end{describe}
\begin{options}
\opt{none}{}
\end{options}
\begin{returncodes}
\rcode{IC\_E\_NONLKHOL}{ calling process is not the lockholder}
\end{returncodes}
\end{routine}

\begin{routine} %RRRRRRRRRRRRRRRRRRRRRRRRRRRRRRRRRRRRRRRRRRRRRRRRRRRRR
\subroutine
   {SUBROUTINE}{IC\_TAKE\_LOCK}{(CANAME, CADAT, IDUM, TIMEOUT, CHOPT, IERR)}
\begin{overview}
Resource lock mode. Take out the specified resource lock.

\end{overview}
\begin{argdeflist}
\argitem{CANAME}{character/readonly}{The resource lock name}
\argitem{CADAT}{8 bytes/readonly}{User defined component of the lock's
    cache}
\argitem{IDUM}{integer/readonly}{Unused. For future use.}
\argitem{TIMEOUT}{real/readonly}{Timeout period, seconds}
\argitem{CHOPT}{character/readonly}{Option string, upper case}
\argitem{IERR}{integer/modify}{Error code. Error if nonzero}
\end{argdeflist}
\begin{describe}

This routine takes out the specified resource lock. If the lock is currently
held by another process, the routine waits for a period TIMEOUT seconds
before returning with a timeout error.

\end{describe}
\begin{options}
\opt{none}{}
\end{options}
\begin{returncodes}
\rcode{IC\_E\_INVMODE}{cache is not the correct usage mode.}
\rcode{IC\_E\_LKHELD}{lock is held by another process, and has not become
   available within the timeout interval.}
\end{returncodes}
\end{routine}

\begin{routine} %RRRRRRRRRRRRRRRRRRRRRRRRRRRRRRRRRRRRRRRRRRRRRRRRRRRRR
\subroutine
   {SUBROUTINE}{IC\_WAIT}{(CANAME, CADAT, TIMEOUT, CHOPT, IERR)}
\begin{overview}
A routine used by interrupt receiving processes in \broadcast,
\alarm, or \mailbox mode. It hibernates the process until an
interrupt is received.

\end{overview}
\begin{argdeflist}
\argitem{CANAME}{character/readonly}{The name assigned to the cache}
\argitem{CADAT}{16 bytes/modify}{The cache contents when the interrupt
  is delivered (the message from the interrupting process)}
\argitem{TIMEOUT}{real/readonly}{Timeout in seconds. {\tt IC\_WAIT} exits
  after this time if a message has not yet been received.
  If zero, there is no timeout (infinite wait).}
\argitem{CHOPT}{character/readonly}{Option string, upper case}
\argitem{IERR}{integer/modify}{Error code. Error if nonzero}
\end{argdeflist}
\begin{describe}

This routine can be used to suspend a process until an interrupt comes
in from the specified cache. The calling process might be waiting for
an alarm in \alarm mode, a broadcast in \broadcast mode, or a message
from a client in \mailbox mode.

When the interrupt is delivered and the process wakes, the contents
of the specified cache at the time of the interrupt (broadcast message,
alarm code, etc.) is returned in CADAT.

The routine \verb|IC_ASSOCIATE| is used more often than \verb|IC_WAIT|,
because \verb|IC_WAIT| can only handle interrupts from one cache, and
it prevents the process from doing anything else while it waits for
an interrupt.

\end{describe}
\begin{options}
\opt{none}{}
\end{options}
\begin{returncodes}
\rcode{IC\_E\_INVMODE}{cache is not the correct usage mode.}
\end{returncodes}
\end{routine}

\begin{routine} %RRRRRRRRRRRRRRRRRRRRRRRRRRRRRRRRRRRRRRRRRRRRRRRRRRRRR
\subroutine
   {SUBROUTINE}{IC\_WATCH}{(CANAME, CADAT, AST, CHOPT, IERR)}
\begin{overview}
A \watcher mode routine.

Establishes a watch on a \watcher cache, such that the caller is
notified when the lockholder process crashes.
\end{overview}
\begin{argdeflist}
\argitem{CANAME}{character/readonly}{The name assigned to the cache.
  Maximum length is 20 characters.}
\argitem{CADAT}{16 bytes/modify}{The cache contents on return, for
option 'W'}
\argitem{AST}{external routine}{AST routine. Used in conjunction with
option {\tt A}.}
\argitem{CHOPT}{character/readonly}{Option string, upper case}
\argitem{IERR}{integer/modify}{Error code. Error if nonzero}
\end{argdeflist}
\begin{describe}

This routine can be used by processes which have a \watcher mode cache
booked in {\tt P} (process watcher) mode. It enables the process to
take action when the cache lock is dropped by the lockholder process
(either because the lockholder has crashed or has released the
lock deliberately at normal termination). The action can either be to
hibernate and wake when the lock is dropped (option {\tt W}),
or establish an AST
routine that is called when the lock is dropped (option {\tt A})
so the watcher process
can do other things in the meantime and can watch several locks.

If the {\tt X} option has been selected, once the wake or AST
is delivered the watcher process becomes the new lockholder. If there are
several watcher processes, either all of them or none of them should use
the {\tt X} option.

Either the {\tt W} or {\tt A} option must be specified or an error will
be returned.

In {\tt A} mode, \verb|IC_DELIVER| can be called inside the AST routine
to read the status stored by the lockholder. In {\tt W} mode, the
status information is available in CADAT upon return.

\end{describe}
\begin{options}
\opt{W}{ Wait for lock release. Process hibernates until
lock is released. Cache at time of release is returned to caller in CADAT}
\opt{A}{ AST routine is established. AST is the name of the routine,
declared EXTERNAL. When lock is released, the AST routine is called.}
\opt{X}{ Exclusive notification. If several processes watch a lock in
X mode, only one of them will be activated when the lock is released.}
\end{options}
\begin{returncodes}
\rcode{IC\_E\_BADWOPT}{ missing an {\tt A} or {\tt W} option selection}
\rcode{IC\_E\_BADWATCH}{ cache mode is not valid for {\tt IC\_WATCH}}
\rcode{IC\_E\_DBLWATCH}{ a call to WATCH from the watched. Illegal.}
\rcode{IC\_E\_NOEXIST}{ cache does not exist (returned only if {\tt X}
  option was selected). No error message is printed.}
\end{returncodes}
\end{routine}

\begin{routine} %RRRRRRRRRRRRRRRRRRRRRRRRRRRRRRRRRRRRRRRRRRRRRRRRRRRRR
\subroutine
   {SUBROUTINE}{IC\_WRITE}{(CANAME, CADAT, CHOPT, IERR)}
\begin{overview}
An \info mode routine.

Writes an information cache.
\end{overview}
\begin{argdeflist}
\argitem{CANAME}{character/readonly}{The name assigned to the cache}
\argitem{CADAT}{16 bytes/readonly}{The cache contents written.}
\argitem{CHOPT}{character/readonly}{Option string, upper case}
\argitem{IERR}{integer/modify}{Error code. Error if nonzero}
\end{argdeflist}
\begin{describe}

Writes CADAT to the specified information cache.

Can be used at any time by any process that has the cache booked.

\end{describe}
\begin{options}
\opt{none}{}
\end{options}
\begin{returncodes}
\rcode{IC\_E\_INVMODE}{cache is not the correct usage mode.}
\end{returncodes}
\end{routine}


%=========================================

\newpage
\part{CPC General Routines Reference}
\vspace*{1cm}
\begin{verbatim}
  List of general CPC routines documented in this section:

  CPC_FINISH    to close out CPC before program exit
\end{verbatim}

\begin{routine} %RRRRRRRRRRRRRRRRRRRRRRRRRRRRRRRRRRRRRRRRRRRRRRRRRRRRR
\subroutine
   {SUBROUTINE}{CPC\_FINISH}{}
\begin{overview}
Closeout routine for CPC.
\end{overview}
\begin{argdeflist}
\argitem{none}{}{}
\end{argdeflist}
\begin{describe}

This routine closes out all CPC activity for \CLUSCOM and \INFOCACHE.
It should always be called just before termination by programs that
use CPC (CLUSCOM or INFOCACHE).

\end{describe}
\end{routine}



%\end{document}


%=========================================
\newpage
\part{Error Handling in CPC}

All CPC routines have as their last call parameter an error code IERR
which is returned from the routine. If zero is returned, there were
no errors. A non-zero value indicates an error. The error codes come
from two sources:
\begin{itemize}
  \item errors defined within CPC are generated by the VMS message
  utility. A FORTRAN include file CPCERR.FOR contains these error
  codes.
  \item error codes from VMS services used within CPC routines are
  propagated up to the IERR parameter.

\end{itemize}

For all serious errors CPC itself gives an error message and traceback
before returning to the user code.

In the reference section of this document, only the CPC specific error
messages which originate within the routine in question. Any errors arising
from system utilities or deeper levels of CPC are reported with a traceback
and an explanatory message. If you get a message number and no message,
you may not be linking properly (you need to include the CPC module in the
library) or it may be a typo or oversight on my part.

%=========================================
\newpage
\part{CPC speed}
INFOCACHE timings for one signal exchange (client sends signal, master reads
and sends reply, client receives reply) in \mailbox mode:

\begin{tabbing}
\hspace*{3.5cm} \= \hspace*{7.0cm} \= \hspace*{2.5cm} \= \kill 
master and client \> both on 8800:                 \>   4.2 msec \\
\>                   both on same VS3200:          \>   10 msec  \\
\>                   both on same VS2000:          \>   33 msec  \\
\>                   both on same VAX750:          \>   38 msec  \\
\>                   on 8800 and on VAX750:        \>   32 msec  \\
\>                   on 8800 and on VS3200:        \>   32 msec  \\
\>                   on 8800 and on VS2000:        \>   40 msec  \\
\end{tabbing}

The speed of \info mode ranges from about 0.4 msec for a read or
write on the 8800 to 12 (27) msec for read (write) over Ethernet (lock
resident on 8800, and read/written from a VAXstation).

The speed of \broadcast and \alarm modes are intermediate between
those of \mailbox and \info, except that at present there is
a delay of about 100 msec hardwired in {\tt IC\_BROADCAST} and
{\tt IC\_ALARM}. This delay will be removed or reduced in a future
version. % $$$

For CLUSCOM, the time for an update check with no read from disk is
the same as an \info mode read; about .4 msec on the 8800.
A disk read/write currently takes about 17 msec for 1 block and 50 msec
for 64 blocks on the 8800.
% $$$ timings with Mike's IO

%=========================================
\newpage
\part{Building CPC Applications}

The CPC material on the L3 online cluster is maintained in a directory
pointed to by the logical name \verb|CPC$DIR|. This directory contains:

\newcommand{\cpcfile}[2]{ \tt #1 \> \begin{minipage}[t]{9.7cm}{\rm #2}
  \end{minipage} \\ \\ }
\begin{tabbing}
\hspace*{5.0cm} \= \hspace*{5.0cm} \= \hspace*{2.5cm} \= \kill 

\noindent
\cpcfile{CPC.OLB}{a library containing all CPC routines 
  (CLUSCOM and INFOCACHE)}
\cpcfile{CPCERR.MSG}{VMS MESSAGE source file for CPC error messages}
\cpcfile{CPCERR.FOR}{error message include file}
\cpcfile{CLUSCOM.FOR}{CLUSCOM source}
\cpcfile{CLUSCOM.INC}{CLUSCOM include file. Used in the source; not to
  be used by user code.}
\cpcfile{INFOCACHE.FOR}{INFOCACHE source}
\cpcfile{INFOCACHE.INC}{INFOCACHE include file. Used in the source; not to
  be used by user code.}
\cpcfile{INFOCACHE.PARS}{INFOCACHE parameters. Discussed in the technical
  section.}
\cpcfile{CPCUTIL.FOR}{CPC common routines and utility routines}
\end{tabbing}

To link a CPC application which uses either INFOCACHE or CLUSCOM (or both),
use
\begin{verbatim}
   $ LINK <mycode>,CPC$DIR:CPC/LIB/INCL=CPC
\end{verbatim}

The \verb|"/INCL=CPC"| is necessary to include the error messages.

The library \verb|CPC.OLB| always contains the current version. Older versions
are maintained (for a while) as, for example, \verb|CPC10.OLB| for
version 1.0.

Please report quickly to the author any problems with CPC. Suggestions
for changes or additions are very welcome.

If you want to be informed of CPC news, bugs, documentation oversights, etc.,
please send me a computer message at
{\tt VXL3ON::WENAUS}.

%=========================================
\newpage
\part{Programming examples}
\section{CLUSCOM example}

\begin{verbatim}
        COMMON /CCTEST/ IFIRST, IARRAY(10), STRING, ILAST
*         force page alignment
CDEC$ PSECT /CCTEST/ ALIGN=9
        CHARACTER COMID*6 /'CCTEST'/, FNAME*60, STRING*10
*
        FNAME = 'CCTEST.MAPCOM'
*         set up the CLUSCOM section, using global section mode
        CALL CC_SETUP(COMID,FNAME,IFIRST,ILAST,'G',IERR)
        IF (IERR.NE.0) GOTO 999
        TYPE '(10I8)',IARRAY
*         get write lock, make changes to common, and release
        CALL CC_WRITELOCK(COMID,' ',IERR)
        IF (IERR.NE.0) GOTO 999
        DO I=1,10
          IARRAY(I)=IARRAY(I)+I
        ENDDO
        CALL CC_RELEASE(COMID,' ',IERR)
        IF (IERR.NE.0) GOTO 999
*         routines to change a single parameter
        CALL CC_SAVEC(STRING,'TEST',' ',IERR)
        IF (IERR.NE.0) GOTO 999
        CALL CC_SAVE(IARRAY(3),999+IARRAY(1),' ',IERR)
        IF (IERR.NE.0) GOTO 999
*         force an update from disk
        CALL CC_UPDATE(COMID,'F',IERR)
        IF (IERR.NE.0) GOTO 999
        TYPE *,'STRING: "',STRING,'"'
        TYPE '(10I8)',IARRAY
*         retrieve info in the common periodically, eg. to refresh
*         a screen display. Note that in CC_UPDATE the disk file is
*         read only if a change has been made, so it is not
*         inefficient to call CC_UPDATE within a loop like this.
        DO I=1,10
          CALL CC_UPDATE(COMID,' ',IERR)
          IF (IERR.NE.0) GOTO 999
          !<refresh display or whatever>
          CALL LIB$WAIT(5.)
        ENDDO
*         mandatory closeout routine
999     CALL CPC_FINISH
        END
\end{verbatim}


\section{INFOCACHE example}
\begin{verbatim}
*
*       'receiver' test program. Run this first, as a subprocess.
*
        EXTERNAL MYAST
        INTEGER LVB(4)
*         information cache
        CALL IC_BOOK('INFO',LVB,' ',IERR)
        IF (IERR.NE.0) GOTO 999
        CALL IC_READ('INFO',LVB,' ',IERR)
        IF (IERR.NE.0) GOTO 999
        DO I=1,4
          LVB(I)=LVB(I)+I**2
        ENDDO
        CALL IC_WRITE('INFO',LVB,' ',IERR)
        IF (IERR.NE.0) GOTO 999
*         alarm cache
        CALL IC_BOOK('ALARM',LVB,'S',IERR)
        IF (IERR.NE.0) GOTO 999
        CALL IC_ASSOCIATE('ALARM',LVB,MYAST,' ',IERR)
        IF (IERR.NE.0) GOTO 999
*         broadcast cache
        CALL IC_BOOK('BCAST',LVB,'B',IERR)
        IF (IERR.NE.0) GOTO 999
        CALL IC_ASSOCIATE('BCAST',LVB,MYAST,' ',IERR)
        IF (IERR.NE.0) GOTO 999
*         mailbox cache
        CALL IC_BOOK('MBX',LVB,'M',IERR)
        IF (IERR.NE.0) GOTO 999
        CALL IC_ASSOCIATE('MBX',LVB,MYAST,' ',IERR)
        IF (IERR.NE.0) GOTO 999
        type *,'receiver waiting'
        CALL LIB$WAIT(999.)
999     CALL CPC_FINISH
        END
*
        SUBROUTINE MYAST(NUM)
        INCLUDE 'INFOCACHE.PARS'
        INTEGER LVB(4)
        CHARACTER NAME*20
        CALL IC_NUM_ID(NUM,NAME,' ',IERR)
        IF (IERR.NE.0) GOTO 999
        CALL IC_DELIVER(NAME,LVB,' ',IERR)
        IF (IERR.NE.0) GOTO 999
        TYPE *,'RECEIVER: CACHE '//NAME//' RECEIVED',LVB
        CALL IC_INFO(NAME,IMODE,IDUM,'M',IERR)
        IF (IERR.NE.0) GOTO 999
        IF (IMODE.EQ.IC__MAILBOX) THEN
          TYPE *,'MAILBOX CACHE '//NAME//' -- SENDING REPLY'
          DO I=1,4
            LVB(I)=-I
          ENDDO
          CALL IC_SEND(NAME,LVB,' ',IERR)
          IF (IERR.NE.0) GOTO 999
        ENDIF
999     END
************************************************************************
*
*       'sender' test program. Run this after starting the receiver.
*
        EXTERNAL MYAST
        INTEGER LVB(4)
*         information cache
        CALL IC_BOOK('INFO',LVB,' ',IERR)
        IF (IERR.NE.0) GOTO 999
        TYPE *,'READ FROM INFO CACHE:',LVB
        DO I=1,4
          LVB(I)=LVB(I)+I**2
        ENDDO
        CALL IC_WRITE('INFO',LVB,' ',IERR)
        IF (IERR.NE.0) GOTO 999
*         alarm cache
        CALL IC_BOOK('ALARM',LVB,'T',IERR)
        IF (IERR.NE.0) GOTO 999
        TYPE *,LVB
        DO I=1,4
          LVB(I)=LVB(I)+I**2
        ENDDO
        CALL IC_ALARM('ALARM',LVB,' ',IERR)
        IF (IERR.NE.0) GOTO 999
*         broadcast cache
        CALL IC_BOOK('BCAST',LVB,'B',IERR)
        IF (IERR.NE.0) GOTO 999
        CALL IC_ASSOCIATE('BCAST',LVB,MYAST,' ',IERR)
        IF (IERR.NE.0) GOTO 999
        TYPE *,LVB
        DO I=1,4
          LVB(I)=LVB(I)+I**2
        ENDDO
        CALL IC_BROADCAST('BCAST',LVB,' ',IERR)
        IF (IERR.NE.0) GOTO 999
*         mailbox cache
        CALL IC_BOOK('MBX',LVB,'C',IERR)
        IF (IERR.NE.0) GOTO 999
        DO I=1,4
          LVB(I)=LVB(I)+I**2
        ENDDO
        CALL IC_SIGNAL('MBX',LVB,1.,' ',IERR)
        IF (IERR.NE.0) GOTO 999
	TYPE *,'Reply from signal = ',LVB
999     CALL CPC_FINISH
        END
*
        SUBROUTINE MYAST(NUM)
        INTEGER LVB*4
        CHARACTER NAME*20
        CALL IC_NUM_ID(NUM,NAME,' ',IERR)
        IF (IERR.NE.0) GOTO 999
        CALL IC_DELIVER(NAME,LVB,' ',IERR)
        IF (IERR.NE.0) GOTO 999
        TYPE *,'SENDER CACHE '//NAME//' RECEIVED',LVB
999     END
\end{verbatim}

\end{document}
\bye
